% Copyright 2002-2024 The University of Maryland Baltimore County (UMBC)
% 1000 Hilltop Circle, Baltimore, Maryland, 21250, USA
% https://www.csee.umbc.edu/

\documentclass[graphics]{beamer}
\usepackage{graphicx}
\usepackage{listings} % Syntax highlighing
\usepackage{fancyvrb} % Inline verbatim
\usepackage{hyperref} % Hyperlinks
\hypersetup{pdfpagemode=FullScreen}

\usepackage[normalem]{ulem}               % to striketrhourhg text
\newcommand\redout{\bgroup\markoverwith
{\textcolor{red}{\rule[0.5ex]{2pt}{0.8pt}}}\ULon}

% header in tables
\newcommand*{\thead}[1]{\multicolumn{1}{c}{\bfseries #1}}

% used for arrows from one point in the slide to another
\usepackage{tikz}
\usetikzlibrary{arrows,shapes,tikzmark}

\usetheme{Boadilla}
\title{Lecture 10: While Loops}
\author{UMBC CMSC 104}
\date{Spring 2024}

\begin{document}

\begin{frame}{}
\centering
    While Loops
\end{frame}

\frame{\tableofcontents}

\section{Review}
\begin{frame}{Review: Repetition Structure}
    \begin{itemize}
        \item A \textbf{repetition structure} allows the programmer to specify that an action is to be repeated while some condition remains true.
        \begin{itemize}
            \item The condition be a single expression or several expressions joined with an \texttt{and} \&\& statement or an \texttt{or} \textbar\textbar statement(s). 
        \end{itemize}
        \item There are three repetition structures in C:
        \begin{enumerate}
            \item \texttt{while}: runs for zero or many times
            \item \texttt{for}: runs for a predetermined number of times
            \item \texttt{do-while}: runs for one or many times
        \end{enumerate}
    \end{itemize}
\end{frame}

\section{The While Loop}
\begin{frame}[fragile]{The while Repetition Structure}
\begin{verbatim}
while( condition ) {
    statement(s)
}
\end{verbatim}
The braces are not required if the loop body contains only a single statement. However, as with \texttt{if} statements, they are a good idea and are required by the CMSC 104 C Coding Standards.
\end{frame}

\begin{frame}[fragile]{Example - Remember This?}
\begin{verbatim}
while( children > 0 ) {
    children = children - 1;
    cookies = cookies * 2;
}
\end{verbatim}
\end{frame}

\begin{frame}{Good Programming Practice}
    \begin{itemize}
        \item Always place braces around the body of a \texttt{while} loop.
        \item Advantages:
        \begin{itemize}
            \item Easier to read
            \item You won't forget to add the braces if you have to add statements to the loop body
            \item Reduces the chance of a semantic error
        \end{itemize}
        \item Indent the body of a while loop by 3 or 4 spaces -- be consistent!
    \end{itemize}
\end{frame}

\begin{frame}[fragile]{Example - Getting Intended Input}
\begin{verbatim}
while( input < 0 )
    scanf("%d", &input);

printf("Finally, got something positive\n");
\end{verbatim}

\footnotesize What's missing here from the coding standards?
\end{frame}

\begin{frame}[fragile]{Example - Getting Intended Input (More Clear)}
\begin{verbatim}
while( input < 0 )
    printf("Enter a positive number: ");
    scanf("%d", &input);

printf("Finally, got something positive\n");
\end{verbatim}

\footnotesize What's missing here from the coding standards?
\end{frame}

\begin{frame}[fragile]{Closer to CMSC104 Coding Standards}
\begin{verbatim}
#include <stdio.h>
int main() {
    int number;
    printf("Enter a positive number: ");
    scanf("%d", &number);
    while( number < 0 ) {
        printf("\nThat's incorrect. Try again.\n")
        printf("Enter a positive number: ");
        scanf("%d", &input);
    }

    printf("You entered: %d\n", number);
    return 0;
}
\end{verbatim}
\end{frame}

\begin{frame}{Another while Loop Example}
    \begin{itemize}
        \item \underline{Problem:} Write a program which calculates the average exam grade for a class of ten students.
        \item What are the program inputs?
        \begin{itemize}
            \item the exam grades
        \end{itemize}
        \item What are the program outputs?
        \begin{itemize}
            \item the average exam grade
        \end{itemize}
    \end{itemize}
\end{frame}

\begin{frame}[fragile]{Pseudocode}
\begin{verbatim}
<total> = 0
<grade_counter> = 1
While( <grade_counter> <= 10 )
    Display "Enter a grade: "
    Read <grade>
    <total> = <total> + <grade>
End_While
<average> = <total> / 10
Display "Class average is: ", <average>
\end{verbatim}
\end{frame}

\begin{frame}[fragile]{C Code}
\begin{lstlisting}[language=C,basicstyle=\footnotesize,keywordstyle=\color{blue},commentstyle=\color{green},showstringspaces=false,stringstyle=\color{red}]
#include <stdio.h>
int main() {
    int counter = 0, grade, total = 0;
    float average;
    while(counter <= 10) {
        printf("Enter a grade: ");
        scanf("%d", &grade);
        total = total + grade;
        counter = counter + 1;
    }
    average = total / 10.0;
    printf("Class average: %1.2f\n", average);
    return 0;
}
\end{lstlisting}
\end{frame}

\section{Program Versatility}
\begin{frame}{Versatile?}
    \begin{itemize}
        \item How versatile is this program?
        \item It only works with class sizes of 10.
        \item We would like it to work with any class size.
        \item Better ways
        \begin{itemize}
            \item Ask the user for the number of students in the class, and use that as the condition in the while loop, and to compute the average.
        \end{itemize}
    \end{itemize}
\end{frame}

\begin{frame}[fragile]{New Pseudocode}
\begin{verbatim}
<total> = 0
<grade_counter> = 1
Display "Enter the number of students:"
Read <num_students>
While( <grade_counter> <= <num_students> )
    Display "Enter a grade: "
    Read <grade>
    <total> = <total> + <grade>
End_While
<average> = <total> / <num_students>
Display "Class average is: ", <average>
\end{verbatim}
\end{frame}

\begin{frame}[fragile]{New C Code}
\begin{lstlisting}[language=C,basicstyle=\footnotesize,keywordstyle=\color{blue},commentstyle=\color{green},showstringspaces=false,stringstyle=\color{red}]
#include <stdio.h>
int main() {
    int num_students, counter = 0, grade, total = 0;
    float average;
    printf("Enter the number of students: ");
    scanf("%d", &num_students);
    while(counter <= num_students) {
        printf("Enter a grade: ");
        scanf("%d", &grade);
        total = total + grade;
        counter = counter + 1;
    }
    // We can convert num_students to a float to get a
    // proper floating point average result!
    average = total / float(num_students);
    printf("Class average: %1.2f\n", average);
    return 0;
}
\end{lstlisting}
\end{frame}

\begin{frame}{Why Bother to Make it Easier?}
    \begin{itemize}
        \item Why do we write programs?
        \begin{itemize}
            \item So the user can perform some task
        \end{itemize}
        \item The more versatile the program, the more difficult it is to write. But it is more \underline{usable}!
        \item The more complex the ask, the more difficult it is to write. But that is often what the user needs.
        \item Always consider the user, don't assume anything on their part.
    \end{itemize}
\end{frame}

\subsection{Sentinel Value \& Priming}
\begin{frame}{Using a Sentinel Value}
    \begin{itemize}
        \item Instead of asking the user for the number of students, we could allow the user to continually enter grades, using a special value to indicate that this task is finished.
        \item This special value is called the \textbf{sentinel value}.
        \item We have to make sure the value chosen for the sentinel value isn't a legal value. For example, we can't use 0 as the sentinel value because it's possible that a student earned a zero on the exam.
    \end{itemize}
\end{frame}

\begin{frame}{The Priming Read}
    \begin{itemize}
        \item When we use a sentinel value to control a while loop, we have to get the first value from the user before we encounter the loop so that it will be tested and the loop can be entered.
        \item This is known as the \textbf{priming read}.
        \item We have to give significant thought to the initialization of variables, the sentinel value, and getting into the loop.
    \end{itemize}
\end{frame}

\begin{frame}[fragile]{New Pseudocode}
\begin{verbatim}
<total> = 0
<grade_counter> = 0
Display "Enter a grade:"
Read <grade>
While( <grade> >= 0 )
    <total> = <total> + <grade>
    <grade_counter> = <grade_counter> + 1
    Display "Enter a grade: "
    Read <grade>
End_While
<average> = <total> / <grade_counter>
Display "Class average is: ", <average>
\end{verbatim}
\end{frame}

\begin{frame}[fragile]{New C Code}
\begin{lstlisting}[language=C,basicstyle=\footnotesize,keywordstyle=\color{blue},commentstyle=\color{green},showstringspaces=false,stringstyle=\color{red}]
#include <stdio.h>
int main() {
    int counter = 0, grade, total = 0;
    float average;
    printf("Enter a grade: ");
    scanf("%d", &grade);
    // Loop while grade values aren't negative
    while(grade >= 0) {
        total = total + grade;
        counter = counter + 1;
        printf("Enter a grade: ");
        scanf("%d", &grade);
    }
    average = total / float(counter);
    printf("Class average: %1.2f\n", average);
    return 0;
}
\end{lstlisting}
\end{frame}

\begin{frame}[fragile]{New C Code - Better}
\begin{lstlisting}[language=C,basicstyle=\footnotesize,keywordstyle=\color{blue},commentstyle=\color{green},showstringspaces=false,stringstyle=\color{red}]
#include <stdio.h>
int main() {
    int counter = 0, grade, total = 0;
    float average;
    printf("Enter a grade: ");
    scanf("%d", &grade);
    // Loop while grade values aren't negative
    while(grade >= 0) {
        total = total + grade;
        counter = counter + 1;
        printf("Enter a grade: ");
        scanf("%d", &grade);
    }
    if (counter == 0) { // Avoid a divide by zero error
        printf("No grades entered.\n");
    } else {
        average = total / (float) counter;
        printf("Class average for %d students: %1.2f\n",
            counter, average);
    }
    return 0;
}
\end{lstlisting}
\end{frame}

\end{document}