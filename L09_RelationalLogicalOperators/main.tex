\documentclass[graphics]{beamer}
\usepackage{graphicx}
\usepackage{listings} % Syntax highlighing
\usepackage{fancyvrb} % Inline verbatim
\usepackage{hyperref} % Hyperlinks
\hypersetup{pdfpagemode=FullScreen}

\usepackage[normalem]{ulem}               % to striketrhourhg text
\newcommand\redout{\bgroup\markoverwith
{\textcolor{red}{\rule[0.5ex]{2pt}{0.8pt}}}\ULon}

% header in tables
\newcommand*{\thead}[1]{\multicolumn{1}{c}{\bfseries #1}}

% used for arrows from one point in the slide to another
\usepackage{tikz}
\usetikzlibrary{arrows,shapes,tikzmark}

\usetheme{Boadilla}
\title{Lecture 9: Relational \& Logical Operators}
\author{UMBC CMSC 104}
\date{Th 30 Sept 2021}

\begin{document}

\begin{frame}{}
\centering
    Relational \& Logical Operators
\end{frame}

\begin{frame}{Relational \& Logical Operators}
    Topics:
    \begin{itemize}
        \item Relational Operators \& Expressions
        \item The \texttt{if} Statement
        \item The \texttt{if}-\texttt{else} Statement
        \item Nesting \texttt{if}-\texttt{else} Statements
        \item Logical Operators \& Expressions
        \item Truth Tables
    \end{itemize}
\end{frame}

\begin{frame}{Relational Operators}
~~ ~~ ~~ ~~ ~~ ~~ ~~ ~~ ~~ ~~ ~~  \begin{tabular}{l l}
        $<$ & less than \\
        $>$ & greater than \\
        $<=$ & less than or equal to \\
        $>=$ & greater than or equal to \\
        $==$ & is equal to \\
        $!=$ & is not equal to
    \end{tabular} \\ ~~ \\
    \textbf{Relational expressions} evaluate to the inter values 1 (true) or 0 (false). \\ ~~ \\
    All of these operators are called \textbf{binary operators} because they take two expressions as \textbf{operands}.
\end{frame}

\begin{frame}{Practice with Relational Expressions}
    \texttt{int a = 1, b = 2, c = 3;} \\ ~~ \\
    \begin{tabular}{l l l l}
        Expression & Value & Expression & Value \\ \hline \\
        $a < c$    &       & $a + b >= c$ &     \\
        $b <= c$   &       & $a + b == c$ &     \\
        $c <= a$   &       & $a != b$     &     \\
        $a > b$    &       & $a + b != c$ &     \\
        $b >= c$   &       &              &
    \end{tabular}
\end{frame}

\begin{frame}{Arithmetic Expressions: True of False}
    \begin{itemize}
        \item \textbf{Arithmetic expressions} evaluate to numeric values.
        \item An arithmetic expression that has a value of zero is false.
        \item An arithmetic expression that has a \underline{value other than zero} is true.
    \end{itemize}
\end{frame}

\begin{frame}{Practice with Arithmetic Expressions}
    \texttt{int a = 1, b = 2, c = 3;} \\
    \texttt{float x = 3.33, y = 6.66;} \\ ~~ \\
    \begin{tabular}{l l l}
        Expression & Numeric Value & True/False  \\ \hline \\
        $a+b$ & \\
        $b-2*a$ & \\
        $c-b-a$ & \\
        $c-a$ & \\
        $y-x$ & \\
        $y-2*x$ & \\
    \end{tabular}
\end{frame}

\begin{frame}{Review: Structured Programming}
    All programs can be written in terms of only three control structures:
    \begin{enumerate}
        \item \textbf{Sequence}: The statements are executed in the order in which they're written.
        \item \textbf{Selection}: Used to choose among alternative courses of action (also known as \textbf{branching}).
        \item \textbf{Repetition}: Statements are repeated while some condition remains true.
    \end{enumerate}
\end{frame}

\begin{frame}[fragile]{Selection: the if statement}
\begin{verbatim}
    if ( condition ) {
        statement(s);  /* body of the if statement */
    }
\end{verbatim}
    The braces are not required if the body contains only one statement. However, they are a good idea and are required by the CMSC104 C coding standards.
\end{frame}

\begin{frame}[fragile]{Examples}
\begin{verbatim}
if (age >= 18) {
    printf("Go vote!\n");
}
if (value == 0) {
    printf("The value is zero.\n");
}
\end{verbatim}
\end{frame}

\begin{frame}[fragile]{Selection: The if-else statement}
\begin{verbatim}
if ( condition ) {
    statement(s); /* the if clause */
} else {
    statements(s); /* the else clause */
}
\end{verbatim}
Note that there is no condition for the else.
\end{frame}

\begin{frame}[fragile]{Example}
\begin{verbatim}
if (age >= 18) {
    printf("Go Vote!\n");
} else {
    printf("Still too young to vote, maybe next time.\n");
}
\end{verbatim}
\end{frame}

\begin{frame}[fragile]{Another Example}
\begin{verbatim}
if (value == 0) {
    printf("The value is zero.\n");
} else {
    printf("Value = %d\n", value);
}
\end{verbatim}
\end{frame}

\begin{frame}{Good Programming Practice}
    \begin{itemize}
        \item Always place braces around the body of an \texttt{if} statement, and around the \texttt{else} statement if present.
        \item Advantages:
        \begin{itemize}
            \item Easier to read
            \item Will not forget to add braces if you have to add more statements later.
            \item Less likely to make an semantic error.
        \end{itemize}
        \item Indent the body of the \texttt{if} statement 3 to 4 spaces. Be consistent!
    \end{itemize}
\end{frame}

\begin{frame}[fragile]{Nesting of if-else Statements}
\begin{verbatim}
if ( condition1 ) {
    statement(s);
} else if (condition 2) {
    different statement(s);
} else if .... {
    /* there can be any number of if-else statements */
} else {
    statement(s); /* the default case */
}
\end{verbatim}
\end{frame}

\begin{frame}[fragile]{Example}
\begin{verbatim}
if (value == 0) {
    printf("The value is zero.\n");
} else if (value < 0) {
    printf("%d is negative.\n", value);
} else {
    printf("%d is positive.\n", value);
}
\end{verbatim}
\end{frame}

\begin{frame}[fragile]{Gotcha! = vs. ==}
\begin{verbatim}
int a = 2;

if (a = 1) { /* semantic error! */
    printf("a is one\n");
} else if (a == 2) {
    printf("a is two\n");
} else {
    printf("a is %d\n", a);
}
\end{verbatim}
\end{frame}

\begin{frame}{= vs. ==}
    \begin{itemize}
        \item The statement \texttt{if (a = 1)} is syntactically correct, so no error message will be produced, though some compilers might produce a warning. However, it is a semantic (logic) error.
        \item An assignment expression has a value, the value being assigned. In this case, the value being assigned is 1, which is true.
        \item If the value being assigned was 0, then the expression would evaluate to 0, which is false.
        \item This is a very common error. If your if-else structure always executed the same, look for this mistake.
    \end{itemize}
\end{frame}

\begin{frame}{Logical Operators}
    \begin{itemize}
        \item So far, we have seen only \textbf{simple conditions}, like \texttt{if (count > 10)...}.
        \item Sometimes we need to test multiple conditions in order to make a decision.
        \item \textbf{Logical operators} are used for combining simple conditions to make \textbf{complex conditions}.
    \end{itemize}
    \begin{tabular}{l l l}
        \&\& & AND & \textbf{if (x $>$ 5 \&\& y $<$ 6)}  \\
        \textbar\textbar & OR & \textbf{if (z == 0 \textbar\textbar ~~ x $>$ 10)} \\
        ! & NOT & \textbf{if (!(bob $>$ 42))}
    \end{tabular}
\end{frame}

\begin{frame}[fragile]{Examples}
\begin{verbatim}
if (age < 1.0 && gender == 'f') {
    printf("You have a baby girl!\n");
}

if (grade == 'D' || grade == 'F') {
    printf("See you next semester!\n");
}

if (!(x==2)) { /* same as (x!=2) */
    printf("x is not equal to 2.\n");
} else {
    printf("x is equal to 2.\n");
}
\end{verbatim}
\end{frame}

\begin{frame}{Truth Table for \&\&}
    \begin{tabular}{l l l}
        Expression 1 & Expression 2 & Expression 1 \&\& Expression 2  \\
        0 & 0 & 0 \\
        0 & non-zero & 0 \\
        non-zero & 0 & 0 \\
        non-zero & non-zero & 1
    \end{tabular} \\ ~~ \\
    $Expression_1$ \&\& $Expression_2$ \&\& $Expression_N$ will evaluate to 1 (true) only if all \textbf{subconditions} are true.
\end{frame}

\begin{frame}{Truth Table for \textbar\textbar}
    \begin{tabular}{l l l}
        Expression 1 & Expression 2 & Expression 1 \textbar\textbar Expression 2  \\
        0 & 0 & 0 \\
        0 & non-zero & 1 \\
        non-zero & 0 & 1 \\
        non-zero & non-zero & 1
    \end{tabular} \\ ~~ \\
    $Expression_1$ \textbar\textbar $Expression_2$ \textbar\textbar $Expression_N$ will evaluate to 1 (true) if one (or more) \textbf{subconditions} are true.
\end{frame}

\begin{frame}{Truth Table for !}
    \begin{tabular}{l l}
        Expression & !Expression \\
        0 & 1 \\
        non-zero & 0
    \end{tabular}
\end{frame}

\begin{frame}{Operator Precedence \& Associativity}
    \begin{tabular}{l l}
        Precedence & Associativity \\ \hline
        ( ) & left to right/inside-out \\
        * / \% & left to right \\
        + (addition) - (subtraction) & left to right \\
        $<$ $<=$ $>$ $>=$ & left to right \\
        == != & left to right \\
        \&\& & left to right \\
        \textbar\textbar & left to right \\
        $=$ & right to left
    \end{tabular}
\end{frame}

\begin{frame}{Some Practice Expressions}
    \texttt{int a = 1, b = 0, c = 7;} \\ ~~ \\
    \begin{tabular}{l l l}
        Expression & Numeric Value & True/False \\ \hline
        a & \\
        b & \\
        c & \\
        $a+b$ & \\
        a \&\& b & \\
        a \textbar\textbar b & \\
        !c & \\
        !!c & \\
        a \&\& !b & \\
        $a < b$ \&\& $b<c$ & \\
        $a>b$ \&\& $b<c$ & \\
        $a>=b$ \textbar\textbar $b>c$ &
    \end{tabular}
\end{frame}

\begin{frame}{More Practice}
    Given: \texttt{int a = 5, b = 7, c = 17;} evaluate as true or false:
    \begin{itemize}
        \item $c / b == 2$
        \item $c \% b <= a \% b$
        \item $b + c / a != c - a$
        \item $(b<c)$ \&\& $(c == 7)$
        \item $(c + 1 - b == 0$ \textbar\textbar $(b = 5)$
    \end{itemize}
\end{frame}

\end{document}