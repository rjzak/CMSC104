% Copyright 2002-2023 The University of Maryland Baltimore County (UMBC) and Richard J. Zak.
% 1000 Hilltop Circle, Baltimore, Maryland, 21250, USA
% https://www.csee.umbc.edu/
% richard.j.zak@gmail.com

\documentclass[graphics]{beamer}
\usepackage{graphicx}
\usepackage{listings} % Syntax highlighing
\usepackage{fancyvrb} % Inline verbatim
\usepackage{hyperref} % Hyperlinks
\hypersetup{pdfpagemode=FullScreen}

\usepackage[normalem]{ulem}               % to striketrhourhg text
\newcommand\redout{\bgroup\markoverwith
{\textcolor{red}{\rule[0.5ex]{2pt}{0.8pt}}}\ULon}

% header in tables
\newcommand*{\thead}[1]{\multicolumn{1}{c}{\bfseries #1}}

% used for arrows from one point in the slide to another
\usepackage{tikz}
\usetikzlibrary{arrows,shapes,tikzmark}

\usetheme{Boadilla}
\title{Structs}
\author{UMBC CMSC 104}
\date{Th 11 November 2021}

\begin{document}

\begin{frame}{}
\centering
    Custom Types
\end{frame}

\frame{\tableofcontents}

\section{Intro}
\begin{frame}{Intro}
    \begin{itemize}
        \item Up until now, we've only looked at individual variables (except strings, which are arrays).
        \item We can make our own data types!
        \begin{itemize}
            \item Create a data type to model something
            \item Return more than one variable from a function
            \item Organize multiple related variables and keep track of them easily
            \item Create other simple types.
        \end{itemize}
    \end{itemize}
\end{frame}

\section{Structs}
\begin{frame}[fragile]{Structs Example}
    Structures (structs) are a \textbf{derived data type}, since they contain existing data types.
    \begin{lstlisting}[language=C,basicstyle=\footnotesize,keywordstyle=\color{blue},commentstyle=\color{green},showstringspaces=false,stringstyle=\color{red}]
#include<stdio.h>

// define structs before main()
struct Dog { // struct name
    char *breed; // fields that are part of the struct
    unsigned int age;
    char *name;
    float weight;
};

int main() {
    struct Dog my_dog; // creating a variable requires "struct"
    my_dog.breed = "German Shepherd";
    my_dog.age = 1;
    my_dog.name = "Bruno";
    my_dog.weight = 80.23;
    return 0;
}
    \end{lstlisting}
\end{frame}

\begin{frame}[fragile]{Example (Cont'd)}
    Functions can receive a struct as a parameter:
    \begin{lstlisting}[language=C,basicstyle=\footnotesize,keywordstyle=\color{blue},commentstyle=\color{green},showstringspaces=false,stringstyle=\color{red}]
// function prototype
void print_dog(struct Dog the_dog);

// function body
void print_dog(struct Dog the_dog) {
    printf("%s weighs %1.2f pounds\n", the_dog.name,
        the_dog.weight);
}
    \end{lstlisting}
\end{frame}

\begin{frame}[fragile]{Example}
    Functions can return a struct:
    \begin{lstlisting}[language=C,basicstyle=\footnotesize,keywordstyle=\color{blue},commentstyle=\color{green},showstringspaces=false,stringstyle=\color{red}]
// function prototype
struct Dog get_dog();

// function body
struct Dog get_dog() {
    struct Dog a_dog;
    a_dog.name = "Fido";
    a_dog.weight = 20;
    return a_dog;
}
    \end{lstlisting}
\end{frame}

\begin{frame}[fragile]{Nested structs!}
    Just like with loops and if statements, we can have nested structs.
    \begin{lstlisting}[language=C,basicstyle=\footnotesize,keywordstyle=\color{blue},commentstyle=\color{green},showstringspaces=false,stringstyle=\color{red}]
struct PetType {
    char *typeName;
    unsigned int num_legs;
    int is_mammal;
};

struct Pet {
    char *name;
    unsigned int age;
    struct PetType type;
};

struct Pet my_pet;
my_pet.name = "Spidey";
my_pet.age = 1;
my_pet.type.typeName = "tarantula";
my_pet.type.num_legs = 8;
my_pet.type.is_mammal = 0;
printf("%s has %d legs.\n", my_pet.name, my_pet.type.num_legs);
    \end{lstlisting}
\end{frame}

\subsection{typedef}
\begin{frame}[fragile]{The typedef}
    Using \texttt{typedef}, we can make the code slightly less verbose.
    \begin{lstlisting}[language=C,basicstyle=\footnotesize,keywordstyle=\color{blue},commentstyle=\color{green},showstringspaces=false,stringstyle=\color{red}]
typedef struct {
    char *breed; // fields that are part of the struct
    unsigned int age;
    char *name;
    float weight;
} Dog; // struct name

Dog my_dog; // no longer use "struct"
my_dog.breed = "German Shepherd";
my_dog.age = 1;
my_dog.name = "Bruno";
my_dog.weight = 80.23;
    \end{lstlisting}
\end{frame}

\section{Enumerations}
\begin{frame}{Enumerations}
    \begin{itemize}
        \item Enumerations (\texttt{enum}) allows creating a simple type which only allows pre-defined values.
        \item Ideal for use with flag variables.
        \begin{itemize}
            \item Helps make code more readable.
        \end{itemize}
    \end{itemize}
\end{frame}

\begin{frame}[fragile]{Example}
    Enums are treated as integers.
    \begin{lstlisting}[language=C,basicstyle=\footnotesize,keywordstyle=\color{blue},commentstyle=\color{green},showstringspaces=false,stringstyle=\color{red}]
#include<stdio.h>

enum Color {
        Red,
        Blue,
        Orange,
        Yellow,
        Green
};

int main() {
        enum Color mycolor;
        mycolor = Red;
        printf("Red: %d\n", mycolor);
        return 0;
}
    \end{lstlisting}
    Output: \texttt{Red: 0}
\end{frame}

\begin{frame}[fragile]{Example}
    Custom integer values can be defined for all or some of the possible values.
    \begin{lstlisting}[language=C,basicstyle=\footnotesize,keywordstyle=\color{blue},commentstyle=\color{green},showstringspaces=false,stringstyle=\color{red}]
#include<stdio.h>

enum Color {
        Red = 72,
        Blue = 34,
        Orange = 0,
        Yellow = 99,
        Green = -1
};

int main() {
        enum Color mycolor;
        mycolor = Red;
        printf("Red: %d\n", mycolor);
        return 0;
}
    \end{lstlisting}
    Output: \texttt{Red: 72}
\end{frame}

\begin{frame}[fragile]{Example}
    Custom integer values can be defined for the first item because the starting point for setting the subsequent values.
    \begin{lstlisting}[language=C,basicstyle=\footnotesize,keywordstyle=\color{blue},commentstyle=\color{green},showstringspaces=false,stringstyle=\color{red}]
#include<stdio.h>

enum Color {
        Red = 72,
        Blue,
        Orange,
        Yellow,
        Green
};

int main() {
        enum Color mycolor;
        mycolor = Green;
        printf("Green: %d\n", mycolor);
        return 0;
}
    \end{lstlisting}
    Output: \texttt{Green: 76}
\end{frame}

\begin{frame}[fragile]{Practical Example}
    Let's revisit the rectangle drawing example.
    \begin{lstlisting}[language=C,basicstyle=\footnotesize,keywordstyle=\color{blue},commentstyle=\color{green},showstringspaces=false,stringstyle=\color{red}]
void draw_line(int width_size, int is_hollow) {
    int i;
    for(i = 0; i < width_size; i++) {
        if (i == 0 || i == width_size - 1 || is_hollow == 0) {
            printf("*");
        } else {
            printf(" ");
        }
    }
    printf("\n");
}
    \end{lstlisting}

\end{frame}

\begin{frame}[fragile]{Practical Example}
    Let's revisit the rectangle drawing example, now with an \texttt{enum}.
    \begin{lstlisting}[language=C,basicstyle=\footnotesize,keywordstyle=\color{blue},commentstyle=\color{green},showstringspaces=false,stringstyle=\color{red}]
enum LineType {
    Hollow,
    Solid
};

void draw_line(int width_size, enum LineType type) {
    int i;
    for(i = 0; i < width_size; i++) {
        if (i == 0 || i == width_size - 1 || type == Solid) {
            printf("*");
        } else {
            printf(" ");
        }
    }
    printf("\n");
}
    \end{lstlisting}
\end{frame}

\subsection{typedef}
\begin{frame}[fragile]{Enum \& typedef}
    Let's revisit the rectangle drawing example, now with an \texttt{enum} and \texttt{typedef}.
    \begin{lstlisting}[language=C,basicstyle=\footnotesize,keywordstyle=\color{blue},commentstyle=\color{green},showstringspaces=false,stringstyle=\color{red}]
typedef enum {
    Hollow,
    Solid
} LineType;

void draw_line(int width_size, LineType type) {
    int i;
    for(i = 0; i < width_size; i++) {
        if (i == 0 || i == width_size - 1 || type == Solid) {
            printf("*");
        } else {
            printf(" ");
        }
    }
    printf("\n");
}
    \end{lstlisting}
\end{frame}

\section{Good Programming Practice}
\begin{frame}[fragile]{Good Programming Practice}
    \begin{itemize}
        \item Don't forget the semicolon at the end of the \texttt{struct} \& \texttt{enum} definitions.
        \item Remember to use the format \texttt{variableName.fieldName} to access fields in a \texttt{struct}.
        \item A \texttt{struct} cannot have itself as a field.
        \begin{verbatim}
struct Pet {
    char *name;
    unsigned int age;
    struct Pet pets_pet; // compiler error
};
        \end{verbatim}
        \item Comparison operators, like \texttt{==} and \texttt{!=}, cannot be done on structs, instead you have to check each field separately.
    \end{itemize}
\end{frame}

\begin{frame}[fragile]{Good Programming Practice}
    \begin{itemize}
        \item Items in different \texttt{enum} statements must be unique.
        \begin{verbatim}
enum ColorA {
    Red,
    Blue
};

enum ColorB {
    Red, // compiler error
    Orange
};
        \end{verbatim}
    \end{itemize}
\end{frame}

\end{document}