% Copyright 2002-2023 The University of Maryland Baltimore County (UMBC)
% 1000 Hilltop Circle, Baltimore, Maryland, 21250, USA
% https://www.csee.umbc.edu/

\documentclass[graphics]{beamer}
\usepackage{graphicx}
\usepackage{listings} % Syntax highlighing
\usepackage{fancyvrb} % Inline verbatim
\usepackage{hyperref} % Hyperlinks
\hypersetup{pdfpagemode=FullScreen}

\usepackage[normalem]{ulem}               % to striketrhourhg text
\newcommand\redout{\bgroup\markoverwith
{\textcolor{red}{\rule[0.5ex]{2pt}{0.8pt}}}\ULon}

% header in tables
\newcommand*{\thead}[1]{\multicolumn{1}{c}{\bfseries #1}}

% used for arrows from one point in the slide to another
\usepackage{tikz}
\usetikzlibrary{arrows,shapes,tikzmark}

\usetheme{Boadilla}
\title{Lecture 12: Assignment Operators}
\author{UMBC CMSC 104}
\date{Spring 2024}

\begin{document}

\begin{frame}{}
\centering
    Assignment Operators
\end{frame}

\frame{\tableofcontents}

\section{Increment and Decrement Operators}
\begin{frame}{Increment and Decrement Operators}
    \begin{itemize}
        \item The \textbf{increment operator}: ++
        \item The \textbf{decrement operator}: - -
        \item Precedence: lower than () but higher than * / \%
        \item Associativity: right to left
        \item Increment and decrement operators can only be applied to variables, not to constants or expressions.
    \end{itemize}
\end{frame}

\subsection{Increment Operator}
\begin{frame}{Increment Operator}
    \begin{itemize}
        \item If we want to add one to a variable, we can use: \textbf{count = count + 1;}
        \item Programs often contain statements that increment variables, so to save on typing, C provides these shortcuts:
        \begin{itemize}
            \item \texttt{count++;}
            \item \texttt{++count;}
        \end{itemize}
        \item Both do the same thing, they change the value of \texttt{count} by adding one to it.
    \end{itemize}
\end{frame}

\begin{frame}[fragile]{Post-increment Operator}
    \begin{itemize}
        \item The position of the \texttt{++} determines when the value is incremented. If the \texttt{++} is after the variable, then the incrementing is done last (a \textbf{postincrement}).
        \begin{verbatim}
int amount, count;
count = 3;
amount = 2 * count++;
        \end{verbatim}
        \item The value of \texttt{amount} gets the value of $2 * 3$, and then 1 gets added to count.
        \item After executing the last line, \texttt{amount} is 6 and \texttt{count} is 4.
    \end{itemize}
\end{frame}

\begin{frame}[fragile]{Pre-increment Operator}
    \begin{itemize}
        \item If the \texttt{++} is before the variable, then the incrementing is done first (a \textbf{preincrement}).
        \begin{verbatim}
int amount, count;
count = 3;
amount = 2 * ++count;
        \end{verbatim}
        \item The value of \texttt{count} is incremented first, then \texttt{amount} gets the value of $2 * 4$.
        \item After executing the last line, \texttt{amount} is 8 and \texttt{count} is 4.
    \end{itemize}
\end{frame}

\begin{frame}[fragile]{Code Example Using ++}
    \begin{lstlisting}[language=C,basicstyle=\footnotesize,keywordstyle=\color{blue},commentstyle=\color{green},showstringspaces=false,stringstyle=\color{red}]
int i = 1;
// Count from 1 to 10
while(i < 11) {
    printf("%d\n", i);
    i++; // same as ++i
}
\end{lstlisting}
or...
    \begin{lstlisting}[language=C,basicstyle=\footnotesize,keywordstyle=\color{blue},commentstyle=\color{green},showstringspaces=false,stringstyle=\color{red}]
int i;
// Count from 1 to 10
for(i = 0; i < 11; i++) {
    printf("%d\n", i);
}
\end{lstlisting}
\end{frame}

\subsection{Decrement Operator}
\begin{frame}{Decrement Operator}
    \begin{itemize}
        \item If we want to subtract one from a variable, we can say: \texttt{count = count - 1};
        \item Programs often contain statements which decrement variables, so to save on typing, C provides these shortcuts:
        \begin{itemize}
            \item \texttt{count--;}
            \item \texttt{--count;}
        \end{itemize}
        \item Both do the same thing, they change the value of \texttt{count} by subtracting one from it, much like \text{++}.
    \end{itemize}
\end{frame}

\begin{frame}[fragile]{Post-decrement Operator}
    \begin{itemize}
        \item The position of the \texttt{--} determines when the value is decremented. If the \texttt{--} is after the variable, then the decrementing is done last (a \textbf{postdecrement}).
\begin{verbatim}
int amount, count;
count = 3;
amount = 2 * count--;
\end{verbatim}
        \item The value for \texttt{amount} gets the value of $2 * 3$, and then 1 gets subtracted from \texttt{count}.
        \item After executing the last line, \texttt{amount} is 6 and \texttt{count} is 2.
    \end{itemize}
\end{frame}

\begin{frame}[fragile]{Pre-decrement Operator}
    \begin{itemize}
        \item If the \texttt{--} is before the variable, then the decrementing is done first (a \textbf{predecrement}).
        \begin{verbatim}
int amount, count;
count = 3;
amount = 2 * --count;
        \end{verbatim}
        \item The value of \texttt{count} gets decremented, then \texttt{amount} gets the value of $2 * 2$.
        \item After executing the last line, \texttt{amount} is 4 and \texttt{count} is 2.
    \end{itemize}
\end{frame}

\begin{frame}[fragile]{Code Example Using - -}
    \begin{lstlisting}[language=C,basicstyle=\footnotesize,keywordstyle=\color{blue},commentstyle=\color{green},showstringspaces=false,stringstyle=\color{red}]
int i = 10;
// Count from 10 to 1
while(i > 1) {
    printf("%d\n", i);
    i--; // same as --i
}
\end{lstlisting}
or...
    \begin{lstlisting}[language=C,basicstyle=\footnotesize,keywordstyle=\color{blue},commentstyle=\color{green},showstringspaces=false,stringstyle=\color{red}]
int i;
// Count from 10 to 1
for(i = 10; i > 0; i--) {
    printf("%d\n", i);
}
\end{lstlisting}
\end{frame}

\subsection{Practice}
\begin{frame}{Practice}
    Given: \texttt{int a = 1, b = 2, c = 3}, \\
    Evaluate: \texttt{++a * b - c-- =} 
    \pause
    \texttt{1} \\ What are the new values of \texttt{a}, \texttt{b}, and \texttt{c}? \\
    \pause
    \texttt{a == 2}, \texttt{b == 2}, \texttt{c == 2} \\ ~~ \\
    \pause
    Evaluate: \texttt{++b / c * a =} 
    \pause
    \texttt{2} \\ What are the new values of \texttt{a}, \texttt{b}, and \texttt{c}? \\
    \pause
    \texttt{a == 2}, \texttt{b == 3}, \texttt{c == 2}
\end{frame}

\section{Assignment Operators}
\begin{frame}{Assignment Operators}
    \begin{tabular}{l l}
        Statement & Equivalent Statement \\ \hline
        a = a + 2; & a += 2; \\
        a = a - 3; & a -= 3; \\
        a = a * 2; & a *= 2; \\
        a = a / 4; & a /= 4; \\
        a = a \% 2; & a \%= 2; \\
        b = b + (c + 2); & b += c + 2; \\
        b = b * (e - 5); & b *= e-5;
    \end{tabular}
\end{frame}

\begin{frame}[fragile]{Code Example Using /= and ++}
    \begin{lstlisting}[language=C,basicstyle=\footnotesize,keywordstyle=\color{blue},commentstyle=\color{green},showstringspaces=false,stringstyle=\color{red}]
#include <stdio.h>

int main() {
    int num, temp, digits = 0;
    temp = num = 4327;
    while(temp > 0) {
        printf("%d\n", temp);
        temp /= 10;
        digits++;
    }
    printf("There are %d digits in %d.\n", digits, num);
    return 0;
}
    \end{lstlisting}
\end{frame}

\subsection{Practice}
\begin{frame}{Practice}
    Given: \texttt{int i = 1, j = 2, k = 3, m = 4;} \\
    \begin{tabular}{l | l}
        Expression & Value \\ \hline
        i += j + k      & \visible<2->{\texttt{i = 6, j = 2,  k = 3, m = 4}} \\
        j *= k = m + 5  & \visible<3->{\texttt{i = 6, j = 18, k = 9, m = 4}} \\
        k -= m /= j * 2 & \visible<4->{\texttt{i = 6, j = 18, k = 9, m = 0}} \\
    \end{tabular}
\end{frame}

\section{Debugging Tips}
\begin{frame}{Debugging Tips}
    \begin{itemize}
        \item Trace your code by hand (a \textbf{hand trace}), keeping track of the value of each variable.
        \item Insert temporary \texttt{printf()} statements so you can see what your program is doing.
        \begin{itemize}
            \item Confirm that the correct value(s) have been read in.
            \item Check the results of arithmetic computations immediately after they are performed.
        \end{itemize}
        \item Remember to remove your debugging statements before submitting assignments, and that your program still works after their removal.
    \end{itemize}
\end{frame}

\section{Exercise: Finding Prime Numbers}
\begin{frame}{In-class Exercise}
    Use nested loops to write a prime number calculator.
    \begin{itemize}
        \item Determine whether each member of a range of numbers is prime by attempting to divide it evenly by each of the smaller numbers.
        \item Strategy:
        \begin{itemize}
            \item Prompt user for upper limit, testing all numbers from 2 to this limit.
            \item Outer loop: iterate over all numbers from 2 to this limit, testing each in an inner loop to see if it's prime.
            \item Inner loop: Iterate over all numbers from 2 to the one less than the number being tested.
            \item At the end of an inner loop, if you were never able to evenly divide the number, then it is prime \& print it to the user.
        \end{itemize}
    \end{itemize}
\end{frame}

\end{document}