% Copyright 2002-2023 The University of Maryland Baltimore County (UMBC)
% 1000 Hilltop Circle, Baltimore, Maryland, 21250, USA
% https://www.csee.umbc.edu/

\documentclass[graphics]{beamer}
\usepackage{graphicx}
\usepackage{listings} % Syntax highlighing
\usepackage{fancyvrb} % Inline verbatim
\usepackage{hyperref} % Hyperlinks
\hypersetup{pdfpagemode=FullScreen}

\usepackage[normalem]{ulem}               % to striketrhourhg text
\newcommand\redout{\bgroup\markoverwith
{\textcolor{red}{\rule[0.5ex]{2pt}{0.8pt}}}\ULon}

% header in tables
\newcommand*{\thead}[1]{\multicolumn{1}{c}{\bfseries #1}}

% used for arrows from one point in the slide to another
\usepackage{tikz}
\usetikzlibrary{arrows,shapes,tikzmark}

\usetheme{Boadilla}
\title{Lecture 6: Intro to C}
\author{UMBC CMSC 104}
\date{Spring 2024}

\begin{document}

\begin{frame}{}
\centering
    Introduction to C
\end{frame}

\frame{\tableofcontents}

\section{Brief History of Programming Languages \& C}
\begin{frame}{History of Programming Languages \& C}
    \only<1> {
        \begin{itemize}
            \item Machine code
            \begin{itemize}
                \item Enter raw sequences of binary patterns (opcodes) \\
                1011010111001011 \\
                1011010110101010
            \end{itemize}
            \item Assembly Language
            \begin{itemize}
                \item Provides human-readable syntax to machine code: \\
                MOV R0, 5 \\
                SUB R0, 2 \\
                INC R0
            \end{itemize}
        \end{itemize}
        \\ ~~ \\
        \footnotesize{\url{https://en.wikipedia.org/wiki/Assembly_code\#Assembly_language}}
    }
    \only<2> {
        Early high-level languages:
        \begin{itemize}
            \item COBOL
            \begin{itemize}
                \item SUBTRACT B FROM A GIVING C
                \item MULTIPLY C BY 2 GIVING D
            \end{itemize}
            \item FORTRAN
            \begin{itemize}
                \item S1 = 3.0
                \item S2 = 4.0
                \item H = SQRT((S1*S2)+(S2*S2))
            \end{itemize}
            \item LISP
            \begin{itemize}
                \item (lambda (a) \\
	            (mapcar (func '+) \\
	            (cons (car (car a)) (car (cadr a)))))
            \end{itemize}
        \end{itemize}
    }
\end{frame}

\subsection{Assembly}
\begin{frame}[fragile]{Assembly Example}
    \begin{columns}
        \column{0.4\textwidth}
            \begin{verbatim}
int square(int num) {
    return num * num;
}
            \end{verbatim}
        \column{0.5\textwidth}
            \footnotesize
            \begin{verbatim}
push    rbp
mov     rbp, rsp
mov     DWORD PTR [rbp-4], edi
mov     eax, DWORD PTR [rbp-4]
imul    eax, eax
pop     rbp
ret
    \end{verbatim}
    \end{columns}
    \vfill
    Intel x64\_64 (64-bit)
\end{frame}

\begin{frame}[fragile]{Assembly Example}
    \begin{columns}
        \column{0.4\textwidth}
            \begin{verbatim}
int square(int num) {
    return num * num;
}
            \end{verbatim}
        \column{0.5\textwidth}
            \footnotesize
            \begin{verbatim}
sub     sp, sp, #16
str     w0, [sp, 12]
ldr     w0, [sp, 12]
mul     w0, w0, w0
add     sp, sp, 16
ret
    \end{verbatim}
    \end{columns}
    \vfill
    ARM 64-bit
\end{frame}

\begin{frame}[fragile]{Assembly Example}
    \begin{columns}
        \column{0.4\textwidth}
            \begin{verbatim}
int square(int num) {
    return num * num;
}
            \end{verbatim}
        \column{0.5\textwidth}
            \footnotesize
            \begin{verbatim}
std 31,-8(1)
stdu 1,-64(1)
mr 31,1
mr 9,3
stw 9,32(31)
lwz 9,32(31)
mullw 9,9,9
extsw 9,9
mr 3,9
addi 1,31,64
ld 31,-8(1)
blr
    \end{verbatim}
    \end{columns}
    \vfill
    PowerPC 64-bit LE-mode
\end{frame}

\begin{frame}[fragile]{Assembly Example}
    \begin{columns}
        \column{0.4\textwidth}
            \begin{verbatim}
int square(int num) {
    return num * num;
}
            \end{verbatim}
        \column{0.5\textwidth}
            \footnotesize
            \begin{verbatim}
addi    sp,sp,-32
sd      s0,24(sp)
addi    s0,sp,32
mv      a5,a0
sw      a5,-20(s0)
lw      a5,-20(s0)
mulw    a5,a5,a5
sext.w  a5,a5
mv      a0,a5
ld      s0,24(sp)
addi    sp,sp,32
jr      ra
    \end{verbatim}
    \end{columns}
    \vfill
    RISC-V 64-bit
\end{frame}

\begin{frame}{Assembly}
    \begin{itemize}
        \item ... is specific to the type of processor
        \item ... is generally only used when your algorithm has to be as fast as possible
        \item ... is much more difficult \& tedious than C.
    \end{itemize}
\end{frame}

\begin{frame}{History of C}
    \begin{itemize}
        \item Derived from... (wait for it...) ``B'' in 1972
        \begin{itemize}
            \item ``B'' (1969) was derived from \href{https://en.wikipedia.org/wiki/BCPL}{BCPL}
        \end{itemize}
        \item Design goals for C:
        \begin{itemize}
            \item Efficient
            \item Close to the machine: control hardware, manipulate memory
            \item Structured: sophisticated control flow and data structures
            \item Unlike Assembly, C is mostly portable between systems
        \end{itemize}
        \item UNIX was re-developed in C in 1973
        \begin{itemize}
            \item First written in assembly for the PDP-11, which had 64 KB of memory.
            \item UNIX, originally Unics: ``Uniplexed Information and Computing Service''\footnote{\url{https://en.wikipedia.org/wiki/Unix}}
        \end{itemize}
    \end{itemize}
\end{frame}

\begin{frame}{Does Programming Language Choice Matter?}
    Yes!
    \begin{itemize}
        \item Not all programming languages do the same thing or have the same capabilities.
        \item Generally, use the best tool for the job.
    \end{itemize}
\end{frame}

\section{The Anatomy of a C Program}
\begin{frame}{Writing C Programs}
    \begin{itemize}
        \item A programmer uses a \textbf{text editor} to create \& modify code, which is completely different from a word processor.
        \item Code is known as \textbf{source code}.
        \item A file containing source code is sometimes known as a \textbf{source file}.
    \end{itemize}
\end{frame}

\begin{frame}[fragile]{Writing C Programs}
    A super simple example:
    \begin{lstlisting}[language=c,showspaces=false,keywordstyle=\color{magenta},stringstyle=\color{purple},basicstyle=\ttfamily\footnotesize,numberstyle=\tiny\color{codegray},commentstyle=\color{green},]
    #include <stdio.h>
    int main() {
        printf("Hello, world!\n");
        return 0;
    }
    \end{lstlisting}
    After creating the code, the programmer invokes the compiler which turns code into a program.
\end{frame}

\section{Compilation}
\begin{frame}{Stages of Compilation}
    Stage 1: Preprocessing
    \begin{itemize}
        \item Main purposes:
        \begin{itemize}
            \item Centralise reused chunks of code
            \item Allow ``extensions'' to the language
            \item Make code more portable
        \end{itemize}
        \item Preformed by a program called the \textbf{preprocessor}.
        \item Modifies source code in memory according to \textbf{preprocessor directives} embedded in the source code.
        \item The code on disk is not modified.
        \item Include, or header, files end with ``.h'' 
    \end{itemize}
\end{frame}

\begin{frame}{Stages of Compilation (cont'd)}
    Stage 2: \textbf{Compilation}
    \begin{itemize}
        \item Performed by a program called the \textbf{compiler}.
        \item Translated the preprocessor-modified source code into \textbf{object code}, or \textbf{machine code}. Turning the code into assembly is part of this process.
        \item Checks for \textbf{syntax errors} and \textbf{warnings}.
        \item Saves the code to disk, depending on how the compiler was invoked.
        \begin{itemize}
            \item If any errors are generated, object code is not created.
        \end{itemize}
    \end{itemize}
\end{frame}

\begin{frame}{Stages of Compilation (cont'd)}
    Stage 3: \textbf{Linking}
    \begin{itemize}
        \item Combines the program object code with other object code to produce the executable file.
        \item The other object code can come from the Run-Time Library, other libraries, or object files that you have created.
        \item Saves the executable code to a disk file.  On the Linux system, that file is called \textbf{a.out} by default, if \texttt{-o $<$output\_name$>$} is not provided.
        \begin{itemize}
            \item If there are any linker errors, no executable file will be created.
        \end{itemize}
    \end{itemize}
\end{frame}

\begin{frame}[fragile]{A Simple C Program}
    \begin{lstlisting}[language=c,showspaces=false,keywordstyle=\color{magenta},stringstyle=\color{purple},basicstyle=\ttfamily\footnotesize,numberstyle=\tiny\color{codegray},commentstyle=\color{green},]
    /* Filename: hello.c
     * Author: Brian Kernighan & Dennis Ritchie
     * Date written: 1978
     * Description: This program prints the
     *               greeting "Hello, World!"
     */
    #include  <stdio.h>
    int main() {
     printf("Hello, World!\n");
     return 0;
    }
    \end{lstlisting}
\end{frame}

\begin{frame}[fragile]{Anatomy of a C Program}
    \begin{verbatim}
    program header comment \\

    preprocessor directives (if any) \\
    
    int main() \{ \\
    ~~~~     statement(s) \\
    ~~~~      return 0; \\
    \}
    \end{verbatim}
\end{frame}

\begin{frame}{Program Header Content}
    \begin{itemize}
        \item A comment is descriptive text used to help a reader of the program understand its content.
        \item All comments must begin with the characters  /*  and end with the characters  */
        \item These are called comment delimiters
        \item The program header comment always comes first.
        \item While not required technically, they are required as part of the coding style.
    \end{itemize}
\end{frame}

\begin{frame}[fragile]{Preprocessor Directives}
    \begin{itemize}
        \item Lines that begin with a \# in column 1 are called \textbf{preprocessor directives} (commands).
        \item Example: the \texttt{\#include <stdio.h>} directive causes the preprocessor to include a copy of the standard input/output header file ``stdio.h'' at this point in the code.
        \item This header file was included because it contains information about the \textbf{printf()} function that is used in this program, which displays text to the user.
    \end{itemize}
\end{frame}

\begin{frame}{int main()}
    \begin{itemize}
        \item Every program must have a \textbf{function} called main. This is where program execution begins.
        \item main() is placed in the source code file as the first function for readability.
        \item The \textbf{reserved word} ``int'' indicates that main() returns an integer value.
        \item The parentheses following ``main'' indicate that it is a function.
    \end{itemize}
\end{frame}

\begin{frame}{Function Body}
    \begin{itemize}
        \item A left brace (curly bracket) \{ begins the \textbf{body} or \textbf{definition} of every function.  A corresponding right brace \} ends the function body.
        \item The style is to place these braces on separate lines in column 1 and to indent the entire function body 3 to 4 spaces.
    \end{itemize}
\end{frame}

\begin{frame}{printf("Hello, world!\\n");}
    \begin{itemize}
        \item This line is a C \textbf{statement}.
        \item It is a \textbf{call} to the function \textbf{printf()} with a single \textbf{argument (parameter)}, namely the string ``Hello, World!\textbackslash n''.
        \item Even though a string may contain many characters, the string itself should be thought of as a single quantity.
        \item Notice that this line ends with a semicolon. All statements in C end with a semicolon.
        \item Also notice the character ``\textbackslash n''. It is the \textbf{new line character}, and represents one character despite being made up of a two characters. It ensures that the output which comes afterward is on the line below the current line.
    \end{itemize}
\end{frame}

\begin{frame}{return 0;}
    \begin{itemize}
        \item The function main() must return an integer value, so there needs to be a statement to indicate this value.
        \item Returning zero lets the operating system know that the program exited normally.
    \end{itemize}
\end{frame}

\section{Using the gcc Compiler}
\begin{frame}{Using the C compiler at UMBC}
    On the GL system at UMBC, use the \texttt{gcc} command to compile your code.
    \begin{itemize}
        \item All code \textbf{MUST} compile correctly on the GL system to receive credit for assignments!
        \begin{itemize}
            \item You may optionally work on assignments locally, but compile, test, and submit on GL.
            \item macOS users: install Xcode; Windows users: install the Windows Subsystem for Linux.
        \end{itemize}
        \item Optionally use the additional command -o output\_name to change the default output file name.
        \begin{itemize}
            \item \texttt{gcc main.c -o hw1} compiles the code in main.c into an executable called ``hw1''
            \item \texttt{gcc main.c} compiles the code in main.c into an executable called ``a.out''
        \end{itemize}
    \end{itemize}
\end{frame}

\begin{frame}{Invoking the gcc Compiler}
    At the prompt, type: \texttt{gcc -Wall program.c -o program}
    \begin{itemize}
        \item \texttt{program.c} is the C program source
        \item \texttt{-Wall} turns on all compiler warnings
        \item There are many other options for \texttt{gcc}, run \texttt{gcc --help} to see for yourself.
    \end{itemize}
\end{frame}

\begin{frame}{The Result: Your Executable}
    \begin{itemize}
        \item If there are no errors, the command produces an \textbf{executable file}, which is one that can be executed (or run).
        \item On Linux or Mac, there isn't a file extension; on Windows, the program would end with \texttt{.exe}
        \item To run your program, type \texttt{./program}, or \texttt{./a.out} if you didn't use the \texttt{-o} option.
        \item Although we call this process ``compiling a program'', what actually happens is more complicated.
    \end{itemize}
\end{frame}

\section{CMSC 104 C Program Standards \& Indentation Styles}
\begin{frame}{Good Programming Practices}
    \begin{itemize}
        \item C programming standards and indentation styles are available on Blackboard
        \item You are expected to conform to these standards for all programming projects in this class. Part of the grade is based on the coding style.
        \item Good practices ensures that you and others can read your code, including when you revisit your code, possibly years later.
    \end{itemize}
\end{frame}

\end{document}