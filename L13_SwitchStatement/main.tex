\documentclass[graphics]{beamer}
\usepackage{graphicx}
\usepackage{listings} % Syntax highlighing
\usepackage{fancyvrb} % Inline verbatim
\usepackage{hyperref} % Hyperlinks
\hypersetup{pdfpagemode=FullScreen}

\usepackage[normalem]{ulem}               % to striketrhourhg text
\newcommand\redout{\bgroup\markoverwith
{\textcolor{red}{\rule[0.5ex]{2pt}{0.8pt}}}\ULon}

% header in tables
\newcommand*{\thead}[1]{\multicolumn{1}{c}{\bfseries #1}}

% used for arrows from one point in the slide to another
\usepackage{tikz}
\usetikzlibrary{arrows,shapes,tikzmark}

\usetheme{Boadilla}
\title{Lecture 13: Switch}
\author{UMBC CMSC 104}
\date{Th 07 April 2022}

\begin{document}

\begin{frame}{}
\centering
    Switch Statements
\end{frame}

\frame{\tableofcontents}

\section{Multiple Selection}
\begin{frame}[fragile]{Multiple Selection}
    So far, we have only seen \textbf{binary selection}.
    \begin{verbatim}
if(age >= 18) {
    printf("Vote!\n"); // Either this...
} else {
    printf("Maybe next election.\n"); // .. or this
}
    \end{verbatim}
\end{frame}

\begin{frame}{Multiple Selection}
    \begin{itemize}
        \item Sometimes it is necessary to \textbf{branch} into more than two directions.
        \item We do this via \textbf{multiple selection}.
        \item The multiple selection mechanism in C is the \textbf{switch} statement.
    \end{itemize}
\end{frame}

\begin{frame}[fragile]{Multiple Selection - if}
    \begin{verbatim}
if (day == 0)
    printf("Sunday\n");
if (day == 1)
    printf("Monday\n");
if (day == 2)
    printf("Tuesday\n");
if (day == 3)
    printf("Wednesday\n");
if (day == 4)
    printf("Thursday\n");
if (day == 5)
    printf("Friday\n");
if (day == 6)
    printf("Saturday\n");
if (day > 6 || day < 0)
    printf("Invalid day.\n");
    \end{verbatim}
\end{frame}

\begin{frame}[fragile]{Multiple Selection - if-else}
    \begin{verbatim}
if (day == 0)
    printf("Sunday\n");
else if (day == 1)
    printf("Monday\n");
else if (day == 2)
    printf("Tuesday\n");
else if (day == 3)
    printf("Wednesday\n");
else if (day == 4)
    printf("Thursday\n");
else if (day == 5)
    printf("Friday\n");
else if (day == 6)
    printf("Saturday\n");
else (day > 6 || day < 0)
    printf("Invalid day.\n");
    \end{verbatim}
\end{frame}

\section{The switch Statement}
\begin{frame}[fragile]{Multiple Selection - The switch Statement}
    \begin{verbatim}
switch(integer expression) {
    case constant1:
        statement(s);
        break; // No break? We "cascade" to the next case
    case constant2:
        statement(s);
        break;
    ...
    default:
        statement(s);
}
    \end{verbatim}
\end{frame}

\begin{frame}{switch Statement Details}
    \begin{itemize}
        \item The last statement of each case in the switch statement should almost always be a \texttt{break}.
        \item The \texttt{break} causes the program control to jump to the closing brace of the \texttt{switch} statement.
        \item Without the \texttt{break}, the code flows to the next \texttt{case}. This is \textit{almost} never what you want, but \textit{might} be appropriate for some algorithms.
        \item The \texttt{switch} statement will compile without a \texttt{default} case, but always consider using one. It might also help with debugging, to catch a situation that you might not expect, for example.
    \end{itemize}
\end{frame}

\begin{frame}[fragile]{Good Programming Practices}
    \begin{itemize}
        \item Include a \texttt{default} case to catch invalid data.
        \item Inform the user of the type of error that has occurred (ex: ``Error - invalid day''.). Errors should always be informative.
        \item If appropriate and available, display the invalid value. Ex: invalid day number, appropriate to display, incorrect password, not appropriate.
        \item If appropriate, terminate program execution \textit{after} displaying an informative error message.\
        \begin{verbatim}
            exit(1);
        \end{verbatim}
    \end{itemize}
\end{frame}

\begin{frame}[fragile]{Switch Example}
    \begin{lstlisting}[language=C,basicstyle=\footnotesize,keywordstyle=\color{blue},commentstyle=\color{green},showstringspaces=false,stringstyle=\color{red}]
switch(day) {
    case 0: printf("Sunday\n"); break;
    case 1: printf("Monday\n"); break;
    case 2: printf("Tuesday\n"); break;
    case 3: printf("Wednesday\n"); break;
    case 4: printf("Thursday\n"); break;
    case 5: printf("Friday\n"); break;
    case 6: printf("Saturday\n"); break;
    default: printf("Error - invalid day.\n");
}
\end{lstlisting}
\end{frame}

\begin{frame}[fragile]{Switch Example}
    \begin{lstlisting}[language=C,basicstyle=\footnotesize,keywordstyle=\color{blue},commentstyle=\color{green},showstringspaces=false,stringstyle=\color{red}]
switch(day) {
    case 1: printf("Monday\n"); break;
    case 2: printf("Tuesday\n"); break;
    case 3: printf("Wednesday\n"); break;
    case 4: printf("Thursday\n"); break;
    case 5: printf("Friday\n"); break;
    case 0: /* cascade to the next case statement */
    case 6: printf("Weekend\n"); break;
    default: printf("Error - invalid day.\n");
}
\end{lstlisting}
\end{frame}

\begin{frame}{Why Use a Switch Statement?}
    \begin{itemize}
        \item A \texttt{switch} statement can be more efficient than an if-else.
        \item A \texttt{switch} statement may also be more readable.
        \item It is easier to add new cases to switch statement than to nested if-else statements.
    \end{itemize}
\end{frame}

\section{The char Data Type}
\begin{frame}[fragile]{The char Data Type}
    \begin{itemize}
        \item The \textbf{char} data type holds a single character, or \textbf{byte}.
        \begin{verbatim}
    char ch;
        \end{verbatim}
        \item Example assignments:
        \begin{verbatim}
    char grade, symbol;
    grade = `B';
    symbol = `$';
        \end{verbatim}
        \item The \texttt{char} is held as a one-byte integer in memory. The ASCII code\footnote{\url{https://www.asciitable.com/}} is what is actually stored, so we can use the \texttt{char} type as characters or integers, depending on the requirements of the algorithm we are implementing.
        \item The maximum value for a \textit{signed} char is 127 (7 bits only zero to 127).
        \item The maximum value for a \texttt{unsigned char} is 255 (8 bits/1 byte holds only zero to 255).
    \end{itemize}
\end{frame}

\begin{frame}[fragile]{The char Data Type}
    \begin{itemize}
        \item Use
        \begin{verbatim}
    scanf("%c", &ch);
        \end{verbatim}
        to read a single character into the variable \texttt{ch}, for example.
        \item Use
        \begin{verbatim}
    printf("%c", ch);
        \end{verbatim}
        to display the value of a character variable.
    \end{itemize}
\end{frame}

\begin{frame}[fragile]{char Example}
    \begin{verbatim}
#include<stdio.h>
int main() {
    char ch;
    printf("Enter a character: ");
    scanf("%c", &ch);
    printf("The value of %c is %d.\n", ch, ch);
    return 0;
}
    \end{verbatim}
    If the user would enter ``A'', the output would be: \\
    \texttt{The value of A is 65.}
\end{frame}

\begin{frame}{The getchar() function}
    \begin{itemize}
        \item The \texttt{getchar()} function is found in the \texttt{stdio.h} library.
        \item The \texttt{getchar()} function reads one character from \textbf{stdin} (the \textbf{standard input} buffer) and returns that character's ASCII value.
        \item The value can be stored in either a character or integer variable.
    \end{itemize}
\end{frame}

\begin{frame}[fragile]{getchar() example}
    \begin{verbatim}
#include<stdio.h>
int main() {
    char ch;
    printf("Enter a character: ");
    ch = getchar(); /*  same as scanf("%c", &ch);  */
    printf("The value of %c is %d.\n", ch, ch);
    return 0;
}
    \end{verbatim}
    If the user would enter ``A'', the output would be: \\
    \texttt{The value of A is 65.}
\end{frame}

\begin{frame}{Problems with Reading Characters}
    \begin{itemize}
        \item When getting characters, whether using \texttt{scanf()} or \texttt{getchar()}, realize that you are reading only one character.
        \item What will the user actually type? The character followed by pressing ENTER.
        \item So, the user is actually entering two characters, their response \& the newline character.
        \item Unless you handle this, the newline character will remain in the stdin stream causing problems the next time you want to read a character.  Another call to \texttt{scanf()} or \texttt{getchar()} will remove it.
    \end{itemize}
\end{frame}

\begin{frame}[fragile]{Improved Character Example}
    \begin{verbatim}
#include <stdio.h>
int main ( ) {
     char ch, newline;

     printf("Enter a character: ");
     scanf("%c", &ch);
     scanf("%c", &newline); /* grab the newline character */
     printf("The value of %c is %d.\n", ch, ch);
     printf("Enter another character: ");
     scanf("%c", &ch);
     scanf("%c", &newline); 
     printf("The value of %c is %d.\n", ch, ch);
     return 0;
}
    \end{verbatim}
\end{frame}

\begin{frame}[fragile]{Additional Concerns with Garbage in stdin}
    \begin{itemize}
        \item When we were reading integers using \texttt{scanf()}, we didn't seem to have problems with the newline character, even though the user was typing ENTER after the integer.
        \item That is because \texttt{scanf()} was looking for the next integer and ignored the newline (\textbf{whitespace}).
        \item If we use \texttt{scanf("\%d", \&num);} to get an integer, the newline is still stuck in the input stream.
        \item If the next item we want to get is a character, whether we use \texttt{scanf()} or \texttt{getchar()}, we will get the newline.
        \item We have to take this into account and remove it.
    \end{itemize}
\end{frame}

\end{document}