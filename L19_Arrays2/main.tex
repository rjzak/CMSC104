\documentclass[graphics]{beamer}
\usepackage{graphicx}
\usepackage{listings} % Syntax highlighing
\usepackage{fancyvrb} % Inline verbatim
\usepackage{hyperref} % Hyperlinks
\usepackage{attachfile} % attach files
\hypersetup{pdfpagemode=FullScreen}

\usepackage{multirow,tabularx}
\newcolumntype{Y}{>{\centering\arraybackslash}X}
\renewcommand{\arraystretch}{2}

\usepackage[normalem]{ulem}               % to striketrhourhg text
\newcommand\redout{\bgroup\markoverwith
{\textcolor{red}{\rule[0.5ex]{2pt}{0.8pt}}}\ULon}

% header in tables
\newcommand*{\thead}[1]{\multicolumn{1}{c}{\bfseries #1}}

% used for arrows from one point in the slide to another
\usepackage{tikz}
\usetikzlibrary{arrows, shapes, tikzmark, positioning}

\newcommand{\backupbegin}{
   \newcounter{framenumberappendix}
   \setcounter{framenumberappendix}{\value{framenumber}}
}
\newcommand{\backupend}{
   \addtocounter{framenumberappendix}{-\value{framenumber}}
   \addtocounter{framenumber}{\value{framenumberappendix}} 
}

\usetheme{Boadilla}
\title{Lecture 19: Arrays, Part II}
\author{UMBC CMSC 104}
\date{Th 04 May 2023}

\begin{document}

\begin{frame}{}
\centering
    Arrays, Part II
\end{frame}

\frame{\tableofcontents}

\section{Array Declarations Revisited}
\begin{frame}{Array Declarations Revisited}
    \texttt{int numbers[5];}
    \begin{itemize}
        \item This declaration sets aside a chunk of memory which is big enough to hold 5 integers (total of 20 bytes or 160 bits)
        \begin{itemize}
            \item The elements of the array are initialized but still contain garbage.
        \end{itemize}
        \item Besides the space needed for the array, this is also a variable allocated which has the name of the array. This variable (a \textbf{pointer}) holds the address of the beginning address of the first element of the array.
        \begin{itemize}
            \item First element is at \textbf{index} zero, which is byte address \texttt{0xFE00}
            \item The second element is at \textbf{index} one, which is byte address \texttt{0xFE04}
        \end{itemize}
    \end{itemize}
    
    \begin{tabularx}{\textwidth}{|*{7}{Y|}}
        \multicolumn{1}{l}{numbers} & \multicolumn{1}{l}{~~} & \multicolumn{5}{l}{FE00}\\ \cline{1-1}\cline{3-7}
        \multicolumn{1}{|c|}{FE00} & \multicolumn{1}{c|}{~~} & -- & -- & -- & -- & -- \\ \cline{1-1}\cline{3-7}
        \multicolumn{2}{r}~~ & \multicolumn{1}{r}0 & \multicolumn{1}{r}1 & \multicolumn{1}{r}2 & \multicolumn{1}{r}3 & \multicolumn{1}{r}4
    \end{tabularx}
\end{frame}

\begin{frame}[fragile]{Array Name Holds an Address}
    \begin{lstlisting}[language=C,basicstyle=\footnotesize,keywordstyle=\color{blue},commentstyle=\color{green},showstringspaces=false,stringstyle=\color{red}]
#include<stdio.h>
int main() {
    int numbers[5] = {97, 68, 55, 73, 84};
    printf("numbers[0] = %d, %c\n", numbers[0], numbers[0]);
    printf("numbers = %X\n", numbers);
    printf("&numbers[0] = %X\n", &numbers[0]);
    return 0;
}
    \end{lstlisting}
    Output:
    \begin{verbatim}
numbers[0] = 97, a
numbers = FE00
&numbers[0] = FE00
    \end{verbatim}
\end{frame}

\subsection{Indexing}
\begin{frame}{How Indexing Works}
    \texttt{numbers[2] = 7;}
    \begin{itemize}
        \item The element assigned the value 7 is stored in a memory location which is calculated using the following formula: \\
        \textbf{Location = (beginning address) + (index * sizeof(array type))} \\
        Assuming a 4-byte (32-bit) integer: \\
        \textbf{Location = FE00 + (2 * 4)}
    \end{itemize}
    
    \begin{tabularx}{\textwidth}{|*{7}{Y|}}
        \multicolumn{1}{l}{numbers} & \multicolumn{1}{l}{~~} & \multicolumn{1}{l}{FE00} & \multicolumn{1}{l}{FE04} & \multicolumn{1}{l}{FE08} & \multicolumn{1}{l}{FE0C} & \multicolumn{1}{l}{FE10} \\ \cline{1-1}\cline{3-7}
        \multicolumn{1}{|l|}{FE00} & \multicolumn{1}{c|}{~~} & 97 & 68 & 7 & 73 & 84 \\ \cline{1-1}\cline{3-7}
        \multicolumn{2}{r}~~ & \multicolumn{1}{r}{0} & \multicolumn{1}{r}{1} & \multicolumn{1}{r}{2} & \multicolumn{1}{r}{3} & \multicolumn{1}{r}{4}
    \end{tabularx}
\end{frame}

\begin{frame}{Indexing Arrays}
    \begin{itemize}
        \item As long as we know:
        \begin{itemize}
            \item the beginning location of an array,
            \item the data type being held in the array, and
            \item the size of the array (so we don't overstep our bounds),
        \end{itemize}
        then we can access or modify any of its elements using indexing.
        \item The array name alone (without []) is just a variable which holds a pointer, which indicates (or \textit{points to}) the block of memory where the array is held.
    \end{itemize}
\end{frame}

\section{Pass by Value vs Reference}
\begin{frame}{Call (Pass) by Value}
    \begin{itemize}
        \item So far, we have passed only values to functions.
        \item The function has a local variable (a formal parameter) to hold its own copy of the value passed in.
        \item When we make changes to this copy, the original (corresponding actual parameter) remains unchanged.
        \item This is known as \textbf{calling (passing) by value}.
    \end{itemize}
\end{frame}

\begin{frame}{Passing Arrays to Functions}
    \begin{itemize}
        \item The function prototype: \texttt{void FillArray(int nums[], int numElements);}
        \item The function call: \texttt{FillArray(numbers, 5);}
        \item This example passes the name of the array (pointer), and doesn't have a return value. This is because the array will be directly modified from the function.
        \item This is known as \textbf{calling (passing) by reference}.
        \item When passing by reference, the function makes changes to the original, and no copy is made.
        \item This is especially helpful for large arrays, where we wouldn't want to waste memory with a copy of the array.
    \end{itemize}
\end{frame}

\begin{frame}[fragile]{Passing an Array to a Function}
    \begin{columns}
        \column[t]{5cm}
            \begin{lstlisting}[language=C,basicstyle=\scriptsize,keywordstyle=\color{blue},commentstyle=\color{green},showstringspaces=false,stringstyle=\color{red},tabsize=2]
#include<stdio.h>
#define SIZE 4
void FillArray(int array[],
  int size);
int main() {
  int i;
  int someArray[SIZE];
  FillArray(someArray, SIZE);
  for(i = 0; i < SIZE; i++) {
      printf("someArray[%d] = %d\n",
        i, someArray[i]);
  }
  return 0;
}
            \end{lstlisting}
        \column[t]{6cm}
            \begin{lstlisting}[language=C,basicstyle=\scriptsize,keywordstyle=\color{blue},commentstyle=\color{green},showstringspaces=false,stringstyle=\color{red}]
void FillArray(int array[], int size) {
    int i;
    for(i = 0; i < size; i++) {
        array[i] = i;
    }
}
            \end{lstlisting}
            ~~ \\ ~~ \\ ~~ \\ ~~ \\ Output:
            \footnotesize
            \begin{verbatim}
someArray[0] = 0
someArray[1] = 1
someArray[2] = 2
someArray[3] = 3
            \end{verbatim}
    \end{columns}
\end{frame}

\section{Example: Grades Program}
\begin{frame}{Grades Program Using Pass by Reference}
    \underline{Problem:} Find the average test score and number of A's, B's, C's, D's, F's for a particular class.
\end{frame}

\begin{frame}[fragile]{``Clean'' Grades (continued from prior lecture)}
    \begin{lstlisting}[language=C,basicstyle=\scriptsize,keywordstyle=\color{blue},commentstyle=\color{green},showstringspaces=false,stringstyle=\color{red}]
#include <stdio.h>
#define  SIZE       39
#define  GRADES     5
#define  A          4
#define  B          3
#define  C          2
#define  D          1
#define  F          0
#define  MAX        100
#define  MIN        0
void PrintInstructions();
void InitArray(int anArray[ ], int size);
void FillArray(int anArray[ ], int size);
float ProcessGrades(int score[ ], int size, int gradeCount[ ]);
float FindAverage(float sum, int num);
void  PrintResults(float average, int gradeCount[ ]);
    \end{lstlisting}
\end{frame}

\begin{frame}[fragile]{``Clean'' Grades (cont'd)}
    \begin{lstlisting}[language=C,basicstyle=\scriptsize,keywordstyle=\color{blue},commentstyle=\color{green},showstringspaces=false,stringstyle=\color{red}]
int main() {
    int score [SIZE];       /* student scores                   */
    int gradeCount[GRADES]; /* count of A's, B's, C's, D's, F's */
    float average;          /* average score                    */
    
    PrintInstructions();
    InitArray(gradeCount, GRADES);
    FillArray(score, SIZE);
    average = ProcessGrades(score, SIZE, gradeCount);
    PrintResults(average, gradeCount);

    return 0;
}
    \end{lstlisting}
\end{frame}

\begin{frame}[fragile]{``Clean'' Grades (cont'd)}
    \begin{lstlisting}[language=C,basicstyle=\scriptsize,keywordstyle=\color{blue},commentstyle=\color{green},showstringspaces=false,stringstyle=\color{red}]
/*********************************************************
** PrintInstructions - prints the user instructions
** Inputs:  None
** Outputs:  None
**********************************************************/
void PrintInstructions() {
    printf("This program calculates the average score\n");
    printf("for a class of 39 students. It also reports the\n");
    printf("number of As, B's, C's, D's, and F's. You will\n");
    printf("be asked to enter the individual scores.\n");
}
    \end{lstlisting}
\end{frame}

\begin{frame}[fragile]{``Clean'' Grades (cont'd)}
    \begin{lstlisting}[language=C,basicstyle=\scriptsize,keywordstyle=\color{blue},commentstyle=\color{green},showstringspaces=false,stringstyle=\color{red}]
/**********************************************************
** InitArray - initializes an integer array to all zeros
** Inputs:  anArray - array to be initialized
**              size - size of the array
** Outputs:  None
***********************************************************/
void InitArray(int anArray[ ], int size) {
    int i;
    for (i = 0; i < size; i++ ) {
        anArray[i] = 0;
    }
}    
    \end{lstlisting}
\end{frame}

\begin{frame}[fragile]{``Clean'' Grades (cont'd)}
    \begin{lstlisting}[language=C,basicstyle=\scriptsize,keywordstyle=\color{blue},commentstyle=\color{green},showstringspaces=false,stringstyle=\color{red}]
/******************************************************************
** FillArray - fills an integer array with valid values that
**                are entered by the user.  
**                Assures the values are between MIN and MAX.
** Inputs:  anArray - array to fill
** Outputs:  size - size of the array
** Side Effect - MIN and MAX must be #defined in this file
********************************************************************/
void FillArray(int anArray [ ], int size) {
     int i;   /* loop counter */
     for( i = 0; i < size; i++ ) {
         printf("Enter next value: ");
         scanf("%d", &anArray[ i ]);
         while( (anArray[ i ] < MIN) || (anArray[ i ] > MAX) ) {
              printf("Values must be between %d and %d\n", MIN, MAX);
              printf("Enter next value: ");
              scanf("%d", &anArray[ i ]);
         }
     }
}
    \end{lstlisting}
\end{frame}

\begin{frame}[fragile]{``Clean'' Grades (cont'd)}
    \begin{lstlisting}[language=C,basicstyle=\scriptsize,keywordstyle=\color{blue},commentstyle=\color{green},showstringspaces=false,stringstyle=\color{red}]
/*******************************************************************
** ProcessGrades- counts letter grade totals, computes average
** Inputs: score- array of student scores, size- size of the array
           gradeCount- grade counts all initialized to zero
** Outputs: gradeCount- number of A's, B's, C's, D's, and F's
** Side Effect:  A, B, C, D, and F must be #defined in this file
********************************************************************/
float ProcessGrades(int score[ ], int size, int gradeCount [ ]) {
    int i, total = 0;  /* loop counter, total of all scores */
    float average;     /* average score      */
    for( i = 0 ; i < size ; i++) {
        total += score[ i ];
        switch(score[ i ] / 10) {
            case 10: case 9: gradeCount [A]++; break;
            case 8: gradeCount[B]++; break;
            case 7: gradeCount[C]++; break;
            case 6: gradeCount[D]++; break;
            case 5: case 4: case 3: case 2:
            case 1: case 0: gradeCount[F]++; break;
            default: printf("Score %d invalid.\n", score[i]);
        }
    }
    average = findAverage(total, size);
    return average;
}
    \end{lstlisting}
\end{frame}

\begin{frame}[fragile]{``Clean'' Grades (cont'd)}
    \begin{lstlisting}[language=C,basicstyle=\scriptsize,keywordstyle=\color{blue},commentstyle=\color{green},showstringspaces=false,stringstyle=\color{red}]
/*****************************************************
** FindAverage - calculates an average
** Inputs:  sum - the sum of all values
**              num - the number of values
** Outputs:  the computed average
******************************************************/
float FindAverage(float sum, int num) {
    float average;   /* computed average */
     if ( num != 0 ) {
          average = sum / num;
     } else {
          average = 0 ;
    }

    return average;
}
    \end{lstlisting}
\end{frame}

\begin{frame}[fragile]{``Clean'' Grades (cont'd)}
    \begin{lstlisting}[language=C,basicstyle=\scriptsize,keywordstyle=\color{blue},commentstyle=\color{green},showstringspaces=false,stringstyle=\color{red}]
/******************************************************************
** PrintResults - prints the class average and grade distribution
** Inputs:  average - class average
**          gradeCount - number of A's, B's, C's, D's, F's
** Outputs:  None
** Side Effect: A, B, C, D, F must be #defined in this file
*******************************************************************/
void PrintResults(float average, int gradeCount[ ]) {
    printf("The class average is %.2f\n", average);
    printf("There were %2d As\n", gradeCount[A]);
    printf("           %2d Bs\n", gradeCount[B]);
    printf("           %2d Cs\n", gradeCount[C]);
    printf("           %2d Ds\n", gradeCount[D]);
    printf("           %2d Fs\n", gradeCount[F]);
}
    \end{lstlisting}
\end{frame}

\section*{Pointers}
\begin{frame}[fragile]{One More Thing: Non-Array Pointers}
    \begin{itemize}
        \item An array declaration is a pointer to the memory, for example: \texttt{int numbers[5];}
        \item We can also have pointers to non-arrays, such as: \\
        \texttt{int myNum = 2;}\\
        \texttt{int *myNumPtr = \&myNum;}
        \item Functions can receive pointers, and change the values they represent.
    \end{itemize}
\end{frame}

\begin{frame}[fragile]{Struct \& Pointer Example}
    \begin{lstlisting}[language=C,basicstyle=\scriptsize,keywordstyle=\color{blue},commentstyle=\color{green},showstringspaces=false,stringstyle=\color{red}]
struct Dog {
    char breed[10];
    char name[10];
    unsigned int age;
    float weight;
}

void DogInit(struct Dog *dog);

int main() {
    struct Dog mydog;
    DogInit(&mydog);
    printf("%s weighs %1.2f pounds.\n", mydog.name, mydog.weight);
}
    \end{lstlisting}
\end{frame}

\begin{frame}[fragile]{Struct \& Pointer Example}
    \begin{lstlisting}[language=C,basicstyle=\scriptsize,keywordstyle=\color{blue},commentstyle=\color{green},showstringspaces=false,stringstyle=\color{red}]
void DogInit(struct Dog *dog) {
    dog->breed = "Retriever";
    dog->name = "Fido";
    dog->age = 2;
    dog->weight = 30.1;
    // -> is shorthand for:
    // &(dog).weight = 31;
}
    \end{lstlisting}
\end{frame}

\begin{frame}[fragile]{Struct Arrays}
    Arrays of structs is also possible.
    \begin{lstlisting}[language=C,basicstyle=\scriptsize,keywordstyle=\color{blue},commentstyle=\color{green},showstringspaces=false,stringstyle=\color{red}]
struct Dog {
    char breed[10];
    char name[10];
    unsigned int age;
    float weight;
}

Struct Dog mydog[5]; // Five-dog array

mydog[2].name = "Fido"; // Index into an array of Structs
                        // like any other array
    \end{lstlisting}
\end{frame}

\section{Matrices}
\begin{frame}[fragile]{How about Matrices?}
    \begin{lstlisting}[language=C,basicstyle=\scriptsize,keywordstyle=\color{blue},commentstyle=\color{green},showstringspaces=false,stringstyle=\color{red}]
#include<stdio.h>
#define M 2
#define N 3
int main() {
  // matrix[outer][inner]
  int matrix[M][N] = {{1,2,3}, {4,5,6}};
  int i, j;
  for(i = 0; i < M; i++) {
     for(j = 0; j < N; j++) {
         printf("matrix[%d][%d] = %d\n", i, j, matrix[i][j]);
     }
  }
  return 0;
}
    \end{lstlisting}
    Output:
    \scriptsize
    \begin{verbatim}
matrix[0][0] = 1
matrix[0][1] = 2
matrix[0][2] = 3
matrix[1][0] = 4
matrix[1][1] = 5
matrix[1][2] = 6
    \end{verbatim}
\end{frame}

\begin{frame}[fragile]{Matrix Function Example}
    \begin{lstlisting}[language=C,basicstyle=\scriptsize,keywordstyle=\color{blue},commentstyle=\color{green},showstringspaces=false,stringstyle=\color{red}]
#include<stdio.h>
#define M 2
#define N 3
void MatrixInit(int matrix[][N], int rows);
void MatrixPrint(int matrix[][N], int rows);
int main() {
  // matrix[outer][inner]
  int matrix[M][N] = {{1,2,3}, {4,5,6}};
  MatrixInit(matrix, M);
  MatrixPrint(matrix, M);
  for(i = 0; i < M; i++) {
     for(j = 0; j < N; j++) {
         printf("matrix[%d][%d] = %d\n", i, j, matrix[i][j]);
     }
  }
  return 0;
}
    \end{lstlisting}
\end{frame}

\begin{frame}[fragile]{Matrix Function Example}
    \begin{lstlisting}[language=C,basicstyle=\scriptsize,keywordstyle=\color{blue},commentstyle=\color{green},showstringspaces=false,stringstyle=\color{red}]
void MatrixInit(int matrix[][N], int rows) {
    int i, j;
    for (i = 0; i < rows;i++) {
        for (j = 0; j < N; j++) {
            matrix[i][j] = i + j;
        }
    }
}

void MatrixPrint(int matrix[][N], int rows) {
    int i, j;
    for(i = 0; i < rows; i++) {
            for(j = 0; j < N; j++) {
                    printf("  %d", matrix[i][j]);
            }
            printf("\n");
    }
}
    \end{lstlisting}
    Output:
    \scriptsize
    \begin{verbatim}
  0  1  2
  1  2  3
    \end{verbatim}
\end{frame}

\begin{frame}{Matrices}
    \begin{itemize}
        \item Matrices, and higher dimension arrays (or, more generally, \textbf{multidimensional arrays}) are possible.
        \item Indexing into a matrix has two brackets, first to index into the array for that row, then for the element of that array. A matrix is basically an array of arrays.
        \item Passing a matrix to a function requires specifying the size of the matrix for the second dimensional size.
        \begin{itemize}
            \item For higher dimensional arrays, only the first dimension can be empty.
            \item 3D array: \texttt{int three\_d\_array[10][20][30];}
        \end{itemize}
    \end{itemize}
\end{frame}

\subsection{Example: Further Grades Program improvement}
\begin{frame}[fragile]{Grades Program with Matrices \& Structs: \#defines, arrays}
    \begin{lstlisting}[language=C,basicstyle=\scriptsize,keywordstyle=\color{blue},commentstyle=\color{green},showstringspaces=false,stringstyle=\color{red}]
#include<stdio.h>
#include<stdbool.h>

#define CLASS_SIZE 10
#define GRADES 5
#define PLUS_MINUS_SIZE 3
#define A 4
#define B 3
#define C 2
#define D 1
#define F 0
#define PLUS 0
#define REGULAR 1
#define MINUS 2
#define MAX 100

const char GRADE_LETTERS[GRADES] = {'F', 'D', 'C', 'B', 'A'};
const char PLUS_MINUS[PLUS_MINUS_SIZE] = {'+', ' ', '-'};
    \end{lstlisting}
\end{frame}

\begin{frame}[fragile]{Grades Program with Matrices \& Structs: Struct \& Prototypes}
    \begin{lstlisting}[language=C,basicstyle=\scriptsize,keywordstyle=\color{blue},commentstyle=\color{green},showstringspaces=false,stringstyle=\color{red}]
typedef struct {
    float mean; /* average of all scores */
    unsigned int median; /* when sorted, which grade is in the middle? */
    unsigned int max;    /* max score */
    unsigned int min;    /* minimum score */
} ClassStatistics;

void PrintInstructions();
void GetScores(unsigned int scores[], const unsigned int size);
ClassStatistics CalcStats(unsigned int scores[],
    const unsigned int size,
    unsigned int gradesCount[GRADES][PLUS_MINUS_SIZE]);

void swap(unsigned int *p, unsigned int *q);
void sort(unsigned int scores[], const unsigned int size);
void PrintStats(const ClassStatistics stats,
    const unsigned int gradesCount[GRADES][PLUS_MINUS_SIZE]);
    \end{lstlisting}
\end{frame}

\begin{frame}[fragile]{Grades Program with Matrices \& Structs: main()}
    \begin{lstlisting}[language=C,basicstyle=\scriptsize,keywordstyle=\color{blue},commentstyle=\color{green},showstringspaces=false,stringstyle=\color{red}]
int main() {
    /* Grades are never negative, so we'll use unsigned integers */
    unsigned int scores[CLASS_SIZE] = {0};
    unsigned int gradesCount[GRADES][PLUS_MINUS_SIZE] = {0};

    PrintInstructions();
    GetScores(scores, CLASS_SIZE);
    ClassStatistics stats = CalcStats(scores, CLASS_SIZE,
        gradesCount);
    PrintStats(stats, gradesCount);

    return 0;
}
    \end{lstlisting}
\end{frame}

\begin{frame}[fragile]{Grades Program with Matrices \& Structs}
    \textit{Almost} the same \texttt{PrintInstructions()} function.
    \begin{lstlisting}[language=C,basicstyle=\scriptsize,keywordstyle=\color{blue},commentstyle=\color{green},showstringspaces=false,stringstyle=\color{red}]
/*********************************************************
** PrintInstructions - prints the user instructions
** Inputs:  None
** Outputs:  None
** Notes: CLASS_SIZE must be #defined in this file
**********************************************************/
void PrintInstructions() {
    printf("This program calculates the average score\n");
    printf("for a class of %d students. It also reports the\n",
        CLASS_SIZE);
    printf("number of As, B's, C's, D's, and F's. You will\n");
    printf("be asked to enter the individual scores.\n");
}
    \end{lstlisting}
\end{frame}

\begin{frame}[fragile]{Grades Program with Matrices \& Structs: GetScores()}
    \begin{lstlisting}[language=C,basicstyle=\scriptsize,keywordstyle=\color{blue},commentstyle=\color{green},showstringspaces=false,stringstyle=\color{red}]
/****************************************************************
** GetScores - fills an unsigned integer array with valid values
**                that are entered by the user.  
**                Assures the values are between zero and MAX.
** Inputs:  scores - array to fill
** Outputs: size - size of the array
** Notes: MAX must be #defined in this file
******************************************************************/
void GetScores(unsigned int scores[], const unsigned int size) {
    int i;
    for(i = 0; i < size; i++) {
        printf("Enter next value: ");
        scanf("%u", &scores[i]); // %u for unsigned ints!
        while( scores[i] > MAX) {
            printf("Values must not be greater than %d.\n", MAX);
            printf("Enter next value: ");
            scanf("%u", &scores[i]);
        }
    }
}
    \end{lstlisting}
\end{frame}

\begin{frame}[fragile]{Grades Program with Matrices \& Structs: CalcStats()}
    \begin{lstlisting}[language=C,basicstyle=\scriptsize,keywordstyle=\color{blue},commentstyle=\color{green},showstringspaces=false,stringstyle=\color{red}]
/*****************************************************************
** CalcStats - Returns statistical information about the grades
** Inputs:  scores array, size of scores array, and grades matrix
** Outputs: statistics struct
** Notes: Letter grades, GRADES, PLUS_MINUS_SIZE, MINUS, REGULAR,
**        PLUS must be #defined in this file
*******************************************************************/
ClassStatistics CalcStats(unsigned int scores[],
    const unsigned int size,
    unsigned int gradesCount[GRADES][PLUS_MINUS_SIZE]) {
    
    unsigned int i, total = 0;
    ClassStatistics stats;
    for(i = 0; i < size; i++) {
        total += scores[i];
        if (i == 0) {// assume first element is largest, smallest
            stats.max = scores[i];
            stats.min = scores[i];
        } else {
            if (stats.max < scores[i])
                stats.max = scores[i];
            if (stats.min > scores[i])
                stats.min = scores[i];
        }
    \end{lstlisting}
\end{frame}

\begin{frame}[fragile]{Grades Program with Matrices \& Structs: CalcStats()}
    \begin{lstlisting}[language=C,basicstyle=\scriptsize,keywordstyle=\color{blue},commentstyle=\color{green},showstringspaces=false,stringstyle=\color{red}]
        switch(scores[i]/10) {
            case 10: gradesCount[A][PLUS]++; break;
            
            /* There has to be a better way to handle plus/minus,
               is there a common way to have the cut-off values? */
            case 9:
                if (scores[i] >= 97) {
                    gradesCount[A][PLUS]++;
                } else if (scores[i] > 92) {
                    gradesCount[A][REGULAR]++;
                } else {
                    gradesCount[A][MINUS]++;
                }
                break;
            case 8:
                if (scores[i] >= 87) {
                    gradesCount[B][PLUS]++;
                } else if (scores[i] > 82) {
                    gradesCount[B][REGULAR]++;
                } else {
                    gradesCount[B][MINUS]++;
                }
                break;
    \end{lstlisting}
\end{frame}

\begin{frame}[fragile]{Grades Program with Matrices \& Structs: CalcStats()}
    \begin{lstlisting}[language=C,basicstyle=\scriptsize,keywordstyle=\color{blue},commentstyle=\color{green},showstringspaces=false,stringstyle=\color{red}]
            case 7:
                if (scores[i] > 77) {
                    gradesCount[C][PLUS]++;
                } else if (scores[i] > 72) {
                    gradesCount[C][REGULAR]++;
                } else {
                    gradesCount[C][MINUS]++;
                }
                break;
            case 6:
                if (scores[i] > 67) {
                    gradesCount[D][PLUS]++;
                } else if (scores[i] > 62) {
                    gradesCount[D][REGULAR]++;
                } else {
                    gradesCount[D][MINUS]++;
                }
                break;
            /* F is a special case, no plus or minus */
            case 5: case 4: case 3: case 2:
            case 1: case 0: gradesCount[F][REGULAR]++; break;
            default: printf("Invalid score %d found at index %d.\n",
                scores[i], i);
        } // end switch
    } // end for
    \end{lstlisting}
\end{frame}

\begin{frame}[fragile]{Grades Program with Matrices \& Structs: CalcStats()}
    \begin{lstlisting}[language=C,basicstyle=\scriptsize,keywordstyle=\color{blue},commentstyle=\color{green},showstringspaces=false,stringstyle=\color{red}]
    stats.mean = (float) total / (float) size;
    
    sort(scores, size); // Must sort the array first...
    stats.median = scores[(size+1) / 2 -1]; // get middle value
    
    return stats;
} // end CalcStats()
    \end{lstlisting}
\end{frame}

\begin{frame}[fragile]{Grades Program with Matrices \& Structs: Sorting}
    \begin{lstlisting}[language=C,basicstyle=\scriptsize,keywordstyle=\color{blue},commentstyle=\color{green},showstringspaces=false,stringstyle=\color{red}]
/*********************************************************
** swap: swaps two locations in an unsigned integer array
** Inputs: p & q are pointers to specific positions in the array
** Outputs: None, modifies the contents of the array
*********************************************************/
void swap(unsigned int *p, unsigned int *q) {
   unsigned int temp = *p;
   *p = *q;
   *q = temp;
}
/*********************************************************
** sort: sorts an array from low to high
** Inputs: unsigned integer array, size
** Outputs: None, modifies the contents of the array
**********************************************************/
void sort(unsigned int scores[], const unsigned int size) { 
   unsigned int i, j;
   for(i = 0; i < size-1; i++) {
      for(j = 0; j < size-i-1; j++) {
         if (scores[j] > scores[j+1])
            swap(&scores[j], &scores[j+1]);
      }
   }
}
    \end{lstlisting}
\end{frame}

\begin{frame}[fragile]{Grades Program with Matrices \& Structs: PrintStats()}
    \begin{lstlisting}[language=C,basicstyle=\scriptsize,keywordstyle=\color{blue},commentstyle=\color{green},showstringspaces=false,stringstyle=\color{red}]
/******************************************************************
** PrintResults - prints the class average and grade distribution
** Inputs:  stats is the struct with several measured data points
**          gradesCount matrix which has the totalled grades with
**                      plus, minus, and regular values
** Outputs:  None
** Notes:: A, B, C, D, F must be #defined in this file
*******************************************************************/
void PrintStats(const ClassStatistics stats,
    const unsigned int gradesCount[GRADES][PLUS_MINUS_SIZE]) {
    
    unsigned int i, j;
    bool first = true;
    printf("The class average is %.2f, median is %d\n",
        stats.mean, stats.median);
    printf("Max: %d, Min: %d\n", stats.max, stats.min);

    for(i = GRADES-1; i >= 0; i--) {
        if (i == F) { /* F is a special case, no plus or minus */
            printf("           %2d %c\n", gradesCount[i][REGULAR],
                GRADE_LETTERS[i]);
            break;
        }
        
    \end{lstlisting}
\end{frame}

\begin{frame}[fragile]{Grades Program with Matrices \& Structs: PrintStats()}
    \begin{lstlisting}[language=C,basicstyle=\scriptsize,keywordstyle=\color{blue},commentstyle=\color{green},showstringspaces=false,stringstyle=\color{red}]
        for(j = 0; j < PLUS_MINUS_SIZE; j++) {
            if (first) { // Easy way to have cleaner output
                printf("There were ");
                first = false;
            } else {
                printf("           ");
            }
            printf("%2d %c%c\n", gradesCount[i][j], GRADE_LETTERS[i],
                PLUS_MINUS[j]);
        }
    }
} // end PrintStats()
\end{lstlisting}
\end{frame}

\begin{frame}[fragile]{Grades Program with Matrices \& Structs: Example Output}
    \scriptsize
    \begin{verbatim}
The class average is 80.28, median is 83
Max: 100, Min: 14
There were  6 A+
            5 A 
            5 A-
            1 B+
            5 B 
            5 B-
            0 C+
            1 C 
            2 C-
            0 D+
            4 D 
            2 D-
            3 F
    \end{verbatim}
    Full code: \textattachfile[color=1 0 0]{grades_matrices.c}{grades\_matrices.c} \\
    Example with dynamic memory: \textattachfile[color=1 0 0]{grades_matrices_dynamic.c}{grades\_matrices\_dynamic.c}
\end{frame}

\end{document}