\documentclass[graphics]{beamer}
\usepackage{graphicx}
\usepackage{listings} % Syntax highlighing
\usepackage{fancyvrb} % Inline verbatim
\usepackage{hyperref} % Hyperlinks
\hypersetup{pdfpagemode=FullScreen}

\usepackage[normalem]{ulem}               % to striketrhourhg text
\newcommand\redout{\bgroup\markoverwith
{\textcolor{red}{\rule[0.5ex]{2pt}{0.8pt}}}\ULon}

% header in tables
\newcommand*{\thead}[1]{\multicolumn{1}{c}{\bfseries #1}}

% used for arrows from one point in the slide to another
\usepackage{tikz}
\usetikzlibrary{arrows,shapes,tikzmark}

\usetheme{Boadilla}
\title{Lecture 15: Functions, Part 2}
\author{UMBC CMSC 104}
\date{Tu 18 Apr 2023}

\begin{document}

\begin{frame}{}
\centering
    Functions, Part 2
\end{frame}

\frame{\tableofcontents}

\section{More Parameters, Return Values}
\begin{frame}[fragile]{Functions Can Return Values}
    \begin{lstlisting}[language=C,basicstyle=\footnotesize,keywordstyle=\color{blue},commentstyle=\color{green},showstringspaces=false,stringstyle=\color{red}]
/***********************************************************
** averageTwo - calculates and returns the average of
**              two numbers
** Inputs:  num1 - an integer value
**          num2 - an integer value
** Outputs:  the floating point average of num1 and num2
************************************************************/
float AverageTwo(int num1, int num2) {
	float average;   /* average of the two numbers */

	average = (num1 + num2) / 2.0;
	return average;
}
    \end{lstlisting}
\end{frame}

\begin{frame}[fragile]{Using averageTwo()}
    \begin{lstlisting}[language=C,basicstyle=\footnotesize,keywordstyle=\color{blue},commentstyle=\color{green},showstringspaces=false,stringstyle=\color{red}]
#include <stdio.h>

float averageTwo(int num1, int num2);

int main() {
    float avg;
    int value1 = 5, value2 = 8
    avg = averageTwo(value1, value2);
    printf("The average of %d and %d is %f\n", value1,
        value2, avg);
    return 0;
}

float AverageTwo(int num1, int num2) {
	float average;   /* average of the two numbers */

	average = (num1 + num2) / 2.0;
	return average;
}
    \end{lstlisting}
\end{frame}

\begin{frame}{Parameter Passing}
    \begin{itemize}
        \item \textbf{Actual parameters} are the parameters that appear in the function call. \texttt{average = averageTwo(value1, value2);}
        \item \textbf{Formal parameters} are the parameters which appear in the function header. \texttt{float averageTwo(int num1, int num2)}
        \item Actual and formal parameters are matched by position. Each formal parameter receives the value of its corresponding actual parameter.
        \item Corresponding actual and formal parameters do not have to have the same name, but they may.
        \item Corresponding actual and formal parameters must be of the same data type, with some exceptions.
    \end{itemize}
\end{frame}

\begin{frame}{Local Variables}
    \begin{itemize}
        \item Functions only ``see'' (have access to) their own \textbf{local variables}. This includes \texttt{main()}.
        \item Formal parameters are declarations of local variables. The values passed are assigned to those variables.
        \item Other local variables can be declared within the function body.
        \item Changes to the parameters by the function is allowed, but will not change the values of the parameters in the calling function.
    \end{itemize}
\end{frame}

\section{Header Files}
\begin{frame}{Header Files}
    \begin{itemize}
        \item Header files contain function prototypes for all of the functions found in the specified library.
        \item They also contain definitions of constants and data types used in that library.
        \item By reusing code, especially from the standard C library, you reduce the chance for errors while making the code easier to maintain (instead of re-implementing everything yourself)
    \end{itemize}
\end{frame}

\begin{frame}{Commonly Used Header Files}
    \begin{tabular}{l l}
        Header File & Contains Function Prototypes for: \\ \hline
        stdio.h  & standard input/output library functions, file input/output \\
        math.h   & math functions \\
        stdlib.h & conversions of data types, allocating memory, other utilities \\
        assert.h & creating assertions, ensuring certain conditions are met \\
        complex.h & working with complex numbers \\
        ctype.h  & functions for testing characters and data types \\
        dirent.h & working with directories \\
        errno.h  & error handling \\
        limits.h & sizes of basic types \\
        locale.h & translating programs into different human languages \\
        stddef.h & working with different specific-sized data types \\
        string.h & functions for working with strings \\
        time.h   & getting date and time information \\
        unistd.h & standard \underline{cross-Unix} \& \underline{Unix-like} operating systems functions \\
                 & \textit{and there are a lot more!}
    \end{tabular}
\end{frame}

\begin{frame}[fragile]{Using Header Files}
    \begin{lstlisting}[language=C,basicstyle=\footnotesize,keywordstyle=\color{blue},commentstyle=\color{green},showstringspaces=false,stringstyle=\color{red}]
#include <stdio.h>
#include <stdlib.h> // for exit()
#include <math.h>   // for sqrt()

int main() {
    float side1, side2, hypotenuse;
    printf("Enter the lengths of the right triangle sides: ");
    scanf("%f%f", &side1, &side2);
    if (side1 <= 0 || side2 <= 0) {
        exit(1);
    }
    hypotenuse = sqrt((side1*side1), (side2*side2));
    printf("The hypotenuse = %f\n", hypotenuse);
    return 0;
}
    \end{lstlisting}
    
\end{frame}

\end{document}