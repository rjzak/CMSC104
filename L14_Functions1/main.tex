% Copyright 2002-2023 The University of Maryland Baltimore County (UMBC)
% 1000 Hilltop Circle, Baltimore, Maryland, 21250, USA
% https://www.csee.umbc.edu/

\documentclass[graphics]{beamer}
\usepackage{graphicx}
\usepackage{listings} % Syntax highlighing
\usepackage{fancyvrb} % Inline verbatim
\usepackage{hyperref} % Hyperlinks
\hypersetup{pdfpagemode=FullScreen}

\usepackage[normalem]{ulem}               % to striketrhourhg text
\newcommand\redout{\bgroup\markoverwith
{\textcolor{red}{\rule[0.5ex]{2pt}{0.8pt}}}\ULon}

% header in tables
\newcommand*{\thead}[1]{\multicolumn{1}{c}{\bfseries #1}}

% used for arrows from one point in the slide to another
\usepackage{tikz}
\usetikzlibrary{arrows,shapes,tikzmark}

\usetheme{Boadilla}
\title{Lecture 14: Functions, Part 1}
\author{UMBC CMSC 104}
\date{Th 13 Apr 2023}

\begin{document}

\begin{frame}{}
\centering
    Functions, Part 1
\end{frame}

\frame{\tableofcontents}

\section{Concepts}
\begin{frame}{Review of Structured Programming}
    Structured programming is a problem solving strategy and a programming methodology that includes the following guidelines:
    \begin{itemize}
        \item The program uses only the sequence, selection, and repetition control structures.
        \item The flow of control in the program should be as simple as possible.
        \item The construction of a program embodies \textbf{top-down} design.
    \end{itemize}
\end{frame}

\begin{frame}{Review of Top-Down Design}
    \begin{itemize}
        \item Involves repeatedly \textbf{decomposing} a program into smaller problems.
        \item Eventually leads to a collection of small problems\footnote{This is also seen in the Unix design philosophy \url{https://en.wikipedia.org/wiki/Unix_philosophy}} or tasks, each of which can be easily coded.
        \item The \textbf{function} construct in C is used to write code for these small, simple programs.
    \end{itemize}
\end{frame}

\section{Functions}
\begin{frame}{Functions}
    \begin{itemize}
        \item A C program is made up of one or more functions, one of which is \texttt{main()}.
        \item Execution always begins with \texttt{main()}, no matter where it is placed in the program. By convention, \texttt{main()} is located before all other functions.
        \item When program control encounters a function name, the function is \texttt{called} (\texttt{invoked}).
        \begin{itemize}
            \item Program control passes to the function.
            \item The function is executed.
            \item Control is passed back to the calling function when finished.
        \end{itemize}
    \end{itemize}
\end{frame}

\begin{frame}[fragile]{Sample Function Call}
    \begin{lstlisting}[language=C,basicstyle=\footnotesize,keywordstyle=\color{blue},commentstyle=\color{green},showstringspaces=false,stringstyle=\color{red}]
#include<stdio.h>

int main() {
    // printf is the name of a predefined function in stdio.h
    // this is a function call
    // "Hello, world!\n" is an argument which is passed to
    //    printf()
    printf("Hello, world!\n");
    return 0;
}
    \end{lstlisting}
\end{frame}

\begin{frame}{Functions Cont'd}
    \begin{itemize}
        \item We have used five predefined functions so far:
        \begin{itemize}
            \item \texttt{printf()}
            \item \texttt{scanf()}
            \item \texttt{getchar()}
            \item \texttt{srandom()}, \texttt{random()} (CW 6)
        \end{itemize}
        \item Programmers can (and should) write their own functions.
        \item Typically, each module in the design hierarchy of a program is implemented as a function.
    \end{itemize}
\end{frame}

\subsection{Examples}
\begin{frame}[fragile]{Sample Programmer-Designed Function}
    \begin{lstlisting}[language=C,basicstyle=\footnotesize,keywordstyle=\color{blue},commentstyle=\color{green},showstringspaces=false,stringstyle=\color{red}]
#include<stdio.h>

void PrintMessage();

int main() {
    PrintMessage();
    return 0;
}

void PrintMessage() {
    printf("A message for you:\n\n");
    printf("Have a nice day!\n");
}
    \end{lstlisting}
\end{frame}

\begin{frame}[fragile]{Examining the PrintMessage()}
    \begin{lstlisting}[language=C,basicstyle=\footnotesize,keywordstyle=\color{blue},commentstyle=\color{green},showstringspaces=false,stringstyle=\color{red}]
#include<stdio.h>

void PrintMessage(void); // Function prototype

int main() {
    PrintMessage(); // Function call
    return 0;
}

void PrintMessage(void) { // Function header

    // Function body
    printf("A message for you:\n\n");
    printf("Have a nice day!\n");
}
    \end{lstlisting}
\end{frame}

\subsection{Function Components}
\begin{frame}{The Function Prototype}
    \begin{itemize}
        \item It informs the compiler that there will be a function defined later.
        \item This is needed because the function call is made before the definition -- the compiler uses it to see if the call is made properly.
    \end{itemize}
\end{frame}

\begin{frame}{The Function Call}
    \begin{itemize}
        \item Passes program control to the function.
        \item Must match the function prototype in name, number of arguments, and data types of the arguments.
    \end{itemize}
\end{frame}

\begin{frame}{The Function Definition}
    \begin{itemize}
        \item Control is passed to the function by the function call. The statements within the function body will then be executed.
        \item After the statements in the function have completed, control is passed back to the \textbf{calling function}, in this case, \texttt{main()}. The calling function doesn't have to be \texttt{main()}, a function can be called from another function.
    \end{itemize}
\end{frame}

\begin{frame}[fragile]{General Function Definition Syntax}
    \begin{verbatim}
type functionName(parameter_1, ..., parameter_n) {
    variable declaration(s);
    statement(s);
    return varName; // if there is something to return
}
    \end{verbatim}
    \begin{itemize}
        \item If there are no parameters, either \texttt{functionName()} or \texttt{functionName(void)} is acceptable.
        \item If the \textbf{function type} (\textbf{return type}) is void, a return statement is not required.
    \end{itemize}
\end{frame}

\begin{frame}[fragile]{Using Input Parameters}
    \begin{lstlisting}[language=C,basicstyle=\footnotesize,keywordstyle=\color{blue},commentstyle=\color{green},showstringspaces=false,stringstyle=\color{red}]
#include<stdio.h>

void PrintMessage(int counter); // no return, receives int

int main() {
    int num;
    printf("Enter an integer: ");
    scanf("%d", &num);
    PrintMessage(num); // one argument, type int
    return 0;
}

void PrintMessage(int counter) { // no return, receives int
    int i;
    for(i = 0; i < counter; i++) {
        printf("Have a nice day!\n");
    }
}
    \end{lstlisting}
\end{frame}

\begin{frame}[fragile]{Function Comment}
    \begin{lstlisting}[language=C,basicstyle=\footnotesize,keywordstyle=\color{blue},commentstyle=\color{green},showstringspaces=false,stringstyle=\color{red}]
/*************************************************************
** PrintMessage - prints a message a specified number of times
** Inputs:  counter - the number of times the message will be
**                             printed
** Outputs:  None
/*************************************************************/
void PrintMessage(int counter) {
    int i;
    for(i = 0; i < counter; i++) {
        printf("Have a nice day!\n");
    }
}
    \end{lstlisting}
\end{frame}

\section{Good Programming Practice}
\begin{frame}{Good Programming Practice}
    \begin{itemize}
        \item A \textbf{function header comment} before the definition of a function is a good practice, and is required by the CMSC 104 Coding Standards.
        \item Your header comments should be neatly formatted and contain the following information:
        \begin{itemize}
            \item Function name
            \item Function description -- what it does
            \item A list of any input parameters and their meanings
            \item A list of any output parameters and their meanings
            \item A description of any special conditions, if any.
        \end{itemize}
    \end{itemize}
\end{frame}

\end{document}