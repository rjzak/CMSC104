\documentclass[letter,11pt]{article}
\usepackage[breaklinks]{hyperref}
\hypersetup{
    bookmarks=true,         % show bookmarks bar?
    unicode=false,          % non-Latin characters in Acrobat’s bookmarks
    pdftoolbar=true,        % show Acrobat’s toolbar?
    pdfmenubar=true,        % show Acrobat’s menu?
    pdffitwindow=false,     % window fit to page when opened
    pdfstartview={XYZ null null 1.00},    % disable zoom
    pdftitle={Practice Final Exam},    % title
    pdfauthor={Richard Zak},     % author
    pdfsubject={UMBC CMSC104 Problem Solving and Computer Programming},   % subject of the document
    pdfkeywords={Computer Science, Programming, Problem Solving, CSEE}, % list of keywords
    pdfnewwindow=true,      % links in new PDF window
    colorlinks=false,       % false: boxed links; true: colored links
    linkcolor=red,          % color of internal links (change box color with linkbordercolor)
    citecolor=green,        % color of links to bibliography
    filecolor=magenta,      % color of file links
    urlcolor=cyan           % color of external links
}
\usepackage{graphicx}
\usepackage{fancyhdr}
\usepackage{multicol}
\pagestyle{fancy}
\usepackage[letterpaper, margin=1in]{geometry}
\geometry{letterpaper}
\usepackage{parskip} % Disable initial indent
\usepackage{color,soul} % Highligher
\usepackage[normalem]{ulem} % Strikethrough with \sout{}

\usepackage[utf8]{inputenc}
\fancyhf{}
\renewcommand{\headrulewidth}{0pt} % Remove default underline from header package
\rhead{CMSC 104 Section 01: Practice Final}
%\rhead{}
\lhead{\begin{picture}(0,0) \put(0,-10){\includegraphics[width=1.1cm]{Images/UMBC-vertical}} \end{picture}}
\cfoot{\thepage}
\rfoot{\input{semester}}
\lfoot{CMSC 104 Section 01}
\AtEndDocument{\vfill \footnotesize{Last modified: 08 December 2021}}
\AtEndDocument{\rfoot{\input{semester}}}
\renewcommand\thesubsection{\arabic{subsection}} % Show only subsection numbers, not section.subsection

\begin{document}

\huge
\textbf{Practice Final}
\normalsize

\paragraph{}Please answer the following questions. Partial credit may be given for incomplete free response questions.

\begin{enumerate}
    %
    % Hardware questions
    %
    \item Order the following storage devices from fastest to slowest:
    \begin{itemize}
        \item \underline{~~ ~~} Tape
        \item \underline{~~ ~~} Spinning hard drive
        \item \underline{~~ ~~} Optical disc
        \item \underline{~~ ~~} USB thumb drive
        \item \underline{~~ ~~} Solid state hard drive
        \item \underline{~~ ~~} Floppy disk
    \end{itemize}
    
    \item Order the following memory (RAM) devices from fastest to slowest:
    \begin{itemize}
        \item \underline{~~ ~~} RAM
        \item \underline{~~ ~~} Paged memory saved paged to disk
        \item \underline{~~ ~~} CPU registers
    \end{itemize}
    
    \item Which item doesn't belong?
    \begin{itemize}
        \item Mouse
        \item Scanner
        \item Keyboard
        \item Printer
        \item Joystick
        \item Bar code reader
    \end{itemize}
    
    \item Which is considered the ``brain'' of the hardware?
    \begin{itemize}
        \item Mouse
        \item CPU
        \item RAM
        \item Hard drive
        \item Network card
    \end{itemize}
    
    \item Which is considered the ``brain'' of the software?
    \begin{itemize}
        \item Compiler
        \item Web browser
        \item Operating System
        \item RAM
    \end{itemize}
    
    \item What might you do with a supercomputer?
    
    %
    % Linux commands
    %
    \item Which is the command used to compile C code?
    \begin{itemize}
        \item \texttt{ld}
        \item \texttt{gcc}
        \item \texttt{vi}
        \item \texttt{g++}
        \item \texttt{gdb}
    \end{itemize}
    
    %
    % Pseudocode questions
    %
    \item What is the purpose of pseudocode? Select all that apply.
    \begin{itemize}
        \item To run the code directly.
        \item To plan how the algorithm will work.
        \item To separate the algorithm from C code.
        \item To make a more verbose version of the algorithm, so we'll know how many lines of code our program will need.
    \end{itemize}
    
    %
    % Numeric formats questions
    %
    \item Convert the following decimals to hexadecimal
    \begin{itemize}
        \item 10
        \item 42
        \item 763
        \item 19
    \end{itemize}
    
    \item Convert the following hexadecimal numbers to octal
    \begin{itemize}
        \item 0x12
        \item 0xF00D
        \item 0xBEEF
        \item 0xCAFE
    \end{itemize}
    
    \item Convert the following binary to hexadecimal
    \begin{itemize}
        \item 01010101
        \item 10101010
        \item 11001100
        \item 00100100
    \end{itemize}
    
    %
    % Coding questions
    %
    \item Explain when you'd use each of the following control structures:
    \begin{itemize}
        \item \texttt{for} loop
        \item \texttt{while} loop
        \item \texttt{do while} loop
        \item \texttt{switch} statement
        \item \texttt{if else} statement
    \end{itemize}
    
    \item Which loop would you use for each scenario?
    \begin{itemize}
        \item Asking the user if the program has guessed their number or not.
        \item Printing the name of each of the 20 students in the class.
        \item Receiving grades for a class until the user decided they are done.
        \item Reading in a list of numbers from a file.
    \end{itemize}
    
    \item If you have an array of \texttt{short} integers which starts at memory address 0xFE00, what's the memory address of the second value?
    \begin{itemize}
        \item 0xFE00
        \item 0xFE01
        \item 0xFE02
        \item 0xFE03
        \item 0xFE04
    \end{itemize}
    
    \item Given: \texttt{long myNum = 7063;}
    \begin{itemize}
        \item How much memory does it take up, in \textit{bytes}?
        \item How much memory does it take up, in \textit{bits}?
        \item If you multiply that number by another variable, defined as \texttt{int otherNum = 3;}, which is the resulting data type?
    \end{itemize}
    
    \item Write code to create an array of 10 integers, setting each element to the next even number, starting with 2.
    
    \item Given an array of ten \texttt{float}s called \texttt{nums}, write code which:
    \begin{itemize}
        \item Set the 3\textsuperscript{rd} element to 7.
        \item Swap the 7\textsuperscript{th} and 8\textsuperscript{th} elements.
        \item Divide element 4 by element 6, storing the value in the last element.
        \item Increment the first element.
    \end{itemize}
    
    \item In your code, you have a function which might need to exit the entire program early if there is an irrecoverable error condition. How would you do that?
    \begin{itemize}
        \item \texttt{return void;}, regardless of the actual data type the function is supposed to return.
        \item Add \texttt{\#include<stdlib.h>} and call \texttt{exit(1);}.
        \item \texttt{return -1;}
        \item Trick question, errors could never occur in functions.
    \end{itemize}
    
    \item Given: \texttt{int a = 1, b = 2, c = 3;}, evaluate the following expressions independently for their numeric value, and indicate if the expression is true or false.
    \begin{itemize}
        \item \texttt{a++ - b / ++c}
        \item \texttt{++c / ++b \% ++a}
        \item \texttt{++c / --a}
    \end{itemize}
    
    \item Given: \texttt{int a = 2, b = 4, c = 6;}, evaluate the following expressions in order. Given the resulting value for each variable.
    \begin{itemize}
        \item \texttt{a += ++b / c;}
        \item \texttt{b /= ++c - b;}
        \item \texttt{c *= ++a - b--};
        \item \texttt{a}=\underline{~~ ~~}
        \item \texttt{b}=\underline{~~ ~~}
        \item \texttt{c}=\underline{~~ ~~}
    \end{itemize}
    
    \item When might you use...
    \begin{itemize}
        \item an \texttt{unsigned} data type?
        \item a \texttt{double} over a \texttt{float}?
        \item a \texttt{double} or a \texttt{float} over an \texttt{int}?
        \item a \texttt{long} over an \texttt{int}?
    \end{itemize}
    
    \item Write function prototypes for the following:
    \begin{itemize}
        \item A function called ``foo'' which takes in an integer ``bar'' and doesn't return a value.
        \item A function called ``average'' which takes an integer array ``data'' and an integer ``size'' and returns a \texttt{float}.
        \item A function called ``force'' which returns a \texttt{double} and takes two \texttt{double} parameters, ``mass'' and ``acceleration''.
    \end{itemize}
    
    \item Briefly describe each operator:
    \begin{itemize}
        \item \texttt{++}
        \item \texttt{--}
        \item \texttt{/=}
        \item \texttt{[ ]}
        \item \texttt{*}
        \item \texttt{\&}
    \end{itemize}
    
    \item What does this function do?
    \begin{verbatim}
int mysteryFunction(int index) {
    if (index <= 0) {
        return 0;
    }
    int first = 0, next = 1;
    int counter;
    for(counter = 2; counter < index; counter++) {
        int temp = first + next;
        first = next;
        next = temp;
    }
    return next;
}
    \end{verbatim}
    
    \item Review the following code and answer the questions:
    \begin{verbatim}
#include<stdio.h>

int multiplyTwoNumbers(int,int);

int main() {
    int num1 = 100, num2 = 20, num3;
    num3 = multiplyTwoNumbers(num1, num2);
    printf("Multiplying num1 * num2 = %d\n", num3);
    return 0;
}

int multiplyTwoNumbers(int num2, int num1) {
    return num1 * num2;
}
    \end{verbatim}
    \begin{itemize}
        \item What is the value of \texttt{num2} in \texttt{multiplyTwoNumbers()}?
        \item What is the value of \texttt{num1} in \texttt{multiplyTwoNumbers()}?
        \item What is the value of \texttt{num2} after the \texttt{printf()} statement in \texttt{main()}?
        \item What is the value of \texttt{num1} after the \texttt{printf()} statement in \texttt{main()}?
        \item What is the value of \texttt{num3} after the \texttt{printf()} statement in \texttt{main()}?
    \end{itemize}
    
    \item You have an array of 20 floats, stored in increasing order in an array called \texttt{averages}. What code to do the following:
    \begin{itemize}
        \item Change the 3\textsuperscript{rd} highest value to 49.9.
        \item Change the 7\textsuperscript{th} lowest value to 2.1.
        \item Add 1 using the \texttt{++} operator to the value in position 5.
        \item Subtract 3 using the \texttt{-=} operator from the last value.
    \end{itemize}
\end{enumerate}

\end{document}