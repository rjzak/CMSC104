\documentclass[letter,11pt]{article}
\usepackage[breaklinks,pdfa=true,bookmarks=true,pdfdisplaydoctitle=true]{hyperref}
\hypersetup{
    unicode=false,          % non-Latin characters in Acrobat’s bookmarks
    pdftoolbar=true,        % show Acrobat’s toolbar?
    pdfmenubar=true,        % show Acrobat’s menu?
    pdffitwindow=false,     % window fit to page when opened
    pdfstartview={XYZ null null 1.00},    % disable zoom
    pdftitle={Problem Solving and Computer Programming Syllabus},    % title
    pdfauthor={Richard Zak},     % author
    pdfsubject={UMBC CMSC104 Problem Solving and Computer Programming},   % subject of the document
    pdfkeywords={Computer Science, Programming, Problem Solving, CSEE}, % list of keywords
    pdfnewwindow=true,      % links in new PDF window
    colorlinks=true,       % false: boxed links; true: colored links
    linkcolor=red,          % color of internal links (change box color with linkbordercolor)
    citecolor=green,        % color of links to bibliography
    filecolor=magenta,      % color of file links
    urlcolor=cyan           % color of external links
}
\usepackage{graphicx}
\usepackage{fancyhdr}
\usepackage{multicol}
\pagestyle{fancy}
\usepackage[letterpaper, margin=1in]{geometry}
\geometry{letterpaper}
\usepackage{parskip} % Disable initial indent
\usepackage{color,soul} % Highligher
\usepackage[normalem]{ulem} % Strikethrough with \sout{}

\definecolor{cadmiumgreen}{rgb}{0.0, 0.42, 0.24}

\usepackage[utf8]{inputenc}
\fancyhf{}
\renewcommand{\headrulewidth}{0pt} % Remove default underline from header package
\rhead{CMSC 104 Section 01: Problem Solving and Computer Programming}
%\rhead{}
\lhead{\begin{picture}(0,0) \put(-65,-8){\includegraphics[width=1.1cm]{Images/UMBC-vertical}} \end{picture}}
\cfoot{\thepage}
\rfoot{\input{semester}}
\lfoot{CMSC 104 Section 01}
%\AtEndDocument{\vfill \LaTeX \hfill}
\AtEndDocument{\hfill \vfill Last modified: 21 April 2022}

\begin{document}

\textbf{Semester:} \input{semester}\\
\textbf{Location:} Engineering 122A \\
\textbf{Time:} Tuesdays \& Thursdays 5:30 -- 6:45p \\
\textbf{Instructor:} Richard Zak \\
\textbf{Email:} \href{mailto:richard.zak@umbc.edu?Subject=CMSC104}{richard.zak@umbc.edu} \\
\textbf{Teaching Fellow:} Fredrick Gough \\
\textbf{Email:} \href{mailto:vf15810@umbc.edu?Subject=CMSC104}{vf15810@umbc.edu}

\section*{Day/Hours Available}
\begin{itemize}
\item We can meet 30 minutes prior to class if you coordinate with us prior to the day of class. We can also stay after class for about 30 minutes.
\item In Email: Anytime, and we will respond within 48 hours.
\end{itemize}

%\section*{Blackboard Collaborate}
%\paragraph{}Do to the COVID-19 pandemic, class will meet virtually. Blackboard Collaborate is accessible via \url{https://blackboard.umbc.edu}, displayed on the side menu bar as "Bb Collaborate". For information on UMBC's response to the Coronavirus outbreak, visit the UMBC website for the Coronavirus at \url{https://covid19.umbc.edu/}.

\section*{Course Description}
\paragraph{}This course is designed to provide an introduction to problem solving and computer programming that does not require prior programming experience. Elementary problem solving skills and algorithm development will be introduced. Students will be taught the basic use of a programming environment and basic programming constructs (including loops, control statements, functions, and arrays). This course also teaches students the fundamentals of using the UNIX operating system, and introduces general computer science concepts.
\paragraph{Note:}This course does not fulfill any of the computer science major requirements. Students who have taken and received transfer credit for, or who are taking concurrently any computer programming course in a high-level programming language, will not receive credit for CMSC 104. The list of such computer programming courses includes, but is not limited to AP Computer Science, CMSC 201, CMSC 202, and sections of CMSC 291 that cover programming topics.

The following is a list of the topics that will be covered:
\begin{itemize}
\item Introduction to Computer Organization and Architecture
\item Data Representation and Memory Usage
\item Introduction to Operating Systems (Linux)
\item Introduction to Software Engineering Using Top-Down Design
\item Programming in C
\item Problem Solving and Algorithm Development
\end{itemize}

\section*{Overall Course Objectives}
\paragraph{}After completion of this course, students will be able to:
\begin{multicols}{2}
\begin{itemize}
    \item describe basic computer hardware and software,
    \item navigate the Linux command line using UMBC's Linux GL environment,
    \item create, edit, and remove files and directories in Unix,
    \item write C programs with loops, control statements, functions, and arrays that compile and solve their intended problems.
\end{itemize}
\end{multicols}

\section*{Course Textbook}
\textbf{C How to Program} by Deitel \& Deitel, Eighth Edition. Publisher: Pearson; ISBN 978-0133976892.

\section*{Course Work}
\paragraph{Attendance}It is important to be present for each class, though it's up to each individual student to best manage their own time. This is your time to have regular interaction with the instructor and Teaching Fellow, to ask questions, and seek clarification. I cannot overstate the importance of being able to ask questions and engage in dialog to help facilitate the learning process.

\paragraph{Classwork \& Homework}Assignments will be submitted using the submit system on GL (ssh userID@gl.umbc.edu). All assignments are due one week after being assigned. For example, Classwork 1 is assigned on Thursday, August 26\textsuperscript{th}, therefore the due date is 11:59 PM on Thursday, September 2\textsuperscript{nd}. An example of the submit command: \texttt{submit cs104 cw1 proj.txt file2.ext}. Assignments may be completed early, but late assignments will \underline{not} be accepted.

\paragraph{Software} Windows users will need Putty \url{https://www.chiark.greenend.org.uk/~sgtatham/putty/latest.html} and WinSCP \url{https://winscp.net/eng/index.php} for connecting to the UMBC Linux servers and transferring files. Both programs are free. Mac users may want to use Cyberduck \url{https://cyberduck.io/} or Transmit \url{https://panic.com/transmit/} to help with file transfers, but this isn't required, as the Mac command line allows for SSH connections and file transfers natively.

\paragraph{Quizzes}There are five quizzes, and they will be held during the first 40 minutes of the class in which they're administered. The best way to study for the quiz is to do your homework and learn from it. Each quiz will focus on the materials since the quiz prior, including assigned readings, and will be \underline{cumulative}. Quizzes will be administered using the Respondus Lockdown Browser in Blackboard, so please read the following section for details on how to set it up and use it. There will be a practice quiz early in the semester to ensure everyone can use the system.

\paragraph{LockDown Browser}This course uses the LockDown Browser for online exams. Watch this short video to get a basic understanding of LockDown Browser and the optional webcam feature (which may be required for some exams) \url{http://www.respondus.com/products/lockdown-browser/student-movie.shtml}. Then download and install LockDown Browser from this link: \url{http://www.respondus.com/lockdown/download.php?id=978442813}. NOTE: This link is unique for UMBC. It cannot be used by non-UMBC students or websites outside of our campus. To take an online test, start LockDown Browser and navigate to the exam. You won't be able to access the exam with a standard web browser. For additional details on using LockDown Browser, review this FAQ \url{https://wiki.umbc.edu/x/BQb9Aw} or the Respondus Student Quick Start Guide (PDF) \url{http://respondus.com/products/lockdown-browser/guides.shtml#student}. Finally, when taking an online exam, follow these guidelines:
\begin{itemize}
%\item Select a location where you won't be interrupted
\item Before starting the test, know how much time is available for it, and that you've allotted sufficient time to complete it
\item Turn off all mobile devices, phones, etc. and don't have them within reach
%\item Clear your area of all external materials — books, papers, other computers, or devices
\item Remain at your desk or workstation for the duration of the test
\item LockDown Browser will prevent you from accessing other websites or applications; you will be unable to exit the test until all questions are completed and submitted
\end{itemize}

\paragraph{Exams}There will be one exam, the final, which will also use the LockDown Browser. It is designed to reinforce the topics discussed throughout the semester in order to promote retention of the information.

\section*{Course Policies}
\subsection*{Grading}

\begin{center}
\begin{tabular}{ c c || c c}
\multicolumn{2}{c||}{\underline{Scale}} & \underline{Course Work} & \underline{Grade Distribution} \\
90\%-100\% & A & Classwork & 15\% \\
80\%-89\% & B & Homework & 25\% \\
70\%-79\% & C & Quizzes & 30\% \\ 
60\%-69\% & D & Final Exam & 30\% \\
$<$60\%     & F & 
\end{tabular}
\end{center}

\paragraph{}For borderline grades, there may be adjustments in the student's favor based on attendance, but under no circumstances will the letter grades be lower than in the standard formula. Grades will not be ``curved'' in the sense that the percentages of A's, B's and C's are not fixed. As a guideline, a student receiving an ``A'' should be able to produce correct programs for the homework assignments and quizzes with ease.

\subsection*{Academic Integrity}
\paragraph{}By enrolling is this course, each student assumes the responsibilities of an active participant in UMBC's scholarly community in which everyone's academic work and behavior are held to the highest standards of honesty. Cheating, fabrication, plagiarism, and helping others to commit these acts are all forms of academic dishonesty, and they are wrong. Academic misconduct could result in disciplinary action that may include, but is not limited to, suspension or dismissal. To find useful information about avoiding plagiarism infractions through appropriate citations, or to read the full policy regarding student academic misconduct for the graduate school, please see \url{http://www.umbc.edu/provost/integrity}.

\paragraph{}While you are allowed \& encouraged to discuss course materials with your peers, \underline{all assignments} \underline{are an individual effort!} Assignments will be checked for evidence of shared or copied code. If you have questions, or find that you are struggling, please contact me and/or Fredrick, \underline{do not share code}\textbf{!}

\section*{Schedule}
\textit{Note: The schedule is subject to change. However, any changes will be thoroughly discussed and well disseminated. The lecture \& textbook topics might not align, that's okay, please read the chapter before class starts anyway.}

\small
\begin{tabular}{c c l l c}
Date & Week & ~~~~~~~~~~~~Topic & Chapter & Assignment \\
\hline
Tu Feb 1  & 1 & Introduction, Syllabus Review, Linux @ UMBC & & Classwork 1  \\
Th Feb 3  & 1 & \hl{Quiz 0}, L2: Machine Architecture \& Number Systems & 1 & Classwork 2 \\
Tu Feb 8  & 2 & L3: Operating Systems & & Homework 1 \\
Th Feb 10 & 2 & L4 \& 5: Algorithms & & Classwork 3 \\
Tu Feb 15 & 3 & hello world (Practice) & & Homework 2 \\
Th Feb 17 & 3 & \hl{Quiz 1} & &  \\
Tu Feb 22 & 4 & L6: Introduction to C & 2 & Homework 3\\
Th Feb 24 & 4 & L7: Variables in C & 3 \\ \hline

Tu Mar 1 & 5 & L8: Arithmetic Operators in C & 4 & Classwork 4 \\
Th Mar 3 & 5 & \texttt{scanf} demo & 5 &  \\
Tu Mar 8 & 6 & L9: Relational \& Logical Operators & 6 & Homework 4 \\
Th Mar 10 & 6 & L10: While Loops & 7 & Classwork 5 \\
Tu Mar 15 & 7 & while loop demo & & Classwork 6 \\
Th Mar 17 & 7 & \hl{Quiz 2} & & \\
Tu Mar 22 & 8 & {\color{cadmiumgreen}Spring Break!} & & \\
Th Mar 24 & 8 & {\color{cadmiumgreen}Spring Break!} & &  \\
Tu Mar 29 & 9 & L11: More Loops, For Loops demo & 8 &  \\
Th Mar 31 & 9 & L12: Assignment Operators & 9 & Homework 5\\ \hline

Tu Apr 5 & 10 & \hl{Quiz 3} & & \\
Th Apr 7 & 10 & L13: Switch Statements & 10 & \\
Tu Apr 12 & 11 & Functions demo & 5 & Homework 6\\
Th Apr 14 & 11 & L14: Functions, Part I & & Classwork 7  \\
Tu Apr 19 & 12 & L15: Functions, Part II & &  Homework 7 \\
Th Apr 21 & 12 & L16: Functions, Part III  & & Classwork 8 \\
Tu Apr 26 & 13 & L17: Header Files & & Homework 8\\
Th Apr 28 & 13 & \hl{Quiz 4} & 6 & Classwork 9 \\ \hline

Tu May 3 & 14 & L18: Arrays, Part I & & Homework 9 \\
Th May 5 & 14 & L19: Arrays, Part II  &  &  \\
Tu May 10 & 15 & \hl{Quiz 5} & & Classwork 10 \\
Th May 12 & 15 & Demos &  & Homework 10 \\
Tu May 17 & 16 & Review & \\
Th May 19 & 16 & \hl{Final Exam, 6-8PM} & \\
\end{tabular}

%\section*{Title IX Statement}
%\paragraph{}As an instructor, I am considered a Responsible Employee, per UMBC’s Policy on Prohibited Sexual Misconduct, Interpersonal Violence, and Other Related Misconduct \footnote{\url{http://humanrelations.umbc.edu/sexual-misconduct/umbc-resource-page-for-sexual-misconduct-and-other-related-misconduct/}}. While my goal is for you to be able to share information related to your life experiences through discussion and written work, I want to be transparent that as a Responsible Employee I am required to report disclosures of sexual assault, domestic violence, relationship violence, stalking, and/or gender-based harassment to the University’s Title IX Coordinator.

%\paragraph{}As an instructor, I also have a mandatory obligation to report disclosures of or suspected instances of child abuse or neglect\footnote{\url{http://www.usmh.usmd.edu/regents/bylaws/SectionVI/VI150.pdf}}.

%\paragraph{}The purpose of these reporting requirements is for the University to inform you of options, supports and resources; you will not be forced to file a report with the police. Further, you are able to receive supports and resources, even if you choose to not want any action taken. Please note that in certain situations, based on the nature of the disclosure, the University may need to take action.

\section*{Accessibility \& Disability Accommodations, Guidance \& Resources}
\paragraph{}Accommodations for students with disabilities are provided for all students with a qualified disability under the Americans with Disabilities Act (ADA \& ADAAA) and Section 504 of the Rehabilitation Act who request and are eligible for accommodations. The Office of Student Disability Services (SDS) is the UMBC department designated to coordinate accommodations that creates equal access for students when barriers to participation exist in University courses, programs, or activities.

\paragraph{}If you have a documented disability and need to request academic accommodations in your courses, please refer to the SDS website at \url{http://sds.umbc.edu} for registration information and office procedures.
\begin{itemize}
\item SDS email: \href{mailto:disAbility@umbc.edu}{disAbility@umbc.edu}
\item SDS phone: \href{tel:+14104552459}{(410) 455-2459}
\end{itemize}
If you will be using SDS approved accommodations in this class, please contact the instructor to discuss implementation of the accommodations. During remote instruction requirements due to COVID, communication and flexibility will be essential for success.

\section*{Sexual Assault, Sexual Harassment, \& Gender Based Violence \& Discrimination}
\paragraph{}UMBC Policy\footnote{\url{https://oei.umbc.edu/gender-discrimination-sexual-misconduct/}} and Federal law (Title IX) prohibit discrimination and harassment on the basis of sex, sexual orientation, and gender identity in University programs and activities. Any student who is impacted by sexual harassment, sexual assault, domestic violence, dating violence, stalking, sexual exploitation, gender discrimination, pregnancy discrimination, gender-based harassment or retaliation should contact the University’s Title IX Coordinator\footnote{\url{https://oei.umbc.edu/title-ix-coordinator/}} to make a report and/or access support and resources:
\begin{itemize}
\item Mikhel A. Kushner, Title IX Coordinator (she/they)
\item \href{tel:+14104551250}{(410) 455-1250} (direct line), \href{mailto:kushner@umbc.edu?Subject=Title\%20IX}{kushner@umbc.edu}
\end{itemize}

\paragraph{}You can access support and resources even if you do not want to take any further action. You will not be forced to file a formal complaint or police report. Please be aware that the University may take action on its own if essential to protect the safety of the community. If you are interested in or thinking about making a report, please use the Online Reporting/Referral Form. Please note that, if you report anonymously,  the University’s ability to respond will be limited.

\section*{Resources to Help you Succeed}
\paragraph{}Click on the following links for helpful resources:
UMBC’s Academic Success Center (ASC) \url{https://academicsuccess.umbc.edu/} provides a range of resources to support students as they progress toward degree completion. They will continue to offer all of their services online. 
The ASC has created a specialized set of Online Learning Resources \url{https://lrc.umbc.edu/online_learning/}, including videos and guides to help students succeed while learning online.
In addition, check out the following resources:
\begin{itemize}
\item Academic Success Center Resources \url{https://academicsuccess.umbc.edu/asc-business-continuity/} include: Online tutoring and writing support, supplemental instruction/peer-assisted study sessions (SI PASS), placement testing, FYI academic alerts, success courses, academic advocacy, academic policy and academic success meetings.
\item Tutoring and Writing Center Appointments \url{https://lrc.umbc.edu/tutor/}b will be online; students can make appointments by going to \url{https://saml2.go-redrock.com/relay.php}.
\item SI PASS \url{https://si.lrc.umbc.edu/} Supplemental Instruction (SI)/ Peer Assisted Study Sessions (PASS). The SI PASS program targets traditionally difficult academic courses, providing regularly scheduled, out-of-class review sessions, happening in Blackboard Collaborate inside your existing Blackboard course.
\item Academic Advocates: Advocates work one-on-one with students who need support navigating academic and institutional challenges, no matter how complex the concerns (i.e., personal, academic, or financial). \url{https://academicadvocacy.umbc.edu/student-referrals/submit-a-referral/}
\item Academic Success Meetings - Schedule a one-to-one virtual meeting with an Academic Success Center Professional who can help you with time management, study skills, and accessing campus resources. \url{https://lrc.umbc.edu/academic-success-meeting/}
\end{itemize}
If you have a question, please contact the ASC at \href{mailto:academicsuccess@umbc.edu}{academicsuccess@umbc.edu}.

%\paragraph{}Additional resources: \url{https://docs.google.com/document/d/1xWWGAR8qEzKYr7qaVHoEhvO6lyXIyn6M3M7EFZPJQgA/edit?usp=sharing}
\end{document}
