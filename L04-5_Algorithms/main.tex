% Copyright 2002-2023 The University of Maryland Baltimore County (UMBC)
% 1000 Hilltop Circle, Baltimore, Maryland, 21250, USA
% https://www.csee.umbc.edu/

\documentclass[graphics]{beamer}
\usepackage{graphicx}
\usepackage{listings} % Syntax highlighing
\usepackage{fancyvrb} % Inline verbatim
\usepackage{hyperref} % Hyperlinks
\hypersetup{pdfpagemode=FullScreen}

\usepackage[normalem]{ulem}               % to striketrhourhg text
\newcommand\redout{\bgroup\markoverwith
{\textcolor{red}{\rule[0.5ex]{2pt}{0.8pt}}}\ULon}

% header in tables
\newcommand*{\thead}[1]{\multicolumn{1}{c}{\bfseries #1}}

% used for arrows from one point in the slide to another
\usepackage{tikz}
\usetikzlibrary{arrows,shapes,tikzmark}

\usetheme{Boadilla}
\title{Lecture 4-5: Algorithms}
\author{UMBC CMSC 104}
\date{Tu 9 Feb 2023}

\begin{document}

\begin{frame}{}
\centering
    Algorithms
\end{frame}

\frame{\tableofcontents}

\section{Part 1: Problem Solving}
\begin{frame}{Problem Solving}
    \begin{itemize}
        \item Problem solving is the process of transforming the description of a problem into the solution of that problem.
        \item We use our knowledge of the problem domain (requirements).
        \item We rely on our ability to select and use appropriate problem-solving strategies, techniques, and tools.
    \end{itemize}
\end{frame}

\subsection{Definition of an Algorithm}
\begin{frame}{Algorithms}
    \begin{itemize}
        \item An \textbf{algorithm} is a step by step solution to a problem.
        \item Why bother writing an algorithm?
        \begin{itemize}
            \item For your own use in the future.  You won’t have to rethink the problem.
            \item So others can use it, even if they know very little about the principles behind how the solution was derived.
        \end{itemize}
    \end{itemize}
\end{frame}

\subsection{Examples of Algorithms}
\begin{frame}{Examples of Algorithms}
    \begin{itemize}
        \item Washing machine instructions
        \item How to make a peanut butter and jelly sandwich
        \item A classic: finding the greatest common divisor (GCD) using Euclid’s Algorithm
    \end{itemize}
\end{frame}

\begin{frame}{Washing Machine Instructions}
    \begin{itemize}
        \item Separate clothes into white clothes and colored clothes.
        \item Add 1 cup of powdered laundry detergent to tub.
        \item For white clothes:
        \begin{itemize}
            \item Set water temperature knob to HOT.
            \item Place white laundry in tub.
        \end{itemize}
        \item For colors:
        \begin{itemize}
            \item Set water temperature knob to COLD.
            \item Place colored laundry in tub.
        \end{itemize}
        \item Close lid and press the start button.
    \end{itemize}
\end{frame}

\begin{frame}{Observations About the Washing Machine Instructions}
    \begin{itemize}
        \item There are a finite number of steps.
        \item We are capable of doing each of the instructions.
        \item When we have followed all of the steps, the washing machine will wash the clothes and then will stop.
    \end{itemize}
\end{frame}

\begin{frame}{Refinement of our Algorithm Definition}
    \begin{itemize}
        \item Old definition:
        \begin{itemize}
            \item An algorithm is a step by step solution to a problem.
        \end{itemize}
        \item Adding to the definition based on our observations:
        \begin{itemize}
            \item An algorithm is a \underline{finite set} of \underline{executable instructions} which \underline{directs a terminating activity}.
        \end{itemize}
    \end{itemize}
\end{frame}

\begin{frame}{How to Make a Peanut Butter and Jelly Sandwich}
    \begin{itemize}
        \item Open a web browser
        \item Go to \url{https://www.google.com}
        \item Search for ``harvard how to make pbj sandwich''
        \item Click any video (or \url{https://youtu.be/okkIyWhN0iQ})
        \item Enjoy
    \end{itemize}
\end{frame}

\subsection{Final Definition}
\begin{frame}{Final Version of the Algorithm Definition}
    \begin{itemize}
        \item Our old definition:
        \begin{itemize}
            \item An algorithm is a finite set of executable instructions that directs a terminating activity.
        \end{itemize}
        \item Final version:
        \begin{itemize}
            \item An algorithm is a finite set of \underline{unambiguous}, executable instructions which directs a terminating activity.
        \end{itemize}
    \end{itemize}
\end{frame}

\begin{frame}{History of Algorithms}
    \begin{itemize}
        \item The study of algorithms began as a subject in mathematics.
        \item The search for algorithms was a significant activity of early mathematicians.
        \item Goal: To find a single set of instructions that can be used to solve any problem of a particular type (a \textbf{general solution}).
    \end{itemize}
\end{frame}

\begin{frame}{Euclid's Algorithm}
    \underline{Problem}: Find the largest positive integer which divides evenly into two given positive integers (the \textbf{greatest common divisor, GCD}). \\ ~~ \\
    \underline{Algorithm}:
    \begin{enumerate}
        \item Assign $M$ and $N$ the values of the larger and smaller of the two positive integers, respectively.
        \item Divide $M$ by $N$ and call the remainder $R$.
        \item If $R$ is not zero, then assign $M$ the value of $N$, assign $N$ the value of $R$, and return to Step 2. Otherwise, the greatest common divisor is the value currently assigned to $N$.
    \end{enumerate}
\end{frame}

\begin{frame}{Finding the GCD of 24 and 9}
    \centering
    \begin{tabular}{ | c | c | c |} \hline
        $M$ & $N$ & $R$ \\ \hline
        24  & \tikz[remember picture] \node[coordinate,anchor=west] (n1) {};9   & \tikz[remember picture] \node[coordinate,anchor=west] (s1) {};6   \\ \hline
        9\tikz[remember picture] \node[coordinate,anchor=north] (n2) {};  & \tikz[remember picture] \node[coordinate,anchor=east] (s2) {};6\tikz[remember picture] \node[coordinate,anchor=north] (s3) {};   & 3   \\ \hline
        6\tikz[remember picture] \node[coordinate,anchor=west] (s4) {};  & \color{red}3   & 0 \\ \hline
    \end{tabular}
    \begin{tikzpicture}[remember picture,overlay]
        \path[draw=orange,thick,->] --  (n1.west) -- (n2.south);
        \path[draw=orange,thick,->] --  (s1.west) -- (s3.north);
        \path[draw=orange,thick,->] --  (s2.south) -- (s4.south);
    \end{tikzpicture} \\
    
    \vfill
    3 is the GCD of 24 and 9.
\end{frame}

\begin{frame}{Euclid's Algorithm (Cont'd)}
    \begin{itemize}
        \item Do we need to know the theory that Euclid used to come up with this algorithm in order to use it?
        \item What do we need in order to find the GCD using this algorithm?
    \end{itemize}
\end{frame}

\begin{frame}{The Idea Behind Algorithms}
    Once an algorithm behind a task has been discovered, we can just use it.
    \begin{itemize}
        \item We don't need to understand the principles.
        \item The task is reduced to following the instructions.
        \item The intelligence needed is ``encoded in the algorithm.''
    \end{itemize}
\end{frame}

\subsection{Syntax vs. Semantics}
\begin{frame}{Algorithm Representation}
    Syntax and Semantics \\ ~~ \\
    \begin{itemize}
        \item \textbf{Syntax} refers to the representation itself.
        \item \textbf{Semantics} refers to the concept represented.
    \end{itemize}
\end{frame}

\begin{frame}{Contrasting Syntax and Semantics}
    \begin{itemize}
        \item Human languages have both syntax and semantics.
        \item Syntax is the grammar of the language.
        \item Semantics is the meaning.
        \item Think about this sentence: \\ I walked to the corner grocery store.
        \begin{itemize}
            \item Is this sentence syntactically correct?
            \item Is it semantically correct?
        \end{itemize}
        \pause \item How about: \\ I talked to the funny grocery store.
        \pause \\ I grocery store walker corner to.
    \end{itemize}
\end{frame}

\begin{frame}{Semantics (Cont'd)}
    \begin{itemize}
        \item \underline{Conclusion}: A sentence may be syntactically correct, yet semantically incorrect.
        \item This is also true of algorithms.
        \item And this is also true of computer code (often called a bug).
    \end{itemize}
\end{frame}

\section{Part 2: Algorithm Construction}
\subsection{Problem Solving Examples}
\begin{frame}{Problem Solving}
    \begin{itemize}
        \item Decode this sentence: Pdeo eo pda yknnayp wjosan.
        \item We have just come up with a specific solution to a problem.
        \item Can this solution be generalized?
        \pause
        \item Now that we know what algorithms are, we are going to try some problem solving and write algorithms for the problems.
        \item We’ll start with step-by-step instructions that solve a particular problem and then write a generic algorithm that will solve any problem of that type.
    \end{itemize}
\end{frame}

\begin{frame}{Someone Stole a Cookie from the Cookie Jar}
    \underline{Problem}: Momma had just filled the cookie jar when the 3 children went to bed.  That night one child woke up, ate half of the cookies and went back to bed.  Later, the second child woke up, ate half of the remaining cookies, and went back to bed.  Still later, the third child woke up, ate half of the remaining cookies, leaving 3 cookies in the jar.  How many cookies were in the jar to begin with?
\end{frame}

\begin{frame}{Specific Solution to the Cookie Problem}
    First, we solve the specific problem to help us identify the steps.
    \begin{itemize}
        \item 3 cookies left $X$ 2 $=$ 6 cookies left before 3$^{rd}$ child
        \item 6 $X$ 2 $=$ 12 cookies left before $2^{nd}$ child
        \item 12 $X$ 2 $=$ 24 $=$ cookies left before $1^{st}$ child \\
        ~~ ~~ ~~ ~~ ~~ ~~ = original number of cookies
    \end{itemize}
\end{frame}

\begin{frame}{A Generic Algorithm}
    \only<1> {
        What is a \textbf{generic algorithm} for this problem? \\
        An algorithm that will work with \underline{any number of remaining cookies}
        \begin{center}
    				        AND
    	\end{center}
    	that will work with \underline{any number of children}.
	}
	\only<2> {
	    What is a \textbf{generic algorithm} for a given problem? \\ ~~ \\
	    
	    An algorithm for solving the problem, that has been generalized to work with a variety of possible input parameters to produce input-specific solutions.
	}
\end{frame}

\begin{frame}{Generic Algorithm for Cookie Problem}
    \begin{itemize}
        \item Get number of children.
        \item Get number of cookies remaining.
        \item While there are still children that have not raided the cookie jar, multiply the number of cookies by 2 and reduce the number of children by 1.
        \item Display the original number of cookies.
    \end{itemize}
\end{frame}

\begin{frame}{Flowchart for Cookie Problem}
    \centering
    % Graphic for TeX using PGF
% Title: /home/rjzak/Desktop/l5_algo_flowchart.dia
% Creator: Dia v0.97+git
% CreationDate: Mon Sep  6 22:21:17 2021
% For: rjzak
% \usepackage{tikz}
% The following commands are not supported in PSTricks at present
% We define them conditionally, so when they are implemented,
% this pgf file will use them.
\ifx\du\undefined
  \newlength{\du}
\fi
\setlength{\du}{15\unitlength}
\begin{tikzpicture}[even odd rule]
\pgftransformxscale{1.000000}
\pgftransformyscale{-1.000000}
\definecolor{dialinecolor}{rgb}{0.000000, 0.000000, 0.000000}
\pgfsetstrokecolor{dialinecolor}
\pgfsetstrokeopacity{1.000000}
\definecolor{diafillcolor}{rgb}{1.000000, 1.000000, 1.000000}
\pgfsetfillcolor{diafillcolor}
\pgfsetfillopacity{1.000000}
\pgfsetlinewidth{0.100000\du}
\pgfsetdash{}{0pt}
\pgfsetmiterjoin
\definecolor{diafillcolor}{rgb}{1.000000, 1.000000, 1.000000}
\pgfsetfillcolor{diafillcolor}
\pgfsetfillopacity{1.000000}
\pgfpathellipse{\pgfpoint{7.021181\du}{1.935590\du}}{\pgfpoint{1.478819\du}{0\du}}{\pgfpoint{0\du}{0.764410\du}}
\pgfusepath{fill}
\definecolor{dialinecolor}{rgb}{0.000000, 0.000000, 0.000000}
\pgfsetstrokecolor{dialinecolor}
\pgfsetstrokeopacity{1.000000}
\pgfpathellipse{\pgfpoint{7.021181\du}{1.935590\du}}{\pgfpoint{1.478819\du}{0\du}}{\pgfpoint{0\du}{0.764410\du}}
\pgfusepath{stroke}
% setfont left to latex
\definecolor{dialinecolor}{rgb}{0.000000, 0.000000, 0.000000}
\pgfsetstrokecolor{dialinecolor}
\pgfsetstrokeopacity{1.000000}
\definecolor{diafillcolor}{rgb}{0.000000, 0.000000, 0.000000}
\pgfsetfillcolor{diafillcolor}
\pgfsetfillopacity{1.000000}
\node[anchor=base,inner sep=0pt, outer sep=0pt,color=dialinecolor] at (7.021181\du,2.106561\du){Start};
\pgfsetlinewidth{0.100000\du}
\pgfsetdash{}{0pt}
\pgfsetmiterjoin
\definecolor{diafillcolor}{rgb}{1.000000, 1.000000, 1.000000}
\pgfsetfillcolor{diafillcolor}
\pgfsetfillopacity{1.000000}
\fill (3.335404\du,3.500000\du)--(10.581881\du,3.500000\du)--(10.126918\du,4.750000\du)--(2.880442\du,4.750000\du)--cycle;
\definecolor{dialinecolor}{rgb}{0.000000, 0.000000, 0.000000}
\pgfsetstrokecolor{dialinecolor}
\pgfsetstrokeopacity{1.000000}
\draw (3.335404\du,3.500000\du)--(10.581881\du,3.500000\du)--(10.126918\du,4.750000\du)--(2.880442\du,4.750000\du)--cycle;
% setfont left to latex
\definecolor{dialinecolor}{rgb}{0.000000, 0.000000, 0.000000}
\pgfsetstrokecolor{dialinecolor}
\pgfsetstrokeopacity{1.000000}
\definecolor{diafillcolor}{rgb}{0.000000, 0.000000, 0.000000}
\pgfsetfillcolor{diafillcolor}
\pgfsetfillopacity{1.000000}
\node[anchor=base,inner sep=0pt, outer sep=0pt,color=dialinecolor] at (6.731161\du,4.294260\du){Get number of children};
\pgfsetlinewidth{0.100000\du}
\pgfsetdash{}{0pt}
\pgfsetmiterjoin
\definecolor{diafillcolor}{rgb}{1.000000, 1.000000, 1.000000}
\pgfsetfillcolor{diafillcolor}
\pgfsetfillopacity{1.000000}
\fill (2.850809\du,5.650000\du)--(10.150000\du,5.650000\du)--(9.749633\du,6.750000\du)--(2.450442\du,6.750000\du)--cycle;
\definecolor{dialinecolor}{rgb}{0.000000, 0.000000, 0.000000}
\pgfsetstrokecolor{dialinecolor}
\pgfsetstrokeopacity{1.000000}
\draw (2.850809\du,5.650000\du)--(10.150000\du,5.650000\du)--(9.749633\du,6.750000\du)--(2.450442\du,6.750000\du)--cycle;
% setfont left to latex
\definecolor{dialinecolor}{rgb}{0.000000, 0.000000, 0.000000}
\pgfsetstrokecolor{dialinecolor}
\pgfsetstrokeopacity{1.000000}
\definecolor{diafillcolor}{rgb}{0.000000, 0.000000, 0.000000}
\pgfsetfillcolor{diafillcolor}
\pgfsetfillopacity{1.000000}
\node[anchor=base,inner sep=0pt, outer sep=0pt,color=dialinecolor] at (6.300221\du,6.371827\du){Get cookies remaining};
\pgfsetlinewidth{0.100000\du}
\pgfsetdash{}{0pt}
\pgfsetmiterjoin
\definecolor{diafillcolor}{rgb}{1.000000, 1.000000, 1.000000}
\pgfsetfillcolor{diafillcolor}
\pgfsetfillopacity{1.000000}
\fill (6.062029\du,8.228356\du)--(9.532219\du,9.637055\du)--(6.062029\du,11.045753\du)--(2.591840\du,9.637055\du)--cycle;
\definecolor{dialinecolor}{rgb}{0.000000, 0.000000, 0.000000}
\pgfsetstrokecolor{dialinecolor}
\pgfsetstrokeopacity{1.000000}
\draw (6.062029\du,8.228356\du)--(9.532219\du,9.637055\du)--(6.062029\du,11.045753\du)--(2.591840\du,9.637055\du)--cycle;
% setfont left to latex
\definecolor{dialinecolor}{rgb}{0.000000, 0.000000, 0.000000}
\pgfsetstrokecolor{dialinecolor}
\pgfsetstrokeopacity{1.000000}
\definecolor{diafillcolor}{rgb}{0.000000, 0.000000, 0.000000}
\pgfsetfillcolor{diafillcolor}
\pgfsetfillopacity{1.000000}
\node[anchor=base,inner sep=0pt, outer sep=0pt,color=dialinecolor] at (6.062029\du,9.808882\du){Still children?};
\pgfsetlinewidth{0.100000\du}
\pgfsetdash{}{0pt}
\pgfsetmiterjoin
\definecolor{diafillcolor}{rgb}{1.000000, 1.000000, 1.000000}
\pgfsetfillcolor{diafillcolor}
\pgfsetfillopacity{1.000000}
\fill (1.936505\du,11.900000\du)--(10.000000\du,11.900000\du)--(9.560186\du,13.108378\du)--(1.496692\du,13.108378\du)--cycle;
\definecolor{dialinecolor}{rgb}{0.000000, 0.000000, 0.000000}
\pgfsetstrokecolor{dialinecolor}
\pgfsetstrokeopacity{1.000000}
\draw (1.936505\du,11.900000\du)--(10.000000\du,11.900000\du)--(9.560186\du,13.108378\du)--(1.496692\du,13.108378\du)--cycle;
% setfont left to latex
\definecolor{dialinecolor}{rgb}{0.000000, 0.000000, 0.000000}
\pgfsetstrokecolor{dialinecolor}
\pgfsetstrokeopacity{1.000000}
\definecolor{diafillcolor}{rgb}{0.000000, 0.000000, 0.000000}
\pgfsetfillcolor{diafillcolor}
\pgfsetfillopacity{1.000000}
\node[anchor=base,inner sep=0pt, outer sep=0pt,color=dialinecolor] at (5.748346\du,12.676016\du){Display original cookies};
\pgfsetlinewidth{0.100000\du}
\pgfsetdash{}{0pt}
\pgfsetmiterjoin
\definecolor{diafillcolor}{rgb}{1.000000, 1.000000, 1.000000}
\pgfsetfillcolor{diafillcolor}
\pgfsetfillopacity{1.000000}
\pgfpathellipse{\pgfpoint{5.592705\du}{14.900000\du}}{\pgfpoint{1.307295\du}{0\du}}{\pgfpoint{0\du}{0.650000\du}}
\pgfusepath{fill}
\definecolor{dialinecolor}{rgb}{0.000000, 0.000000, 0.000000}
\pgfsetstrokecolor{dialinecolor}
\pgfsetstrokeopacity{1.000000}
\pgfpathellipse{\pgfpoint{5.592705\du}{14.900000\du}}{\pgfpoint{1.307295\du}{0\du}}{\pgfpoint{0\du}{0.650000\du}}
\pgfusepath{stroke}
% setfont left to latex
\definecolor{dialinecolor}{rgb}{0.000000, 0.000000, 0.000000}
\pgfsetstrokecolor{dialinecolor}
\pgfsetstrokeopacity{1.000000}
\definecolor{diafillcolor}{rgb}{0.000000, 0.000000, 0.000000}
\pgfsetfillcolor{diafillcolor}
\pgfsetfillopacity{1.000000}
\node[anchor=base,inner sep=0pt, outer sep=0pt,color=dialinecolor] at (5.592705\du,15.071827\du){End};
\pgfsetlinewidth{0.100000\du}
\pgfsetdash{}{0pt}
\pgfsetmiterjoin
{\pgfsetcornersarced{\pgfpoint{0.000000\du}{0.000000\du}}\definecolor{diafillcolor}{rgb}{1.000000, 1.000000, 1.000000}
\pgfsetfillcolor{diafillcolor}
\pgfsetfillopacity{1.000000}
\fill (11.472500\du,8.991622\du)--(11.472500\du,10.200000\du)--(15.300000\du,10.200000\du)--(15.300000\du,8.991622\du)--cycle;
}{\pgfsetcornersarced{\pgfpoint{0.000000\du}{0.000000\du}}\definecolor{dialinecolor}{rgb}{0.000000, 0.000000, 0.000000}
\pgfsetstrokecolor{dialinecolor}
\pgfsetstrokeopacity{1.000000}
\draw (11.472500\du,8.991622\du)--(11.472500\du,10.200000\du)--(15.300000\du,10.200000\du)--(15.300000\du,8.991622\du)--cycle;
}% setfont left to latex
\definecolor{dialinecolor}{rgb}{0.000000, 0.000000, 0.000000}
\pgfsetstrokecolor{dialinecolor}
\pgfsetstrokeopacity{1.000000}
\definecolor{diafillcolor}{rgb}{0.000000, 0.000000, 0.000000}
\pgfsetfillcolor{diafillcolor}
\pgfsetfillopacity{1.000000}
\node[anchor=base,inner sep=0pt, outer sep=0pt,color=dialinecolor] at (13.386250\du,9.767638\du){Cookies X 2};
\pgfsetlinewidth{0.100000\du}
\pgfsetdash{}{0pt}
\pgfsetmiterjoin
{\pgfsetcornersarced{\pgfpoint{0.000000\du}{0.000000\du}}\definecolor{diafillcolor}{rgb}{1.000000, 1.000000, 1.000000}
\pgfsetfillcolor{diafillcolor}
\pgfsetfillopacity{1.000000}
\fill (10.988750\du,11.250000\du)--(10.988750\du,12.458378\du)--(15.900000\du,12.458378\du)--(15.900000\du,11.250000\du)--cycle;
}{\pgfsetcornersarced{\pgfpoint{0.000000\du}{0.000000\du}}\definecolor{dialinecolor}{rgb}{0.000000, 0.000000, 0.000000}
\pgfsetstrokecolor{dialinecolor}
\pgfsetstrokeopacity{1.000000}
\draw (10.988750\du,11.250000\du)--(10.988750\du,12.458378\du)--(15.900000\du,12.458378\du)--(15.900000\du,11.250000\du)--cycle;
}% setfont left to latex
\definecolor{dialinecolor}{rgb}{0.000000, 0.000000, 0.000000}
\pgfsetstrokecolor{dialinecolor}
\pgfsetstrokeopacity{1.000000}
\definecolor{diafillcolor}{rgb}{0.000000, 0.000000, 0.000000}
\pgfsetfillcolor{diafillcolor}
\pgfsetfillopacity{1.000000}
\node[anchor=base,inner sep=0pt, outer sep=0pt,color=dialinecolor] at (13.444375\du,12.026016\du){Remove 1 child};
\pgfsetlinewidth{0.100000\du}
\pgfsetdash{}{0pt}
\pgfsetbuttcap
{
\definecolor{diafillcolor}{rgb}{0.000000, 0.000000, 0.000000}
\pgfsetfillcolor{diafillcolor}
\pgfsetfillopacity{1.000000}
% was here!!!
\pgfsetarrowsend{stealth}
\definecolor{dialinecolor}{rgb}{0.000000, 0.000000, 0.000000}
\pgfsetstrokecolor{dialinecolor}
\pgfsetstrokeopacity{1.000000}
\draw (7.021181\du,2.700000\du)--(6.958643\du,3.500000\du);
}
\pgfsetlinewidth{0.100000\du}
\pgfsetdash{}{0pt}
\pgfsetbuttcap
{
\definecolor{diafillcolor}{rgb}{0.000000, 0.000000, 0.000000}
\pgfsetfillcolor{diafillcolor}
\pgfsetfillopacity{1.000000}
% was here!!!
\pgfsetarrowsend{stealth}
\definecolor{dialinecolor}{rgb}{0.000000, 0.000000, 0.000000}
\pgfsetstrokecolor{dialinecolor}
\pgfsetstrokeopacity{1.000000}
\draw (6.503680\du,4.750000\du)--(6.500404\du,5.650000\du);
}
\pgfsetlinewidth{0.100000\du}
\pgfsetdash{}{0pt}
\pgfsetbuttcap
{
\definecolor{diafillcolor}{rgb}{0.000000, 0.000000, 0.000000}
\pgfsetfillcolor{diafillcolor}
\pgfsetfillopacity{1.000000}
% was here!!!
\pgfsetarrowsend{stealth}
\definecolor{dialinecolor}{rgb}{0.000000, 0.000000, 0.000000}
\pgfsetstrokecolor{dialinecolor}
\pgfsetstrokeopacity{1.000000}
\draw (6.100037\du,6.750000\du)--(6.062029\du,8.228356\du);
}
\pgfsetlinewidth{0.100000\du}
\pgfsetdash{}{0pt}
\pgfsetbuttcap
{
\definecolor{diafillcolor}{rgb}{0.000000, 0.000000, 0.000000}
\pgfsetfillcolor{diafillcolor}
\pgfsetfillopacity{1.000000}
% was here!!!
\pgfsetarrowsend{stealth}
\definecolor{dialinecolor}{rgb}{0.000000, 0.000000, 0.000000}
\pgfsetstrokecolor{dialinecolor}
\pgfsetstrokeopacity{1.000000}
\draw (6.062029\du,11.045753\du)--(5.968253\du,11.900000\du);
}
\pgfsetlinewidth{0.100000\du}
\pgfsetdash{}{0pt}
\pgfsetbuttcap
{
\definecolor{diafillcolor}{rgb}{0.000000, 0.000000, 0.000000}
\pgfsetfillcolor{diafillcolor}
\pgfsetfillopacity{1.000000}
% was here!!!
\pgfsetarrowsend{stealth}
\definecolor{dialinecolor}{rgb}{0.000000, 0.000000, 0.000000}
\pgfsetstrokecolor{dialinecolor}
\pgfsetstrokeopacity{1.000000}
\draw (5.528439\du,13.108378\du)--(5.592705\du,14.250000\du);
}
\pgfsetlinewidth{0.100000\du}
\pgfsetdash{}{0pt}
\pgfsetbuttcap
{
\definecolor{diafillcolor}{rgb}{0.000000, 0.000000, 0.000000}
\pgfsetfillcolor{diafillcolor}
\pgfsetfillopacity{1.000000}
% was here!!!
\pgfsetarrowsend{stealth}
\definecolor{dialinecolor}{rgb}{0.000000, 0.000000, 0.000000}
\pgfsetstrokecolor{dialinecolor}
\pgfsetstrokeopacity{1.000000}
\draw (9.532219\du,9.637055\du)--(11.472500\du,9.595811\du);
}
\pgfsetlinewidth{0.100000\du}
\pgfsetdash{}{0pt}
\pgfsetbuttcap
{
\definecolor{diafillcolor}{rgb}{0.000000, 0.000000, 0.000000}
\pgfsetfillcolor{diafillcolor}
\pgfsetfillopacity{1.000000}
% was here!!!
\pgfsetarrowsend{stealth}
\definecolor{dialinecolor}{rgb}{0.000000, 0.000000, 0.000000}
\pgfsetstrokecolor{dialinecolor}
\pgfsetstrokeopacity{1.000000}
\draw (13.386250\du,10.200000\du)--(13.444375\du,11.250000\du);
}
\pgfsetlinewidth{0.100000\du}
\pgfsetdash{}{0pt}
\pgfsetmiterjoin
\pgfsetbuttcap
{
\definecolor{diafillcolor}{rgb}{0.000000, 0.000000, 0.000000}
\pgfsetfillcolor{diafillcolor}
\pgfsetfillopacity{1.000000}
% was here!!!
\pgfsetarrowsend{stealth}
{\pgfsetcornersarced{\pgfpoint{0.000000\du}{0.000000\du}}\definecolor{dialinecolor}{rgb}{0.000000, 0.000000, 0.000000}
\pgfsetstrokecolor{dialinecolor}
\pgfsetstrokeopacity{1.000000}
\draw (13.444375\du,12.458378\du)--(13.444375\du,12.950000\du)--(16.400000\du,12.950000\du)--(16.400000\du,7.489178\du)--(6.081033\du,7.489178\du);
}}
% setfont left to latex
\definecolor{dialinecolor}{rgb}{0.000000, 0.000000, 0.000000}
\pgfsetstrokecolor{dialinecolor}
\pgfsetstrokeopacity{1.000000}
\definecolor{diafillcolor}{rgb}{0.000000, 0.000000, 0.000000}
\pgfsetfillcolor{diafillcolor}
\pgfsetfillopacity{1.000000}
\node[anchor=base west,inner sep=0pt,outer sep=0pt,color=dialinecolor] at (6.250000\du,11.650000\du){No};
% setfont left to latex
\definecolor{dialinecolor}{rgb}{0.000000, 0.000000, 0.000000}
\pgfsetstrokecolor{dialinecolor}
\pgfsetstrokeopacity{1.000000}
\definecolor{diafillcolor}{rgb}{0.000000, 0.000000, 0.000000}
\pgfsetfillcolor{diafillcolor}
\pgfsetfillopacity{1.000000}
\node[anchor=base west,inner sep=0pt,outer sep=0pt,color=dialinecolor] at (9.300000\du,9.500000\du){Yes};
\end{tikzpicture}

\end{frame}

\subsection{Pseudocode}
\begin{frame}{Pseudocode}
    \only<1> {
        \begin{itemize}
            \item When we broke down the previous problem into steps, we expressed each step as an English phrase.
            \item We can think of this as writing \textbf{pseudocode} for the problem.
            \item Typically, pseudocode is a combination of phrases and formulas.
        \end{itemize}
    }
    \only<2> {
        \begin{itemize}
            \item Pseudocode is used in
            \begin{itemize}
                \item designing algorithms
                \item communicating an algorithm to the customer
                \item converting an algorithm to code (used by the programmer)
                \item \textbf{debugging} logic (semantic) errors in a solution before coding (\textbf{hand tracing})
            \end{itemize}
            \item Let’s write the Cookie Problem algorithm using a more formal pseudocode and being more precise.
        \end{itemize}
    }
\end{frame}

\begin{frame}[fragile]\frametitle{Improved Pseudocode}\label{slide:cookieloop}
    \begin{verbatim}
Display "Enter the number of children:  "
Read <number of children>
Display "Enter the number of cookies remaining:  "
Read <cookies remaining>
<original cookies> = <cookies remaining>
While (<number of children>  >  0)
    <original cookies> = <original cookies> X 2
    <number of children> = <number of children> - 1
End_While
Display "Original number of cookies = ", <original cookies>
    \end{verbatim}
\end{frame}

\begin{frame}{Observations}
    \only<1> {
        \begin{itemize}
            \item Any \textbf{user prompts} should appear exactly as you wish the programmer to code them.
            \item The destination of any output data should be stated, such as in ``Display'', which implies the screen.
            \item Make the data items clear (e.g., surround them by $<$  and  $>$ ) and give them descriptive names.
            \item Use formulas wherever possible for clarity and brevity.
            \item Use keywords (such as Read and While) and use them consistently.  Accent them in some manner.
        \end{itemize}
    }
    \only<2> {
        \begin{itemize}
            \item Use indentation for clarity of logic.
            \item Avoid using code.  Pseudocode should not be programming language-specific.
            \item Always keep in mind that you may not be the person translating your pseudocode into programming language code.  It must, therefore, be unambiguous.
            \item You may make up your own pseudocode guidelines, but you MUST be consistent.
        \end{itemize}
    }
\end{frame}

\begin{frame}{Brian’s Shopping Trip}
    \underline{Problem}:  Brian bought a belt for \$9 and a shirt that cost 4 times as much as the belt.  He then had \$10.  How much money did Brian have before he bought the belt and shirt?
\end{frame}

\begin{frame}{Specific Solution to Shopping Problem}
    \centering
    \begin{tabular}{r l}
        $Start\$$ & $  =  Belt\$ + Shirt\$ + \$10$ \\
        $Start\$$ & $  =  Belt\$ + (4 X Belt\$) + \$10$ \\
	    $Start\$$ & $  =  9  +  (4 X 9)  +  10  =  \$55$
    \end{tabular}
\end{frame}

\begin{frame}{Generic Algorithm for Shopping Problem}
    \begin{itemize}
        \item Now, let’s write a generic algorithm to solve any problem of this type.
        \item What are the inputs to the algorithm?
        \begin{itemize}
            \item the cost of the first item (doesn’t matter that it’s a belt):  $<$item1 price$>$
            \item the number to multiply the cost of the first item by to get the cost of the second item:  $<$multiplier$>$
            \item the amount of money left at the end of shopping: $<$amount left$>$
        \end{itemize}
        \pause
        \item What are the outputs from the algorithm?
        \begin{itemize}
            \item the amount of money available at the start of the shopping trip: $<$start amount$>$
        \end{itemize}
        \item Note that we may end up needing some intermediate variables.
    \end{itemize}
\end{frame}

\begin{frame}[fragile]\frametitle{Pseudocode}
    \begin{verbatim}
Display "Enter the price of the first item:  "
Read <item 1 price>
Display "Enter the multiplier:  "
Read <multiplier>
Display "Enter the amount left after shopping:  "
Read <amount left>
<item2 price> = <multiplier> X <item1 price>
<start amount> = <item1 price> + <item2 price> +
                            <amount left>
Display "The starting amount was ", <start amount> 
    \end{verbatim}
\end{frame}

\subsection{Control Structures}
\begin{frame}{Control Structures}
    Any problem can be solved using only three logical \textbf{control structures}:
    \begin{itemize}
        \item Sequence
        \item Selection
        \item Repetition
    \end{itemize}
\end{frame}

\begin{frame}[fragile]\frametitle{Sequence}
    \begin{itemize}
        \item A series of steps or statements that are executed in the order they are written.
        \item Example:
        \begin{verbatim}
  Display "Enter two numbers:  "
  Read <number1>
  Read <number2>
  <sum> = <number1> + <number2>
  Display "sum = ", <sum>
        \end{verbatim}
    \end{itemize}
\end{frame}

\begin{frame}[fragile]\frametitle{Selection}
    \begin{itemize}
        \item Defines one or more courses of action depending on the evaluation of a condition.
        \item Synonyms: \textbf{conditional}, \textbf{branching}, \textbf{decision}.
        \item Examples:
    \end{itemize}
    \begin{columns}
        \column{0.4\textwidth}
            \begin{verbatim}
if (condition is true)
    do something
end_if


            \end{verbatim}
        \column{0.5\textwidth}
            \begin{verbatim}
if (condition is true)
    do this
else
    do that
end_if
            \end{verbatim}
    \end{columns}
\end{frame}

\begin{frame}[fragile]\frametitle{Selection}
    A more complex example:
    \begin{verbatim}
DISPLAY "Enter number to invert"
READ <my_num>
If (<my_num> < 0)
    DISPLAY "Don’t like negative numbers"
    if (<my_num> < -999)
        DISPLAY "... but I guess you really do!"
    End_if
Else_if (<my_num>  == 0)
    <result> = 0
Else
    <result> = 1 / <my_num>
End_if
    \end{verbatim}
\end{frame}

\begin{frame}[fragile]\frametitle{Repetition}
    \begin{itemize}
        \item Allows one or more statements to be repeated as long as a given condition is true.
        \item Synonyms:  looping, iteration
        \item Example:
        \begin{verbatim}
    While (condition is true)
  		     do this
  		End_while
        \end{verbatim}
        \item Notice the repetition structure in the Cookie Problem pseudocode on Slide \ref{slide:cookieloop}.
    \end{itemize}
\end{frame}

\begin{frame}[fragile]\frametitle{Repetition}
    More complex example (with mistakes)
    \begin{verbatim}
    DISPLAY "Enter number to compute factorial for"
    READ <my_num>
    
    While (<my_num> > 0)
        <factorial> = <factorial> * <my_num>
        <my_num> = <my_num> - 1
    End_while
    
    DISPLAY "The factorial of", <my_num>, " is ", <factorial>
    \end{verbatim}
\end{frame}

\begin{frame}[fragile]\frametitle{Repetition}
    More complex example (with mistakes)
    \begin{verbatim}
    DISPLAY "Enter number to compute factorial for"
    READ <my_num>
    <factorial> = 1
    
    While (<my_num> > 0)
        <factorial> = <factorial> * <my_num>
        <my_num> = <my_num> - 1
    End_while
    
    DISPLAY "The factorial of", <my_num>, " is ", <factorial>
    \end{verbatim}
\end{frame}

\begin{frame}[fragile]\frametitle{Repetition}
    More complex example (with mistakes)
    \begin{verbatim}
    DISPLAY "Enter number to compute factorial for"
    READ <my_num>
    <factorial> = 1
    <saved_my_num> = <my_num>
    
    While (<my_num> > 0)
        <factorial> = <factorial> * <my_num>
        <my_num> = <my_num> - 1
    End_while
    
    <my_num> = <saved_my_num>
    DISPLAY "The factorial of", <my_num>, " is ", <factorial>
    \end{verbatim}
\end{frame}

\begin{frame}{Writing Algorithms from Scratch}
    \begin{itemize}
        \item Given a problem statement, we are going to write the corresponding generic algorithm for the solution.
        \item We will use the following procedure:
        \begin{itemize}
            \item Determine the algorithm inputs and outputs
            \item Complete the pseudocode
        \end{itemize}
    \end{itemize}
\end{frame}

\section{Algorithm Exercises}
\begin{frame}{Warm-up: Simple Tip Calculator}
    \underline{Problem}:  Write an interactive program to calculate the dollar amount of tip on a restaurant bill at the standard 15\% rate.
    \\ ~~ \\
	You should allow for changes in the total price of the bill. \\ ~~ \\
	\underline{Tip:} Whenever getting user input, always check for invalid input. Sometimes bad input is accidental, sometimes it's malicious behaviour.
\end{frame}

\begin{frame}{Drawing a Rectangle}
    \underline{Problem}:  Write an interactive program that will draw a solid rectangle of asterisks (*) of user-specified dimensions.  The program must also display the dimensions of the rectangle at the end. 
\end{frame}

\begin{frame}{General Tip Calculator}
    \underline{Problem}:  Write an interactive program to calculate a table of dollar amounts of tip on a restaurant bill.  You should allow for changes in the total price of the bill. You should also ask the user for the range of tipping rates to calculate (ex: high or low amount, or 10\%, 15\%, or custom \%).
    \\ ~~ \\
    Error checking should be done to be sure that the amount of the bill is greater than 0, that the tip amount is between zero and one.
\end{frame}
%
%\begin{frame}{Group Exercise}
%    \begin{columns}
%        \column{0.5\textwidth}
%            \textbf{Drawing a Rectangle:} Write an interactive program that will draw a solid rectangle of asterisks (*) of user-specified dimensions.  The program must also display the dimensions of the rectangle.  Error checking must be done to be sure that the dimensions are greater than zero.
%            \\ ~~
%        \column{0.5\textwidth}
%            \textbf{Tip Calculator:} Write an interactive program to calculate a table of dollar amounts of tip on a restaurant bill.  You should allow for changes in the total price of the bill. You should also ask the user for the range of tipping rates to calculate (i.e., low and high ends).  Error checking should be done to be sure that the amount of the bill is greater than 0.
%    \end{columns}
%    ~~ \\ ~~ \\
%    \underline{Tip:} Whenever getting user input, always check for invalid input. Sometimes bad input is accidental, sometimes it's malicious behaviour.
%\end{frame}

\end{document}