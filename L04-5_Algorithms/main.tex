\documentclass[graphics]{beamer}
\usepackage{graphicx}
\usepackage{listings} % Syntax highlighing
\usepackage{fancyvrb} % Inline verbatim
\usepackage{hyperref} % Hyperlinks
\hypersetup{pdfpagemode=FullScreen}

\usepackage[normalem]{ulem}               % to striketrhourhg text
\newcommand\redout{\bgroup\markoverwith
{\textcolor{red}{\rule[0.5ex]{2pt}{0.8pt}}}\ULon}

% header in tables
\newcommand*{\thead}[1]{\multicolumn{1}{c}{\bfseries #1}}

% used for arrows from one point in the slide to another
\usepackage{tikz}
\usetikzlibrary{arrows,shapes,tikzmark}

\usetheme{Boadilla}
\title{Lecture 4-5: Algorithms}
\author{UMBC CMSC 104}
\date{Tu 09 Sept 2021}

\begin{document}

\begin{frame}{}
\centering
    Algorithms
\end{frame}

\section*{Part 1}
\begin{frame}{Topics}
    \begin{itemize}
        \item Definition of an Algorithm
        \item Algorithm Examples
        \item Syntax versus Semantics
    \end{itemize}
\end{frame}

\begin{frame}{Problem Solving}
    \begin{itemize}
        \item Problem solving is the process of transforming the description of a problem into the solution of that problem.
        \item We use our knowledge of the problem domain (requirements).
        \item We rely on our ability to select and use appropriate problem-solving strategies, techniques, and tools.
    \end{itemize}
\end{frame}

\begin{frame}{Algorithms}
    \begin{itemize}
        \item An \textbf{algorithm} is a step by step solution to a problem.
        \item Why bother writing an algorithm?
        \begin{itemize}
            \item For your own use in the future.  You won’t have to rethink the problem.
            \item So others can use it, even if they know very little about the principles behind how the solution was derived.
        \end{itemize}
    \end{itemize}
\end{frame}

\begin{frame}{Examples of Algorithms}
    \begin{itemize}
        \item Washing machine instructions
        \item How to make a peanut butter and jelly sandwich
        \item A classic: finding the greatest common divisor (GCD) using Euclid’s Algorithm
    \end{itemize}
\end{frame}

\begin{frame}{Washing Machine Instructions}
    \begin{itemize}
        \item Separate clothes into white clothes and colored clothes.
        \item Add 1 cup of powdered laundry detergent to tub.
        \item For white clothes:
        \begin{itemize}
            \item Set water temperature knob to HOT.
            \item Place white laundry in tub.
        \end{itemize}
        \item For colors:
        \begin{itemize}
            \item Set water temperature knob to COLD.
            \item Place colored laundry in tub.
        \end{itemize}
        \item Close lid and press the start button.
    \end{itemize}
\end{frame}

\begin{frame}{Observations About the Washing Machine Instructions}
    \begin{itemize}
        \item There are a finite number of steps.
        \item We are capable of doing each of the instructions.
        \item When we have followed all of the steps, the washing machine will wash the clothes and then will stop.
    \end{itemize}
\end{frame}

\begin{frame}{Refinement of our Algorithm Definition}
    \begin{itemize}
        \item Old definition:
        \begin{itemize}
            \item An algorithm is a step by step solution to a problem.
        \end{itemize}
        \item Adding to the definition based on our observations:
        \begin{itemize}
            \item An algorithm is a \underline{finite set} of \underline{executable instructions} which \underline{directs a terminating activity}.
        \end{itemize}
    \end{itemize}
\end{frame}

\begin{frame}{How to Make a Peanut Butter and Jelly Sandwich}
    \begin{itemize}
        \item Open a web browser
        \item Go to \url{https://www.google.com}
        \item Search for ``harvard how to make pbj sandwich''
        \item Click any video (or \url{https://youtu.be/okkIyWhN0iQ})
        \item Enjoy
    \end{itemize}
\end{frame}

\begin{frame}{Final Version of the Algorithm Definition}
    \begin{itemize}
        \item Our old definition:
        \begin{itemize}
            \item An algorithm is a finite set of executable instructions that directs a terminating activity.
        \end{itemize}
        \item Final version:
        \begin{itemize}
            \item An algorithm is a finite set of \underline{unambiguous}, executable instructions which directs a terminating activity.
        \end{itemize}
    \end{itemize}
\end{frame}

\begin{frame}{History of Algorithms}
    \begin{itemize}
        \item The study of algorithms began as a subject in mathematics.
        \item The search for algorithms was a significant activity of early mathematicians.
        \item Goal: To find a single set of instructions that can be used to solve any problem of a particular type (a \textbf{general solution}).
    \end{itemize}
\end{frame}

\begin{frame}{Euclid's Algorithm}
    \underline{Problem}: Find the largest positive integer which divides evenly into two given positive integers (the \textbf{greatest common divisor, GCD}). \\ ~~ \\
    \underline{Algorithm}:
    \begin{enumerate}
        \item Assign $M$ and $N$ the values of the larger and smaller of the two positive integers, respectively.
        \item Divide $M$ by $N$ and call the remainder $R$.
        \item If $R$ is not zero, then assign $M$ the value of $N$, assign $N$ the value of $R$, and return to Step 2. Otherwise, the greatest common divisor is the value currently assigned to $N$.
    \end{enumerate}
\end{frame}

\begin{frame}{Finding the GCD of 24 and 9}
    \centering
    \begin{tabular}{ | c | c | c |} \hline
        $M$ & $N$ & $R$ \\ \hline
        24  & \tikz[remember picture] \node[coordinate,anchor=west] (n1) {};9   & \tikz[remember picture] \node[coordinate,anchor=west] (s1) {};6   \\ \hline
        9\tikz[remember picture] \node[coordinate,anchor=north] (n2) {};  & \tikz[remember picture] \node[coordinate,anchor=east] (s2) {};6\tikz[remember picture] \node[coordinate,anchor=north] (s3) {};   & 3   \\ \hline
        6\tikz[remember picture] \node[coordinate,anchor=west] (s4) {};  & \color{red}3   & 0 \\ \hline
    \end{tabular}
    \begin{tikzpicture}[remember picture,overlay]
        \path[draw=orange,thick,->] --  (n1.west) -- (n2.south);
        \path[draw=orange,thick,->] --  (s1.west) -- (s3.north);
        \path[draw=orange,thick,->] --  (s2.south) -- (s4.south);
    \end{tikzpicture} \\
    
    \vfill
    3 is the GCD of 24 and 9.
\end{frame}

\begin{frame}{Euclid's Algorithm (Cont'd)}
    \begin{itemize}
        \item Do we need to know the theory that Euclid used to come up with this algorithm in order to use it?
        \item What do we need in order to find the GCD using this algorithm?
    \end{itemize}
\end{frame}

\begin{frame}{The Idea Behind Algorithms}
    Once an algorithm behind a task has been discovered, we can just use it.
    \begin{itemize}
        \item We don't need to understand the principles.
        \item The task is reduced to following the instructions.
        \item The intelligence needed is ``encoded in the algorithm.''
    \end{itemize}
\end{frame}

\begin{frame}{Algorithm Representation}
    Syntax and Semantics \\ ~~ \\
    \begin{itemize}
        \item \textbf{Syntax} refers to the representation itself.
        \item \textbf{Semantics} refers to the concept represented.
    \end{itemize}
\end{frame}

\begin{frame}{Contrasting Syntax and Semantics}
    \begin{itemize}
        \item Human languages have both syntax and semantics.
        \item Syntax is the grammar of the language.
        \item Semantics is the meaning.
        \item Think about this sentence: \\ I walked to the corner grocery store.
        \begin{itemize}
            \item Is this sentence syntactically correct?
            \item Is it semantically correct?
        \end{itemize}
        \pause \item How about: \\ I talked to the funny grocery store.
        \pause \\ I grocery store walker corner to.
    \end{itemize}
\end{frame}

\begin{frame}{Semantics (Cont'd)}
    \begin{itemize}
        \item \underline{Conclusion}: A sentence may be syntactically correct, yet semantically incorrect.
        \item This is also true of algorithms.
        \item And this is also true of computer code (often called a bug).
    \end{itemize}
\end{frame}

\section*{Part 2}
\begin{frame}{Algorithms, Part 2}
    Topics:
    \begin{itemize}
        \item Problem Solving Examples
        \item Pseudocode
        \item Control Structures
    \end{itemize}
\end{frame}

\end{document}