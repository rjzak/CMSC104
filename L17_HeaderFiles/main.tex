\documentclass[graphics]{beamer}
\usepackage{graphicx}
\usepackage{listings} % Syntax highlighing
\usepackage{fancyvrb} % Inline verbatim
\usepackage{hyperref} % Hyperlinks
\usepackage{attachfile} % attach files
\hypersetup{pdfpagemode=FullScreen}

\usepackage[normalem]{ulem}               % to striketrhourhg text
\newcommand\redout{\bgroup\markoverwith
{\textcolor{red}{\rule[0.5ex]{2pt}{0.8pt}}}\ULon}

% header in tables
\newcommand*{\thead}[1]{\multicolumn{1}{c}{\bfseries #1}}

% used for arrows from one point in the slide to another
\usepackage{tikz}
\usetikzlibrary{arrows, shapes, tikzmark, positioning}

\newcommand{\backupbegin}{
   \newcounter{framenumberappendix}
   \setcounter{framenumberappendix}{\value{framenumber}}
}
\newcommand{\backupend}{
   \addtocounter{framenumberappendix}{-\value{framenumber}}
   \addtocounter{framenumber}{\value{framenumberappendix}} 
}

\usetheme{Boadilla}
\title{Lecture 17: Header Files}
\author{UMBC CMSC 104}
\date{Tu 26 April 2022}

\begin{document}

\begin{frame}{}
\centering
    Header Files
\end{frame}

\frame{\tableofcontents}

\section{Creating Header Files}
\begin{frame}{Why Use .h Files?}
    \begin{itemize}
        \item Allows us to organize parts of our programs.
        \item Allows us to reuse functions we have written for earlier assignments, or different projects.
    \end{itemize}
\end{frame}

\subsection{Separate Compilation}
\begin{frame}[fragile]{.h Example}
    \begin{lstlisting}[language=C,basicstyle=\footnotesize,keywordstyle=\color{blue},commentstyle=\color{green},showstringspaces=false,stringstyle=\color{red}]
#ifndef EXAMPLE_H__
#define EXAMPLE_H__

int plusfive(int x);

#endif
    \end{lstlisting}
\end{frame}

\begin{frame}[fragile]{Corresponding .c Example}
    \begin{lstlisting}[language=C,basicstyle=\footnotesize,keywordstyle=\color{blue},commentstyle=\color{green},showstringspaces=false,stringstyle=\color{red}]
#include "example.h"

int plusfive(int x) {
    return x+5;
}
    \end{lstlisting}
\end{frame}

\begin{frame}[fragile]{Main}
    \begin{lstlisting}[language=C,basicstyle=\footnotesize,keywordstyle=\color{blue},commentstyle=\color{green},showstringspaces=false,stringstyle=\color{red}]
#include <stdio.h>
#include "example.h"

int main() {
    int y = plusfive(3);
    printf("y = %d\n", y);
    return 0;
}
    \end{lstlisting}
\end{frame}

\begin{frame}[fragile]{Compiling}
    \begin{verbatim}
$ gcc -Wall main.c example.c -o example
$ ./example
y = 8
$
    \end{verbatim}
\end{frame}

\begin{frame}{What Goes in the .h?}
    Typically, the .h file would contain:
    \begin{itemize}
        \item Function prototypes
        \item Constants
        \begin{itemize}
            \item \texttt{\#define}, such as \texttt{\#define PI = 3.1415}
            \item \texttt{const} values, such as \texttt{const float PI = 3.1415;}
        \end{itemize}
        \item Definitions for custom data types:
        \begin{itemize}
            \item Structs
            \item Enumerations
        \end{itemize}
    \end{itemize}
\end{frame}

\subsection{Header Guards}
\begin{frame}{Header Guards}
    \begin{itemize}
        \item Did you notice the \texttt{\#ifndef EXAMPLE\_H\_\_}?
        \item It prevents \texttt{gcc} from trying to use the contents of the header file twice.
        \begin{itemize}
            \item First time would be compiling \texttt{main.c}
            \item Second time would be compiling \texttt{example.c}
            \item The value for the \texttt{\#ifndef} and \texttt{\#define} is typically the file name in capital letters, period replaced by underscore, and sometimes an underscore or two before and/or after. The actual value isn't so important as long as the value is only used for this purpose.
        \end{itemize}
        \item The compiler isn't so smart, and it would complain about redefining \texttt{int plusfive()}, so we have to out-smart \texttt{gcc} by only letting it \textit{see} the function prototype once.
        \begin{itemize}
            \item The second time \texttt{gcc} would look at \texttt{example.h}, our variable \texttt{EXAMPLE\_H\_\_} would already be defined, so \texttt{gcc} would skip the contents of the file.
        \end{itemize}
    \end{itemize}
\end{frame}

\begin{frame}{Makefiles}
    \begin{itemize}
        \item This discussion on separate compilation should remind you of Makefiles, from earlier in the semester.
        \item For larger projects, using separate compilation and \texttt{make} can reduce the time it takes to compile your project by only compiling the pieces needed.
        \begin{itemize}
            \item \texttt{gcc -c main.c}
            \item \texttt{gcc -c example.c}
            \item \texttt{gcc main.o example.o -o example}
        \end{itemize}
    \end{itemize}
\end{frame}

\end{document}