\documentclass[graphics]{beamer}
\usepackage{graphicx}
\usepackage{listings} % Syntax highlighing
\usepackage{fancyvrb} % Inline verbatim
\usepackage{hyperref} % Hyperlinks
\usepackage{attachfile} % attach files
\hypersetup{pdfpagemode=FullScreen}

\usepackage[normalem]{ulem}               % to striketrhourhg text
\newcommand\redout{\bgroup\markoverwith
{\textcolor{red}{\rule[0.5ex]{2pt}{0.8pt}}}\ULon}

% header in tables
\newcommand*{\thead}[1]{\multicolumn{1}{c}{\bfseries #1}}

% used for arrows from one point in the slide to another
\usepackage{tikz}
\usetikzlibrary{arrows, shapes, tikzmark, positioning}

\newcommand{\backupbegin}{
   \newcounter{framenumberappendix}
   \setcounter{framenumberappendix}{\value{framenumber}}
}
\newcommand{\backupend}{
   \addtocounter{framenumberappendix}{-\value{framenumber}}
   \addtocounter{framenumber}{\value{framenumberappendix}} 
}

\usetheme{Boadilla}
\title{Lecture 17: Header Files}
\author{UMBC CMSC 104}
\date{Tu 25 Apr 2023}

\begin{document}

\begin{frame}{}
\centering
    Header Files
\end{frame}

\frame{\tableofcontents}

\section{Creating Header Files}
\begin{frame}{Why Use .h Files?}
    \begin{itemize}
        \item Allows us to organize parts of our programs.
        \item Allows us to reuse functions we have written for earlier assignments, or different projects.
    \end{itemize}
\end{frame}

\subsection{Separate Compilation}
\begin{frame}[fragile]{.h Example}
    \begin{lstlisting}[language=C,basicstyle=\footnotesize,keywordstyle=\color{blue},commentstyle=\color{green},showstringspaces=false,stringstyle=\color{red}]
#ifndef EXAMPLE_H__
#define EXAMPLE_H__

int plusfive(int x);

#endif
    \end{lstlisting}
\end{frame}

\begin{frame}[fragile]{Corresponding .c Example}
    \begin{lstlisting}[language=C,basicstyle=\footnotesize,keywordstyle=\color{blue},commentstyle=\color{green},showstringspaces=false,stringstyle=\color{red}]
#include "example.h"

int plusfive(int x) {
    return x+5;
}
    \end{lstlisting}
\end{frame}

\begin{frame}[fragile]{Main}
    \begin{lstlisting}[language=C,basicstyle=\footnotesize,keywordstyle=\color{blue},commentstyle=\color{green},showstringspaces=false,stringstyle=\color{red}]
#include <stdio.h>
#include "example.h"

int main() {
    int y = plusfive(3);
    printf("y = %d\n", y);
    return 0;
}
    \end{lstlisting}
\end{frame}

\begin{frame}[fragile]{Compiling}
    \begin{verbatim}
$ gcc -Wall main.c example.c -o example
$ ./example
y = 8
$
    \end{verbatim}
\end{frame}

\begin{frame}{What Goes in the .h?}
    Typically, the .h file would contain:
    \begin{itemize}
        \item Function prototypes
        \item Constants
        \begin{itemize}
            \item \texttt{\#define}, such as \texttt{\#define PI = 3.1415}
            \item \texttt{const} values, such as \texttt{const float PI = 3.1415;}
        \end{itemize}
        \item Definitions for custom data types:
        \begin{itemize}
            \item Structs
            \item Enumerations
        \end{itemize}
    \end{itemize}
\end{frame}

\subsection{Header Guards}
\begin{frame}{Header Guards}
    \begin{itemize}
        \item Did you notice the \texttt{\#ifndef EXAMPLE\_H\_\_}?
        \item It prevents \texttt{gcc} from trying to use the contents of the header file twice.
        \begin{itemize}
            \item First time would be compiling \texttt{main.c}
            \item Second time would be compiling \texttt{example.c}
            \item The value for the \texttt{\#ifndef} and \texttt{\#define} is typically the file name in capital letters, period replaced by underscore, and sometimes an underscore or two before and/or after. The actual value isn't so important as long as the value is only used for this purpose.
        \end{itemize}
        \item The compiler isn't so smart, and it would complain about redefining \texttt{int plusfive()}, so we have to out-smart \texttt{gcc} by only letting it \textit{see} the function prototype once.
        \begin{itemize}
            \item The second time \texttt{gcc} would look at \texttt{example.h}, our variable \texttt{EXAMPLE\_H\_\_} would already be defined, so \texttt{gcc} would skip the contents of the file.
        \end{itemize}
    \end{itemize}
\end{frame}

\subsection{Custom Types}
\begin{frame}{Custom Types}
    So far, we've been using integers, floating point numbers, characters, and strings. It's possible to have our own types.
    \begin{itemize}
        \item Enumerations
        \item Structures
    \end{itemize}
    
    Definitions for these would be in the header file.
\end{frame}

\begin{frame}[fragile]{Enumerations}
    Enumerations, or \texttt{enum}s, allow for specifying variables which have values that are custom defined. Internally, they are just integers.
    \begin{lstlisting}[language=C,basicstyle=\footnotesize,keywordstyle=\color{blue},commentstyle=\color{green},showstringspaces=false,stringstyle=\color{red}]
#include <stdio.h>

enum color {
    Red, Green,
    Blue, Purple,
    Orange, Yellow
};

void print_color(enum color the_color); // Enums as a parameter

int main() {
    enum color my_color = Red;
    printf("Red: %d\n"); // Prints "Red: 0"
    return 0;
}
    \end{lstlisting}
\end{frame}

\begin{frame}[fragile]{Structures}
    Structures, or \texttt{struct}s, allow for combining different values into one object. This is used to store related variables, or to model real-world things.
    
    \begin{lstlisting}[language=C,basicstyle=\footnotesize,keywordstyle=\color{blue},commentstyle=\color{green},showstringspaces=false,stringstyle=\color{red}]
#include <stdio.h>
enum color {Red, Green, Blue, Purple, Orange, Yellow};
struct car {
    char* make;
    char* model;
    enum color the_color;
    int miles;
}
void print_car(struct car the_car); // Struct as a parameter
int main() {
    struct car my_car;
    my_car.make = "Honda";
    my_car.model = "Civic";
    my_car.color = Blue;
    my_car.miles = 1234;
    print_car(my_car);
    return 0;
}
    \end{lstlisting}
\end{frame}

\section{Make}
\begin{frame}{Separate Compilation}
    Often, programs are too large for one .c file, in which case similar parts are put in to their own .c files. But the code still works together.
    \begin{itemize}
        \item \texttt{gcc -c my\_program.c}
        \begin{itemize}
            \item Yields ``my\_program.o''
        \end{itemize}
        \item \texttt{gcc -c main.c}
        \begin{itemize}
            \item Yields ``main.o''
        \end{itemize}
        \item \texttt{gcc my\_program.o main.o -o my\_program}
        \begin{itemize}
            \item Yields ``my\_program''
            \item Combined code from main.c \& my\_program.c into one executable file, despite being in separate source code files.
        \end{itemize}
    \end{itemize}
\end{frame}

\begin{frame}{Makefile}
    Introducing \textit{Make}!
    \begin{itemize}
        \item The file is called ``Makefile'' and is in the main directory of your code.
        \item It contains the commands needed to compile everything
        \begin{itemize}
            \item And often the commands to clean up after (remove the executable \& intermediate .o files)
        \end{itemize}
        \item Easier than remembering the commands
        \item Only compiles what's needed
        \begin{itemize}
            \item If you only changed one .c file, you don't need to re-compile the other .c files each time you want to test the code.
        \end{itemize}
    \end{itemize}
\end{frame}

\begin{frame}[fragile]{Makefile example}
\begin{verbatim}
OPTS=-Wall
all: my_program
my_program: main.o part1.o part2.o
    gcc -o my_program main.o part1.o part2.o ${OPTS}
main.o: main.c part1.h
    gcc -c main.c ${OPTS}
part1.o: part1.c part1.h part2.h
    gcc -c part1.c ${OPTS}
part2.o: part2.c part2.h
    gcc -c part2.c ${OPTS}

.phony: clean
clean:
    rm -f *.o my_program
\end{verbatim}
\end{frame}

\subsection{Software Development}
\begin{frame}{Build Environments}
    \begin{itemize}
        \item C, C++: AutoTools, Make, CMake, others
        \item Java: Ant, Maven, Gradle
        \item Go: Go modules
        \item Rust: Cargo
    \end{itemize}
    The person compiling someone else’s code shouldn't have to worry about the needed compiler flags, build order, etc.
    \\ ~~ \\
    Whatever the language or the build environment, the goal is to make easily reproducible builds regardless of the environment.
    \\ ~~ \\
    \footnotesize
    And so many more: \url{https://en.wikipedia.org/wiki/List\_of\_build\_automation\_software}
\end{frame}

\begin{frame}{Software Sharing \& Collaboration}
    People, independent developers, small teams all the way to large companies and big projects use code management software to store and manage code.  These programs are said to host \textit{code repositories}, with the most popular being \texttt{git}. It allows for people to keep track when/how a file was created/edited and by whom. It shows all the files and the history for each.
    \\ ~~ \\
    A great example is that of the Linux kernel hosted on the popular software repository hosting website, Github. \\
    \url{https://github.com/torvalds/linux}
\end{frame}

\end{document}