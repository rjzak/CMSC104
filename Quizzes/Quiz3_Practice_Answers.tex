% Copyright 2002-2024 The University of Maryland Baltimore County (UMBC)
% 1000 Hilltop Circle, Baltimore, Maryland, 21250, USA
% https://www.csee.umbc.edu/

\documentclass[letter,11pt]{article}
\usepackage[breaklinks]{hyperref}
\hypersetup{
    bookmarks=true,         % show bookmarks bar?
    unicode=false,          % non-Latin characters in Acrobat’s bookmarks
    pdftoolbar=true,        % show Acrobat’s toolbar?
    pdfmenubar=true,        % show Acrobat’s menu?
    pdffitwindow=false,     % window fit to page when opened
    pdfstartview={XYZ null null 1.00},    % disable zoom
    pdftitle={Practice Quiz 3 Answers},    % title
    pdfauthor={Richard Zak},     % author
    pdfsubject={UMBC CMSC104 Problem Solving and Computer Programming},   % subject of the document
    pdfkeywords={Computer Science, Programming, Problem Solving, CSEE}, % list of keywords
    pdfnewwindow=true,      % links in new PDF window
    colorlinks=false,       % false: boxed links; true: colored links
    linkcolor=red,          % color of internal links (change box color with linkbordercolor)
    citecolor=green,        % color of links to bibliography
    filecolor=magenta,      % color of file links
    urlcolor=cyan           % color of external links
}
\usepackage{graphicx}
\usepackage{fancyhdr}
\usepackage{multicol}
\pagestyle{fancy}
\usepackage[letterpaper, margin=1in]{geometry}
\geometry{letterpaper}
\usepackage{parskip} % Disable initial indent
\usepackage{color,soul} % Highligher
\usepackage[normalem]{ulem} % Strikethrough with \sout{}

\usepackage[utf8]{inputenc}
\fancyhf{}
\renewcommand{\headrulewidth}{0pt} % Remove default underline from header package
\rhead{CMSC 104 Section 01: Practice Quiz 3 Answers}
%\rhead{}
\lhead{\begin{picture}(0,0) \put(0,-10){\includegraphics[width=1.1cm]{Images/UMBC-vertical}} \end{picture}}
\cfoot{\thepage}
\rfoot{\input{semester}}
\lfoot{CMSC 104 Section 01}
\AtEndDocument{\vfill \footnotesize{Last modified: 19 October 2021}}
\AtEndDocument{\rfoot{\input{semester}}}
\renewcommand\thesubsection{\arabic{subsection}} % Show only subsection numbers, not section.subsection

\begin{document}

\huge
\textbf{Practice Quiz 3 Answers}
\normalsize

\paragraph{}Please answer the following questions. Partial credit may be given for incomplete free response questions.

\begin{enumerate}
    \item Fill in the blanks:
    \begin{itemize}
        \item A(n) \underline{~~ event ~~}-controlled loop does not know how many times it will iterate.
        \item A(n) \underline{~~ counter ~~}-controlled loop knows when the condition will be true.
        \item A(n) \underline{~~ sentinel ~~} value can terminate a loop (value type, not \texttt{break;}).
        \item A(n) \underline{~~ variable ~~} can initialize a loop.
    \end{itemize}
    
    \item Complete the program below so that it will print out numbers 4 through 10, 1 per line.
    \begin{verbatim}
#include<stdio.h>
int main() {
    int num;
    for(num = 4; num <= 10; num++) {
        printf("%d\n", num);
    }
    return 0;
}
    \end{verbatim}
    
    \item Fill in the blank in each of the following statements:
    \begin{itemize}
        \item To exit a loop with the condition (\texttt{number == 0}), \texttt{number} must be \underline{greater} than 0.
        \item To exit a loop with the condition (\texttt{number != 0}), \texttt{number} must be \underline{greater or less} than 0.
        \item Loop \underline{~~ modification ~~} is done to change the value of a loop control variable.
        \item A do-while loop will always iterate at \underline{~~ least ~~} one time.
    \end{itemize}
    
    \item Write C code to give an example of an event-controlled loop and an example of a counter-controlled loop, and make sure you do not show infinite loop. \\
    \underline{Event-controlled}:
    \begin{verbatim}
int num;
scanf("%d", &num):
while(num < 0) {
    printf("%d is negative, enter a positive integer: ");
    scanf("%d", &num);
}
    \end{verbatim}
    \underline{Counter-controlled}:
    \begin{verbatim}
int num;
for(num = 0; num <= 10; num += 2) {
    printf("%d\n", num);
}
    \end{verbatim}
    
    \item Describe logic issues with the following code:
\begin{verbatim}
do {
    printf("Is the introduction over (y/n)? ");
    scanf("%c%c", &reply, &cr);
    if (reply == 'n' && reply == 'N') {
        printf("Do what you want.\n");
    }
} while (reply != 'y' && reply != 'Y');
\end{verbatim}
\begin{itemize}
        \item Variables not declared
        \item Reply can't be both 'y' AND 'Y'
    \end{itemize}

    \item Discuss when you would use a for loop vs a while loop or a do-while loop, and give a pseudocode example for each. \\
    For loops are typically for counter-controlled loops when you know how many times a loop should run. While and do-while loops are typically for event-controlled loops when you need to either reach a sentinel value or make some condition false to break the loop. 
    \begin{verbatim}
For example:
int num;
for(num = 1; num <= 10; num = num + 1) {
    printf(“%d\n”, num);
}

While example:
while(not edge) {
    run();
}

Do-While example:
do {
    run();
} while (not edge);
    \end{verbatim}
    
    \item Which type(s) of loops would you use in each of the following situations?
    \begin{itemize}
        \item Asking the user if the song is over. \\
            \textbf{while, do-while}
        \item Counting how many times it takes the user to guess the program's number. \\
            \textbf{while, do-while}
        \item Asking the user to enter six classwork grades so far. \\
            \textbf{for}
        \item Printing the 16 steps of the Cupid Shuffle. \\
            \textbf{for}
    \end{itemize}
    
    \item How are the loops controlled (event or counter) in the following situations?
    \begin{itemize}
        \item Asking the user if the program has guessed their number or not. \\
            \textbf{event}
        \item Printing the name of each of the 20 students in the class. \\
            \textbf{counter}
        \item Asking the user for an unknown number of class grades. \\
            \textbf{event}
        \item Counting how many times it takes the user to guess the program's number. \\
            \textbf{event}
    \end{itemize}
    
    \item What is the output of the following code?
    \begin{verbatim}
int heightCounter, height = 2, widthCounter, width = 4;
int totalStars = 0;

heightCounter = 0;
while (heightCounter < height) {
   widthCounter = 0;
   while (widthCounter < width) {
      printf("*");
      widthCounter = widthCounter + 1;
      totalStars = totalStars + 1;
   }
   printf("\n");
   heightCounter = heightCounter + 1;
}
    \end{verbatim}
    \begin{verbatim}
Output:
****
****
    \end{verbatim}
    
    \item Convert the following numbers to hexadecimal (pay attention, not all are decimals).
    \begin{itemize}
        \item 14 \\
            \textbf{0x0E}
        \item 010110 \\
            \textbf{0x16}
        \item 001100 \\
            \textbf{0x0C}
        \item 017 (octal) \\
            \textbf{0x0F}
    \end{itemize}
\end{enumerate}

\end{document}