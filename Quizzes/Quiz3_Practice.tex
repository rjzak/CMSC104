% Copyright 2002-2023 The University of Maryland Baltimore County (UMBC)
% 1000 Hilltop Circle, Baltimore, Maryland, 21250, USA
% https://www.csee.umbc.edu/

\documentclass[letter,11pt]{article}
\usepackage[breaklinks]{hyperref}
\hypersetup{
    bookmarks=true,         % show bookmarks bar?
    unicode=false,          % non-Latin characters in Acrobat’s bookmarks
    pdftoolbar=true,        % show Acrobat’s toolbar?
    pdfmenubar=true,        % show Acrobat’s menu?
    pdffitwindow=false,     % window fit to page when opened
    pdfstartview={XYZ null null 1.00},    % disable zoom
    pdftitle={Practice Quiz 3},    % title
    pdfauthor={Richard Zak},     % author
    pdfsubject={UMBC CMSC104 Problem Solving and Computer Programming},   % subject of the document
    pdfkeywords={Computer Science, Programming, Problem Solving, CSEE}, % list of keywords
    pdfnewwindow=true,      % links in new PDF window
    colorlinks=false,       % false: boxed links; true: colored links
    linkcolor=red,          % color of internal links (change box color with linkbordercolor)
    citecolor=green,        % color of links to bibliography
    filecolor=magenta,      % color of file links
    urlcolor=cyan           % color of external links
}
\usepackage{graphicx}
\usepackage{fancyhdr}
\usepackage{multicol}
\pagestyle{fancy}
\usepackage[letterpaper, margin=1in]{geometry}
\geometry{letterpaper}
\usepackage{parskip} % Disable initial indent
\usepackage{color,soul} % Highligher
\usepackage[normalem]{ulem} % Strikethrough with \sout{}

\usepackage[utf8]{inputenc}
\fancyhf{}
\renewcommand{\headrulewidth}{0pt} % Remove default underline from header package
\rhead{CMSC 104 Section 01: Practice Quiz 3}
%\rhead{}
\lhead{\begin{picture}(0,0) \put(0,-10){\includegraphics[width=1.1cm]{Images/UMBC-vertical}} \end{picture}}
\cfoot{\thepage}
\rfoot{\input{semester}}
\lfoot{CMSC 104 Section 01}
\AtEndDocument{\vfill \footnotesize{Last modified: 19 October 2021}}
\AtEndDocument{\rfoot{\input{semester}}}
\renewcommand\thesubsection{\arabic{subsection}} % Show only subsection numbers, not section.subsection

\begin{document}

\huge
\textbf{Practice Quiz 3}
\normalsize

\paragraph{}Please answer the following questions. Partial credit may be given for incomplete free response questions.

\begin{enumerate}
    \item Fill in the blanks:
    \begin{itemize}
        \item A(n) \underline{~~ ~~ ~~ ~~ ~~ ~~ ~~ ~~}-controlled loop does not know how many times it will iterate.
        \item A(n) \underline{~~ ~~ ~~ ~~ ~~ ~~ ~~ ~~}-controlled loop knows when the condition will be true.
        \item A(n) \underline{~~ ~~ ~~ ~~ ~~ ~~ ~~ ~~} value can terminate a loop (value type, not \texttt{break;}).
        \item A(n) \underline{~~ ~~ ~~ ~~ ~~ ~~ ~~ ~~} can initialize a loop.
    \end{itemize}
    
    \item Complete the program below so that it will print out numbers 4 through 10, 1 per line.
    \begin{verbatim}
____________<stdio.h>
int main() {
    int num;
    for(________; ________; ________) {
        printf("%d\n", __________);
    }
    return 0;
}
    \end{verbatim}
    
    \item Fill in the blank in each of the following statements:
    \begin{itemize}
        \item To exit a loop with the condition (\texttt{number == 0}), \texttt{number} must be \underline{~~ ~~ ~~} than 0.
        \item To exit a loop with the condition (\texttt{number != 0}), \texttt{number} must be \underline{~~ ~~ ~~} than 0.
        \item Loop \underline{~~ ~~ ~~ ~~ ~~ ~~ ~~ ~~} is done to change the value of a loop control variable.
        \item A do-while loop will always iterate at \underline{~~ ~~ ~~ ~~ ~~ ~~ ~~ ~~} one time.
    \end{itemize}
    
    \item Write C code to give an example of an event-controlled loop and an example of a counter-controlled loop, and make sure you do not show infinite loop.
    
    \item Describe logic issues with the following code:
\begin{verbatim}
do {
    printf("Is the introduction over (y/n)? ");
    scanf("%c%c", &reply, &cr);
    if (reply == 'n' && reply == 'N') {
        printf("Do what you want.\n");
    }
} while (reply != 'y' && reply != 'Y');
\end{verbatim}

    \item Discuss when you would use a for loop vs a while loop or a do-while loop, and give a pseudocode example for each.
    
    \item Which type(s) of loops would you use in each of the following situations?
    \begin{itemize}
        \item Asking the user if the song is over.
        \item Counting how many times it takes the user to guess the program's number.
        \item Asking the user to enter six classwork grades so far.
        \item Printing the 16 steps of the Cupid Shuffle.
    \end{itemize}
    
    \item How are the loops controlled (event or counter) in the following situations?
    \begin{itemize}
        \item Asking the user if the program has guessed their number or not.
        \item Printing the name of each of the 20 students in the class.
        \item Asking the user for an unknown number of class grades.
        \item Counting how many times it takes the user to guess the program's number.
    \end{itemize}
    
    \item What is the output of the following code?
    \begin{verbatim}
int heightCounter, height = 2, widthCounter, width = 4;
int totalStars = 0;

heightCounter = 0;
while (heightCounter < height) {
   widthCounter = 0;
   while (widthCounter < width) {
      printf("*");
      widthCounter = widthCounter + 1;
      totalStars = totalStars + 1;
   }
   printf("\n");
   heightCounter = heightCounter + 1;
}
    \end{verbatim}
    
    \item Convert the following numbers to hexadecimal (pay attention, not all are decimals).
    \begin{itemize}
        \item 14
        \item 010110
        \item 001100
        \item 017 (octal)
    \end{itemize}
\end{enumerate}

\end{document}