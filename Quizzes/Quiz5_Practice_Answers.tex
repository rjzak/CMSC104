% Copyright 2002-2024 The University of Maryland Baltimore County (UMBC)
% 1000 Hilltop Circle, Baltimore, Maryland, 21250, USA
% https://www.csee.umbc.edu/

\documentclass[letter,11pt]{article}
\usepackage[breaklinks]{hyperref}
\hypersetup{
    bookmarks=true,         % show bookmarks bar?
    unicode=false,          % non-Latin characters in Acrobat’s bookmarks
    pdftoolbar=true,        % show Acrobat’s toolbar?
    pdfmenubar=true,        % show Acrobat’s menu?
    pdffitwindow=false,     % window fit to page when opened
    pdfstartview={XYZ null null 1.00},    % disable zoom
    pdftitle={Practice Quiz 5 Answers},    % title
    pdfauthor={Richard Zak},     % author
    pdfsubject={UMBC CMSC104 Problem Solving and Computer Programming},   % subject of the document
    pdfkeywords={Computer Science, Programming, Problem Solving, CSEE}, % list of keywords
    pdfnewwindow=true,      % links in new PDF window
    colorlinks=false,       % false: boxed links; true: colored links
    linkcolor=red,          % color of internal links (change box color with linkbordercolor)
    citecolor=green,        % color of links to bibliography
    filecolor=magenta,      % color of file links
    urlcolor=cyan           % color of external links
}
\usepackage{graphicx}
\usepackage{fancyhdr}
\usepackage{multicol}
\pagestyle{fancy}
\usepackage[letterpaper, margin=1in]{geometry}
\geometry{letterpaper}
\usepackage{bbding} % For checkmark in itemize
\newcommand*\tick{\item[\Checkmark]}
\newcommand*\fail{\item[\XSolidBrush]}
\usepackage{parskip} % Disable initial indent
\usepackage{color,soul} % Highligher
\usepackage[normalem]{ulem} % Strikethrough with \sout{}

\usepackage[utf8]{inputenc}
\fancyhf{}
\renewcommand{\headrulewidth}{0pt} % Remove default underline from header package
\rhead{CMSC 104 Section 01: Practice Quiz 5 Answers}
%\rhead{}
\lhead{\begin{picture}(0,0) \put(0,-10){\includegraphics[width=1.1cm]{Images/UMBC-vertical}} \end{picture}}
\cfoot{\thepage}
\rfoot{\input{semester}}
\lfoot{CMSC 104 Section 01}
\AtEndDocument{\vfill \footnotesize{Last modified: 30 November 2021}}
\AtEndDocument{\rfoot{\input{semester}}}
\renewcommand\thesubsection{\arabic{subsection}} % Show only subsection numbers, not section.subsection

\begin{document}

\huge
\textbf{Practice Quiz 5 Answers}
\normalsize

\paragraph{}Please answer the following questions. Partial credit may be given for incomplete free response questions.

\begin{enumerate}
    \item Fill in the blank in each of the following statements:
    \begin{enumerate}
        \item Function \underline{~~ calls ~~} are made after the function prototype.
        \item The \underline{~~ main ~~} function returns an \texttt{int}.
        \item All elements of an array have the \underline{~~ same~~} data type.
        \item When we don't want to change the parameters, they are passed by \underline{~~ value ~~}.
    \end{enumerate}
    
    \item Write a function prototype for a function which calculates the total bill given the bill subtotal, the percent tax, and suggested tip percentage. The function return type should be \texttt{float}, the function name should be \texttt{calculateTotalBill}, and it will take the following parameters: a \texttt{float} for the bill subtotal called \texttt{subTotal}, an integer for the percent tax called \texttt{percentTax}, and a \texttt{float} for the suggested tip percentage called \texttt{suggestedTipPercentage}. \\
    \underline{Extra Credit:} Implement the function definition.
    
    \begin{verbatim}
float calculateTotalBill(float, int, float);
   -- or --
float calculateTotalBill(float subTotal, int percentTax, float suggestedTipPercentage);

Extra credit:
float calculateTotalBill(float subtotal, int percentTax, float suggestedTipPercentage) {
    float suggestedTip = subtotal * suggestedTipPercentage;
    float taxAsPercentage = percentTax / 100.0;
    float totalTax = subtotal * taxAsPercentage;
    float billBeforeTax = subtotal + totalTax;
    float totalBill = billBeforeTax + suggestedTip;
    return totalBill;
};
    \end{verbatim}
    
    \item Which of the following is \textit{generally not} part of the function definition?
    \begin{itemize}
        \tick function call
    \end{itemize}
    
    \item Write a block of code to declare a 25-element character array called \texttt{charArray} and set all of its elements to `x'.
    \begin{verbatim}
char charArray[25];
int i;
for(i = 0; i < 25; i++) {
    charArray[i] = 'x';
}
    \end{verbatim}
    
    \newpage
    \item Explain why we use external libraries, then give an example of an external library we've already used this semester. \\
    \underline{Extra Credit:} Write a code snippet using this example.
    \begin{verbatim}
We use external libraries to be able to do utilize code other people have written for us 
so we don’t have to rewrite it from scratch every time. For example, stdio.h is a
common external library we use to be able to do printf and scanf in our programs.

Extra Credit:
#include <stdio.h>
int main() {
    printf("Hello World.\n");
    return 0;
}
    \end{verbatim}
    
    \item Review the following code and answer the questions below:
    \begin{verbatim}
#include <stdio.h>

int multiplyTwoNumbers(int , int );

int main() {
   int num1 = 3, num2 = 6, num3;
   num3 = multiplyTwoNumbers(num1, num2);
   printf("Multiplying num1 by num2 = %d\n", num3);
   return 0;
}

int multiplyTwoNumbers(int num2, int num1) {
   return num1 * num2;
}
    \end{verbatim}
    \begin{enumerate}
        \item What is the value of \texttt{num2} in the function \texttt{multiplyTwoNumbers}? {\color{red}3}
        \item What is the value of \texttt{num1} in the function \texttt{multiplyTwoNumbers}? {\color{red}6}
        \item What is the value of \texttt{num3} in the \texttt{printf} statement in main? {\color{red}18}
        \item What is the value of \texttt{num1} after the \texttt{printf} statement in main? {\color{red}3}
        \item What is the value of \texttt{num2} after the \texttt{printf} statement in main?
        {\color{red}6}
    \end{enumerate}
    
    \item Write function prototypes for:
    \begin{enumerate}
        \item A function \texttt{volume} that takes three integers \texttt{a}, \texttt{b}, and \texttt{c} and returns an \texttt{int}. \\
        {\color{red}int volume(int, int, int);} or {\color{red}int volume(int a, int b, int c);}
        \item A function \texttt{e} that takes two \texttt{double}s \texttt{mc} and \texttt{hammered} and returns a \texttt{double}. \\
        {\color{red}double e(double, double);} or {\color{red}double e(double mc, double hammered);}
        \item function \texttt{bac} that takes a double \texttt{pct} and three integers, \texttt{oz}, \texttt{weight}, and \texttt{hours}, and returns a \texttt{double}. \\
        {\color{red}double bac(double, int, int, int);} or {\color{red}double bac(double pct, int oz, int weight, int hours);}
    \end{enumerate}
    
    \item You have a list of 10 class grades, stored as integer values between 0 and 100 and sorted from lowest to highest in an array called \texttt{grades}. Write code to do the following, calculating the index for a and b and using the provided index for c and d:
    \begin{enumerate}
        \item Change the 3\textsuperscript{rd} highest score to a 50. {\color{red}grades[7] = 50;}
        \item Change the 4\textsuperscript{th} lowest score to a 100. {\color{red}grades[3] = 100;}
        \item Add 7 (using +=) to the score in position 2. {\color{red}grades[2] += 7;}
        \item Subtract 1 (using post-decrement -- --) from the score in position 5. {\color{red}grades[5]- -;}
    \end{enumerate}
    
    \item What do the following functions do? Assume function prototypes already exist.
    \begin{verbatim}
int countMisses(int hits[5], int rolls[5], int bonuses[5]) {
  int numMisses = 0;
  for(int i = 0; i < 5; ++i) {
    numMisses += hitOrMiss(hits[i], rolls[i], bonuses[i]);
  }
  return numMisses;
}

int hitOrMiss(int hit, int roll, int bonus) {
  return ((roll == 1) || ((roll + bonus) < hit)));   
}
    \end{verbatim}
    \underline{Extra credit:} What is the maximum possible value from \texttt{countMisses()}? \\
    {\color{red}Notice that countMisses takes 3 integer arrays of length 5 and counts how many times each combination of hits/rolls/bonuses misses. The hitOrMiss function determines whether the provided roll and bonus misses or not, returning true if miss and false if hit. \\ \\ Maximum possible return value from countMisses(): 5.}
    
    \item Fill in the blanks of the following program that initializes an array to store its index values in each element (i.e. index 1 stores 1) and counts how many odd and even values there are. \\
    \underline{Extra Credit:} Print the program’s output.
    \begin{verbatim}
include <stdio.h>
#define SIZE 5
int main() {
  int myArray[SIZE];
  int numEvens = 0, numOdds = 0;
  for (int i = 0; i < SIZE; ++i) {
    myArray[i] = i;
    if ((myArray[i] % 2) != 0) ++numOdds;
    else ++numEvens;
  }
  printf("In an array of size %d, there are:\n", SIZE);
  printf("Odds: %d\n", numOdds);
  printf("Evens: %d\n", numEvens);
  return 0;
}
    \end{verbatim}
    {\color{red}Extra Credit: \\
In an array of size 5, there are: \\
Odds: 2 \\
Evens: 3}
\end{enumerate}

\end{document}