% Copyright 2002-2023 The University of Maryland Baltimore County (UMBC)
% 1000 Hilltop Circle, Baltimore, Maryland, 21250, USA
% https://www.csee.umbc.edu/

\documentclass[letter,11pt]{article}
\usepackage[breaklinks]{hyperref}
\hypersetup{
    bookmarks=true,         % show bookmarks bar?
    unicode=false,          % non-Latin characters in Acrobat’s bookmarks
    pdftoolbar=true,        % show Acrobat’s toolbar?
    pdfmenubar=true,        % show Acrobat’s menu?
    pdffitwindow=false,     % window fit to page when opened
    pdfstartview={XYZ null null 1.00},    % disable zoom
    pdftitle={Practice Quiz 4 Answers},    % title
    pdfauthor={Richard Zak},     % author
    pdfsubject={UMBC CMSC104 Problem Solving and Computer Programming},   % subject of the document
    pdfkeywords={Computer Science, Programming, Problem Solving, CSEE}, % list of keywords
    pdfnewwindow=true,      % links in new PDF window
    colorlinks=false,       % false: boxed links; true: colored links
    linkcolor=red,          % color of internal links (change box color with linkbordercolor)
    citecolor=green,        % color of links to bibliography
    filecolor=magenta,      % color of file links
    urlcolor=cyan           % color of external links
}
\usepackage{graphicx}
\usepackage{fancyhdr}
\usepackage{multicol}
\pagestyle{fancy}
\usepackage[letterpaper, margin=1in]{geometry}
\geometry{letterpaper}
\usepackage{parskip} % Disable initial indent
\usepackage{color,soul} % Highligher
\usepackage[normalem]{ulem} % Strikethrough with \sout{}
\usepackage{xcolor}

\usepackage[utf8]{inputenc}
\fancyhf{}
\renewcommand{\headrulewidth}{0pt} % Remove default underline from header package
\rhead{CMSC 104 Section 01: Practice Quiz 4 Answers}
%\rhead{}
\lhead{\begin{picture}(0,0) \put(0,-10){\includegraphics[width=1.1cm]{Images/UMBC-vertical}} \end{picture}}
\cfoot{\thepage}
\rfoot{\input{semester}}
\lfoot{CMSC 104 Section 01}
\AtEndDocument{\vfill \footnotesize{Last modified: 16 November 2021}}
\AtEndDocument{\rfoot{\input{semester}}}
\renewcommand\thesubsection{\arabic{subsection}} % Show only subsection numbers, not section.subsection

\begin{document}

\huge
\textbf{Practice Quiz 4 Answers}
\normalsize

\paragraph{}Please answer the following questions. Partial credit may be given for incomplete free response questions.

\begin{enumerate}
    \item Fill in the blank in each of the following statements:
    \begin{enumerate}
        \item \underline{~~ switch ~~} statements deal with integer expressions.
        \item \underline{~~ if ~~ ~~} statements deal with conditional expressions.
        \item \underline{~~ break ~~} is usually the last statement in a \texttt{switch} statement.
        \item \underline{~~ default ~~} usually indicates an invalid value in a \texttt{switch} statement.
    \end{enumerate}
    
    \item Write a \texttt{switch} statement for the character variable \texttt{letter}, which should be a value between `a' and `z'. When \texttt{letter} is `a', `e', `i', `o', or `u', tell the user that \texttt{letter} is a vowel. The default case should let the user know that \texttt{letter} is a consonant.
    \begin{verbatim}
switch (letter) {
    case `A': case `a':
    case `E': case `e':
    case `I': case `i':
    case `O': case `o':
    case `U': case `u':
        printf(“%c is a vowel.\n”, letter);
        break;
    default:
        printf(“%c is a consonant.\n”, letter);
        break;
}
    \end{verbatim}
    
    \item Given: \texttt{int a = 1, b = 2, c = 3, d = 4;} what is the value for each of the following expressions? Treat each expression independently, each expression does not impact the result of the others.
    \begin{itemize}
        \item \texttt{++d \% b * c++ -a =} {\color{red}2}
        \item \texttt{c / d++ - a+ ++b =} {\color{red}2}
        \item \texttt{b * ++a / c++ +d =} {\color{red}5}
    \end{itemize}
    
    \item Given: \texttt{int a = 4, b = 3, c = 2, d = 1;} what is the value of the following calculations. Each expression impacts the result of the subsequent expressions. Indicate the ending value of the variables. \\
    \texttt{a += d \% c;} \\
    \texttt{b *= c /= a;} \\
    \texttt{d -= a + 2;}
    \begin{enumerate}
        \item {\color{red}5}
        \item {\color{red}0}
        \item {\color{red}0}
        \item {\color{red}-6}
    \end{enumerate}
    
    \item Describe the logic issues with this code snippet:
    \begin{verbatim}
int day = 5;
switch (day) {
    case 1:
        printf("Monday\n");
    case 2:
        printf("Tuesday\n");
        break;
    case 3:
        printf("Wednesday\n");
        break;
    case 4:
        printf("Thursday\n");
        break;
    case 5:
        printf("Friday\n");
    default:
        printf("Weekend\n");
        break;
}
    \end{verbatim}
    {\color{red}There are missing \texttt{break} statements in the \texttt{case} statements for Monday and Friday.} \\
    How would you correct the issues? {\color{red}Add the \texttt{break} statements.}
    
    \item Which assignment operator should be used for each scenario?
    \begin{enumerate}
        \item Combining ingredients in a recipe. {\color{red}+=}
        \item Calculating force equals mass times acceleration. {\color{red}*=}
        \item Determining students per teacher. {\color{red}/=}
        \item Determining if a number is even or odd. {\color{red}\%=}
    \end{enumerate}
    
    \item Fill in the blank in each of the following statements:
    \begin{enumerate}
        \item A(n) \underline{~~ switch or case ~~} statement cannot handle conditional expressions.
        \item A(n) \underline{~~ if ~~} statement can handle more than just integer expressions.
        \item An \texttt{if} statement may be \underline{~~ harder ~~} to read than a \texttt{switch} statement.
        \item It's harder to add new cases to a(n) \underline{~~ if ~~} statement.
    \end{enumerate}
    
    \item What does the following code snippet do? Assume all required \texttt{\#include} statements exist.
    \begin{verbatim}
int numR=0, numS=0, numT=0, numL=0, numN=0, numE=0;
int count, totalLetters = 25;
char letter;

srandom(time(0));
for (count = 0; count < totalLetters; ++count) {
   letter = "RSTLNE" [random() % 6];
   switch (letter) {
      case `R': ++numR; break;
      case `S': ++numS; break;
      case `T': ++numT; break;
      case `L': ++numL; break;
      case `N': ++numN; break;
      default: ++numE; break;
   }
}
    \end{verbatim}
    {\color{red}The code snippet generates 25 random letters from RSTLNE, counting how many times each letter is randomly generated.} \\
    Extra credit: Why is it okay for the \texttt{default} case to increment \texttt{numE}? {\color{red}It's okay because the variable was initially had the value of zero. It would be a problem if \texttt{numE} wasn't initialized.}
    
    \item Write a function which receives three integers and returns the average. Carefully consider your data type choice.
    \begin{verbatim}
float average_of_three(int a, int b, int c) {
    float avg = a + b + c;
    avg /= 3.0;
    return avg;
}
    \end{verbatim}
\end{enumerate}

\end{document}