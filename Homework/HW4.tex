% Copyright 2002-2023 The University of Maryland Baltimore County (UMBC)
% 1000 Hilltop Circle, Baltimore, Maryland, 21250, USA
% https://www.csee.umbc.edu/

\documentclass[letter,11pt]{article}
\usepackage[breaklinks]{hyperref}
\hypersetup{
    bookmarks=true,         % show bookmarks bar?
    unicode=false,          % non-Latin characters in Acrobat’s bookmarks
    pdftoolbar=true,        % show Acrobat’s toolbar?
    pdfmenubar=true,        % show Acrobat’s menu?
    pdffitwindow=false,     % window fit to page when opened
    pdfstartview={XYZ null null 1.00},    % disable zoom
    pdftitle={Homework 4},    % title
    pdfauthor={Richard Zak},     % author
    pdfsubject={UMBC CMSC104 Problem Solving and Computer Programming},   % subject of the document
    pdfkeywords={Computer Science, Programming, Problem Solving, CSEE}, % list of keywords
    pdfnewwindow=true,      % links in new PDF window
    colorlinks=false,       % false: boxed links; true: colored links
    linkcolor=red,          % color of internal links (change box color with linkbordercolor)
    citecolor=green,        % color of links to bibliography
    filecolor=magenta,      % color of file links
    urlcolor=cyan           % color of external links
}
\usepackage{graphicx}
\usepackage{fancyhdr}
\usepackage{multicol}
\pagestyle{fancy}
\usepackage[letterpaper, margin=1in]{geometry}
\geometry{letterpaper}
\usepackage{parskip} % Disable initial indent
\usepackage{color,soul} % Highligher
\usepackage[normalem]{ulem} % Strikethrough with \sout{}

\usepackage[utf8]{inputenc}
\fancyhf{}
\renewcommand{\headrulewidth}{0pt} % Remove default underline from header package
\rhead{CMSC 104 Section 01: Homework 4}
%\rhead{}
\lhead{\begin{picture}(0,0) \put(0,-10){\includegraphics[width=1.1cm]{Images/UMBC-vertical}} \end{picture}}
\cfoot{\thepage}
\rfoot{\input{semester}}
\lfoot{CMSC 104 Section 01}
\AtEndDocument{\vfill \footnotesize{Last modified: 08 February 2023}}
\AtEndDocument{\rfoot{\input{semester}}}
\renewcommand\thesubsection{\arabic{subsection}} % Show only subsection numbers, not section.subsection

\begin{document}

\huge
\textbf{Homework 4: More Input \& Output}
\normalsize
\\ ~~ \\
\textbf{Assigned: Tuesday 05 March} \\
\textbf{Due Date: Monday 11 March}

\section*{Objectives}
\paragraph{}To gain more experience with \texttt{printf()} and \texttt{scanf()}, variables, and arithmetic operators.

\paragraph{Reminder:} Assignments are an independent effort. \underline{This is not a group effort}. Assignments are checked to ensure they aren't too similar to that of other students'.

\section*{Part 1: Inches to Feet and Inches}
\paragraph{}Write a program that asks the user for their name and their height in inches, then replies with their name and height in feet and inches. 

\subsection*{Example}
\begin{verbatim}
[rzak1@linux1 hw4]$ gcc -Wall height2.c
[rzak1@linux1 hw4]$ ./a.out
What is your name? Alice
How tall are you in inches? 79
Hello, Alice. You are 6 feet 7 inches tall.
[rzak1@linux1 hw4]$ 
\end{verbatim}

\subsection*{Notes}
\begin{enumerate}
    \item Login to GL and make sure you are in your home directory (pwd).
    \item Change directory to hw4 (cd cmsc104/hw4) so you can do this assignment in a fresh workspace.
    \item A good starting point is height.c from Classwork 4, which should be in your cmsc104/cw4 directory. Make a copy of height.c  (\texttt{cp ../cw4/height.c height2.c}) and update the header comment block accordingly.
    \item The first thing you will need to do is decide if you need any new variables, and if so, what data type they should be. Remember that these should all be declared under the existing variables.
    \item In your calculation block, you will use the division operator / and the modulus operator \% to convert inches to feet and inches.
    \item REMEMBER: In C, when dividing two integer values, the remainder (everything after the decimal point) is thrown out. For example, 17/5 gives you 3.
\end{enumerate}

\section*{Part 2: Centimeters to Feet and Inches}
\paragraph{}Write a program that asks the user for their name and their height in centimeters, then replies with their name and height in feet and inches. 

\subsection*{Example}
\begin{verbatim}
[rzak1@linux1 hw4]$ gcc -Wall height3.c
[rzak1@linux1 hw4]$ ./a.out
What is your name? Gawain
How tall are you in centimeters? 195
Hello, Gawain. You are 6 feet 5 inches tall.
[rzak1@linux1 hw4]$ 
\end{verbatim}

\subsection*{Notes}
\begin{enumerate}
    \item A good starting point is height2.c from Part 1. Make a copy of height2.c (\texttt{cp height2.c height3.c}) and update the header comment block accordingly.
    \item The first thing you will need to do is decide if you need any new variables, and if so, what data type they should be. Remember that these should all be declared under the existing variables.
    \item REMEMBER: Feet and inches are always whole numbers, but centimeters do not have to be.
    \item Once you are confident you have all the variables you need, you must change the second user prompt to ask for centimeters instead of inches.
    \item Next, in your calculation block, convert centimeters to inches, leaving the rest of the calculations from height2.c, as they should still work to calculate feet and inches.
    \item REMEMBER: When you divide an integer value by a floating point value, you get a floating point value. If you assign a floating point value to an integer variable, the fractional part is thrown away. So, mathematically 10 / 2.54 is 3.937... but if you assign this to an integer variable, you get 3. For example, after the assignment $n = 10 / 2.54$ the integer variable $n$ would have value 3. To achieve rounding, you can add 0.5 to the calculation: $n = 10 / 2.54 + 0.5$; $m = 8 / 2.54 + 0.5$. The $n$ would hold 4 and $m$ would hold 3. (Assuming that both $n$ and $m$ are integer variables.)
\end{enumerate}

\section*{Grading Rubric}
\begin{itemize}
    \item height2.c header comments: 2 points
    \item height2.c body comments: 3 points
    \item height2.c compiles: 15 points
    \item height2.c does accurate calculation: 25 points
    \item height3.c header comments: 2 points
    \item height3.c body comments: 3 points
    \item height3.c compiles: 15 points
    \item height3.c does accurate calculation: 25 points
    \item typescript: 10 points
\end{itemize}

\section*{What to Submit}
\paragraph{}Use the \texttt{script} command to record yourself compiling your programs and running each 3 times, using a different name and number for inches and centimeters each time, or you will not get full credit. \underline{Do not record yourself editing your file!} Use \texttt{exit} to terminate the recording. Then submit your programs and typescript file.
\begin{verbatim}
[rzak1@linux1 hw4]$ submit cmsc104_rzak1 hw4 height2.c height3.c typescript
\end{verbatim}

\subsection*{Verify Submission}
\paragraph{}If you \textit{think} you submitted the assignment, but the \texttt{submitls} command doesn't show you your file names, then the files were \textbf{not} submitted and no grade will be given.
\begin{verbatim}
[rzak1@linux1 hw4]$ submitls cmsc104_rzak1 hw4
\end{verbatim}

\end{document}