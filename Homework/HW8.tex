% Copyright 2002-2024 The University of Maryland Baltimore County (UMBC)
% 1000 Hilltop Circle, Baltimore, Maryland, 21250, USA
% https://www.csee.umbc.edu/

\documentclass[letter,11pt]{article}
\usepackage[breaklinks]{hyperref}
\hypersetup{
    bookmarks=true,         % show bookmarks bar?
    unicode=false,          % non-Latin characters in Acrobat’s bookmarks
    pdftoolbar=true,        % show Acrobat’s toolbar?
    pdfmenubar=true,        % show Acrobat’s menu?
    pdffitwindow=false,     % window fit to page when opened
    pdfstartview={XYZ null null 1.00},    % disable zoom
    pdftitle={Homework 8},    % title
    pdfauthor={Richard Zak},     % author
    pdfsubject={UMBC CMSC104 Problem Solving and Computer Programming},   % subject of the document
    pdfkeywords={Computer Science, Programming, Problem Solving, CSEE}, % list of keywords
    pdfnewwindow=true,      % links in new PDF window
    colorlinks=false,       % false: boxed links; true: colored links
    linkcolor=red,          % color of internal links (change box color with linkbordercolor)
    citecolor=green,        % color of links to bibliography
    filecolor=magenta,      % color of file links
    urlcolor=cyan           % color of external links
}
\usepackage{graphicx}
\usepackage{fancyhdr}
\usepackage{multicol}
\usepackage{placeins}
\pagestyle{fancy}
\usepackage[letterpaper, margin=1in]{geometry}
\geometry{letterpaper}
\usepackage{parskip} % Disable initial indent
\usepackage{color,soul} % Highligher
\usepackage[normalem]{ulem} % Strikethrough with \sout{}

\usepackage[utf8]{inputenc}
\fancyhf{}
\renewcommand{\headrulewidth}{0pt} % Remove default underline from header package
\rhead{CMSC 104 Section 01: Homework 8}
%\rhead{}
\lhead{\begin{picture}(0,0) \put(0,-10){\includegraphics[width=1.1cm]{Images/UMBC-vertical}} \end{picture}}
\cfoot{\thepage}
\rfoot{\input{semester}}
\lfoot{CMSC 104 Section 01}
\AtEndDocument{\vfill \footnotesize{Last modified: 08 February 2023}}
\AtEndDocument{\rfoot{\input{semester}}}
\renewcommand\thesubsection{\arabic{subsection}} % Show only subsection numbers, not section.subsection

\begin{document}

\huge
\textbf{Homework 8: Class Grades Simulator++}.
\normalsize
\\ ~~ \\
\textbf{Assigned: Tuesday 25 April} \\
\textbf{Due Date: Monday 01 May} %\footnote{Extra time provided due to Thanksgiving.}

\section*{Objectives}
\paragraph{}More practice writing a program that uses a \texttt{switch} statement, the \texttt{+=} assignment operator, and a \texttt{for} loop.

\section*{Assignment}
\paragraph{}This assignment is a continuation of Classwork 8. (NOTE: You must still submit Classwork 8 and Homework 8 separately.) For this homework assignment, embellish your program from Classwork 8 with the following features:
\begin{enumerate}
    \item Do a second randomization inside each case statement to determine an actual numeric value for the grade, and for any non-F grade, determine whether the grade is a +, regular, or -. This will require a nested if-else if-else block. For example, 90-93 is A-, 94-97 is A, and 98-100 is A+.
    \item After the \texttt{for} loop terminates, print out the \textit{highest} and \textit{lowest} grades generated by the program.
    \item After the \texttt{for} loop terminates, print out the \textit{average} grade generated by the program. Use a comment when you're doing this calculation to explain why you can calculate the highest and lowest grades inside of the loop, but the average grade is best computed outside of it.
\end{enumerate}

\subsection*{Example Compilation and Execution}
\begin{verbatim}
[rzak1@linux1 hw8]$ gcc -Wall grades2.c
[rzak1@linux1 hw8]$ ./a.out
Out of 36 students, here is the class grade breakdown:
A+: 4
A: 0
A-: 0
B+: 3
B: 3
B-: 3
C+: 3
C: 1
C-: 3
D+: 3
D: 3
D-: 2
F: 9
Highest grade: 100
Lowest grade: 2
Average grade: 66.53
[rzak1@linux1 hw8]$ 
\end{verbatim}

\subsection*{Notes}
\begin{itemize}
    \item Make a \textit{copy} of your program from Classwork 8.
    \begin{verbatim}
    [rzak1@linux1 ~]$ cp ~/cs104/cw8/grades.c ~/cs104/hw8/grades2.c   
    \end{verbatim}
    \item Think about how many additional \texttt{num} variables you need to create to account for all the +s and -s.
    \item Think about how to calculate the sum of the numbers generated by the program. Use a \texttt{float} variable so you get a floating-point value when you compute the average.
    \item Since the numbers generated by the program must be positive, you can initialize the maximum value to 0. \underline{Explain why this works in the variable declaration comment.}
    \item Make sure that \underline{you indent each block of code} so that you don't get lost and your code compiles.
    \item Make sure you review the notes for Classwork 8, as they still apply here as well.
    \item Use the following commands to generate the random numerical grades:
    \begin{itemize}
        \item If A: \texttt{grade = (random() \% 11) + 90;}
        \item If B: \texttt{grade = (random() \% 10) + 80;}
        \item If C: \texttt{grade = (random() \%10) + 70;}
        \item If D: \texttt{grade = (random() \% 10) + 60;}
        \item If F: \texttt{grade = random() \% 60;}
    \end{itemize}
\end{itemize}

\subsection*{Extra Credit}
\paragraph{}Try to do the following embellishments (notice they are the same as Classwork 8):
\paragraph{Option 1:} After the printout of results, add an if-else block that tells the user if First-Year Intervention (FYI) notifications need to be sent, and if so, how many. To determine if any need to be sent, count how many Ds and Fs there are.
\paragraph{Option 2:} Ask the user for how many grades to simulate. Warn the user if they enter something less than 0. If the user enters 0, print the following to the screen: ``Sorry to see you don’t want to use my simulator'' with a newline at the end for proper display on screen.
\paragraph{Option 3:} Change the provided for loop to a while loop. 

\section*{Grading Rubric}
\begin{itemize}
    \item Typescript: 15 points
    \item Header comments: 2 points
    \item Body comments: 3 points
    \item Compiles: 40 points
    \item Gets Max: 15 points
    \item Gets Min: 15 points
    \item Gets Avg: 10 points
    \item EC1: +5 points
    \item EC2: +5 points
    \item EC3: +5 points
\end{itemize}

\section*{What to Submit}
\paragraph{}Use the \texttt{script} command to record yourself compiling and running your program 3 times. (Do not record yourself editing your program!) Exit from script. Submit your program and the typescript file.
\begin{verbatim}
[rzak1@linux1 hw8]$ submit cmsc104_rzak1 hw8 grades2.c typescript
\end{verbatim}

\subsection*{Verify Submission}
\paragraph{}If you \textit{think} you submitted the assignment, but the \texttt{submitls} command doesn't show you your file names, then the files were \textbf{not} submitted and no grade will be given.
\begin{verbatim}
[rzak1@linux1 hw8]$ submitls cmsc104_rzak1 hw8
\end{verbatim}

\end{document}