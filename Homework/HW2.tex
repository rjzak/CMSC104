% Copyright 2002-2024 The University of Maryland Baltimore County (UMBC)
% 1000 Hilltop Circle, Baltimore, Maryland, 21250, USA
% https://www.csee.umbc.edu/

\documentclass[letter,11pt]{article}
\usepackage[breaklinks]{hyperref}
\hypersetup{
    bookmarks=true,         % show bookmarks bar?
    unicode=false,          % non-Latin characters in Acrobat’s bookmarks
    pdftoolbar=true,        % show Acrobat’s toolbar?
    pdfmenubar=true,        % show Acrobat’s menu?
    pdffitwindow=false,     % window fit to page when opened
    pdfstartview={XYZ null null 1.00},    % disable zoom
    pdftitle={Homework 2},    % title
    pdfauthor={Richard Zak},     % author
    pdfsubject={UMBC CMSC104 Problem Solving and Computer Programming},   % subject of the document
    pdfkeywords={Computer Science, Programming, Problem Solving, CSEE}, % list of keywords
    pdfnewwindow=true,      % links in new PDF window
    colorlinks=false,       % false: boxed links; true: colored links
    linkcolor=red,          % color of internal links (change box color with linkbordercolor)
    citecolor=green,        % color of links to bibliography
    filecolor=magenta,      % color of file links
    urlcolor=cyan           % color of external links
}
\usepackage{graphicx}
\usepackage{fancyhdr}
\usepackage{multicol}
\pagestyle{fancy}
\usepackage[letterpaper, margin=1in]{geometry}
\geometry{letterpaper}
\usepackage{parskip} % Disable initial indent
\usepackage{color,soul} % Highligher
\usepackage[normalem]{ulem} % Strikethrough with \sout{}

\usepackage[utf8]{inputenc}
\fancyhf{}
\renewcommand{\headrulewidth}{0pt} % Remove default underline from header package
\rhead{CMSC 104 Section 01: Homework 2}
%\rhead{}
\lhead{\begin{picture}(0,0) \put(0,-10){\includegraphics[width=1.1cm]{Images/UMBC-vertical}} \end{picture}}
\cfoot{\thepage}
\rfoot{\input{semester}}
\lfoot{CMSC 104 Section 01}
\AtEndDocument{\vfill \footnotesize{Last modified: 08 February 2023}}
\AtEndDocument{\rfoot{\input{semester}}}
\renewcommand\thesubsection{\arabic{subsection}} % Show only subsection numbers, not section.subsection

\begin{document}

\huge
\textbf{Homework 2: More Practice with Linux} \includegraphics[scale=0.07]{Images/Tux.png}
\normalsize
\\ ~~ \\
\textbf{Assigned: Tuesday 13 February} \\
\textbf{Due Date: Monday 19 February}

\section*{Objectives}
\paragraph{}To become more familiar with navigating the Linux operating system, and understand how to use basic Unix commands. This assignment builds on the work which was done in Classwork 2.

\section*{Assignment: Exploring Unix Commands}
\paragraph{}The commands used in this assignment: \texttt{cat}, \texttt{cd}, \texttt{ls}, \texttt{mkdir}, \texttt{mv}, \texttt{pwd}, \texttt{rm}, \texttt{rmdir}, \texttt{tree}. With practice, you'll remember what these commands do and when to use them, which will be important for subsequent assignments.

\begin{enumerate}
    \item Log in to GL.
    \item Change directory to your \texttt{hw2} directory, and make sure you are there.
    \begin{verbatim}
        [rzak1@linux1 ~]$ cd ~/cs104/hw2
        [rzak1@linux1 hw2]$ pwd
        /afs/umbc.edu/users/r/z/rzak1/home/cs104/hw2
        [rzak1@linux1 hw2]$\end{verbatim}
    \item Create a new directory called ``Personal''. Verify that the directory exists.
    \begin{verbatim}
        [rzak1@linux1 hw2]$ mkdir Personal
        [rzak1@linux1 hw2]$ ls
        Personal
        [rzak1@linux1 hw2]$\end{verbatim}
    \item Move to the ``Personal'' directory. Verify that you are there.
    \begin{verbatim}
        [rzak1@linux1 hw2]$ cd Personal
        [rzak1@linux1 Personal]$ pwd
        /afs/umbc.edu/users/r/z/rzak1/home/cs104/hw2/Personal
        [rzak1@linux1 Personal]$\end{verbatim}
    \item Use the \texttt{nano} text editor to create a file called ``things2do.txt''.
    \begin{verbatim}
        [rzak1@linux1 Personal]$ nano things2do.txt\end{verbatim}
    \item Once you've opened the file, type in the following:
    \begin{verbatim}
        1. Finish homework 2.
        2. Do a few push-ups
        3. Figure out where my pet tarantula went (maybe this should be #1)\end{verbatim}
    \item Save the file and exit nano.
    \item Move back to the homework 2 directory. Verify that you are there.
    \begin{verbatim}
        [rzak1@linux1 Personal]$ cd ..
        [rzak1@linux1 hw2]$ pwd
        /afs/umbc.edu/users/r/z/rzak1/home/cs104/hw2
        [rzak1@linux1 hw2]$\end{verbatim}
    \item Start a transcript of your Unix session. \textbf{Do not} skip this step!
    \begin{verbatim}
        [rzak1@linux1 hw2]$ script
        Script started, file is typescript
        [rzak1@linux1 hw2]$\end{verbatim}
    \item List the contents of the homework 2 directory. It should contain the ``Personal'' subdirectory and a placeholder for the transcript you've just started.
    \begin{verbatim}
        [rzak1@linux1 hw2]$ ls
        Personal typescript
        [rzak1@linux1 hw2]$\end{verbatim}
    \item List the contents of ``Personal'', which should only contain the file ``things2do.txt''.
    \begin{verbatim}
        [rzak1@linux1 hw2]$ ls Personal
        things2do.txt
        [rzak1@linux1 hw2]$\end{verbatim}
    \item Display the contents of ``things2do.txt''
    \begin{verbatim}
        [rzak1@linux1 hw2]$ cat Personal/things2do.txt
        NOTE: The contents of your file should display here.
        [rzak1@linux1 hw2]$\end{verbatim}
    \item Create a new subdirectory called ``PersonalBackup''. Verify that it exists.
    \begin{verbatim}
        [rzak1@linux1 hw2]$ pwd
        /afs/umbc.edu/users/r/z/rzak1/home/cs104/hw2
        [rzak1@linux1 hw2]$ mkdir PersonalBackup
        [rzak1@linux1 hw2]$ ls
        Personal  PersonalBackup  typescript
        [rzak1@linux1 hw2]$\end{verbatim}
    \item Copy the file ``things2do.txt'' from ``Personal'' to ``PersonalBackup''. \\
    \textit{Make sure you have the slash at the end, or the computer will think you're trying to copy the file to a new file called PersonalBackup, and not into the directory called PersonalBackup}.
    \begin{verbatim}
        [rzak1@linux1 hw2]$ cp Personal/things2do.txt PersonalBackup/
        [rzak1@linux1 hw2]$\end{verbatim}
    \item Look at the contents of ``PersonalBackup'', it should contain ``things2do.txt''
    \begin{verbatim}
        [rzak1@linux1 hw2]$ ls PersonalBackup
        things2do.txt
        [rzak1@linux1 hw2]$\end{verbatim}
    \item Try to delete the subdirectory ``Personal''. Notice that this won't work right away.
    \begin{enumerate}
        \item Delete ``things2do.txt'' from ``Personal''
        \item List the contents of ``Personal'' to make sure it's empty.
        \item Delete ``Personal''.
        \item List the contents of the directory to make sure ``Personal'' has been deleted.
    \end{enumerate}
    \begin{verbatim}
        [rzak1@linux1 hw2]$ rmdir Personal
        rmdir: `Personal/': Directory not empty
        [rzak1@linux1 hw2]$ rm Personal/things2do.txt
        rm: remove regular file `Personal/things2do.txt'? y
        [rzak1@linux1 hw2]$ ls Personal
        [rzak1@linux1 hw2]$ rmdir Personal
        [rzak1@linux1 hw2]$ ls
        PersonalBackup  typescript
        [rzak1@linux1 hw2]$\end{verbatim}
    \item Move ``things2do.txt'' from ``PersonalBackup'' to your current working directory. \\
    \textit{Note the period at the end of the command, which refers to your current directory.}
    \begin{verbatim}
        [rzak1@linux1 hw2]$ mv PersonalBackup/things2do.txt .
        [rzak1@linux1 hw2]$\end{verbatim}
    \item Look at the contents of your ``hw2'' directory to be sure that ``things2do.txt'' is there.
    \begin{verbatim}
        [rzak1@linux1 hw2]$ ls
        PersonalBackup  things2do.txt  typescript
        [rzak1@linux1 hw2]$\end{verbatim}
    \item Look at the contents of ``PersonalBackup'' to be sure that ``things2do.txt'' is not there.
    \begin{verbatim}
        [rzak1@linux1 hw2]$ ls PersonalBackup
        [rzak1@linux1 hw2]$\end{verbatim}
    \item Delete ``PersonalBackup``
    \begin{verbatim}
        [rzak1@linux1 hw2]$ rmdir PersonalBackup
        [rzak1@linux1 hw2]$\end{verbatim}
    \item Move back to your ``cs104'' directory and show all files and directories that you have created.
    \begin{verbatim}
        [rzak1@linux1 hw2]$ cd ..
        [rzak1@linux1 ~/cs104]$ tree
        NOTE: You should see all your classwork and homework directories, 
        along with the files you created for this assignment, in a tree-like structure.
        [rzak1@linux1 ~/cs104]$\end{verbatim}
    \item Now you have to stop the recording of the transcript of your Unix session.
    \begin{verbatim}
        [rzak1@linux1 ~/cs104]$ exit
        exit
        Script done, file is typescript
        [rzak1@linux1 hw2]$\end{verbatim}
    \item Check that you have a file named ``typescript''.
    \begin{verbatim}
        [rzak1@linux1 hw2]$ ls
        things2do.txt  typescript
        [rzak1@linux1 hw2]$\end{verbatim}
    \item Check that the file is not empty.
    \begin{verbatim}
        [rzak1@linux1 hw2]$ cat typescript
        NOTE: The contents of your file should display here.
        [rzak1@linux1 hw2]$\end{verbatim}
    \textit{Optionally, check the size of the file using the ``ls -l'' command:}
    \begin{verbatim}
        [rzak1@linux1 hw2]$ ls -l typescript
        -rw-r--r-- 1 rzak1 rzak 160 Jul  5 13:48 typescript
        [rzak1@linux1 hw2]$\end{verbatim}
    \item Submit your ``things2do.txt'' and ``typescript'' files.
    \begin{verbatim}
        [rzak1@linux1 hw2]$ submit cmsc104_rzak1 hw2 things2do.txt typescript\end{verbatim}
    \item Double-check that your files were submitted using the \texttt{submitls} command:
    \begin{verbatim}
        [rzak1@linux1 hw2]$ submitls cmsc104_rzak1 hw2
        drwxr-xr-x  2 rzak1 rpc 2048 Sep  1 21:51 .
        drwxr-xr-x 23 rzak1 rpc 2048 Sep  1 21:49 ..
        -rw-r--r--  1 rzak1 rpc   54 Sep 15 21:48 things2do.txt
        -rw-r--r--  1 rzak1 rpc  160 Sep 15 21:51 typescript\end{verbatim}
    \item Disconnect from GL.
    \begin{verbatim}
        [rzak1@linux1 hw2]$ exit\end{verbatim}
\end{enumerate}

\section*{Grading Rubric}
\begin{itemize}
    \item things2do.txt is complete: 50 points
    \item typescript is complete and not garbled: 20 points
    \item tree in typescript shows proper directory structure: 30 points
\end{itemize}

\section*{What to Submit}
\paragraph{}You should have already submitted the two necessary files (things2do.txt and typescript) by following the instructions above.

\end{document}