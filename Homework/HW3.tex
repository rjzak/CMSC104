% Copyright 2002-2023 The University of Maryland Baltimore County (UMBC)
% 1000 Hilltop Circle, Baltimore, Maryland, 21250, USA
% https://www.csee.umbc.edu/

\documentclass[letter,11pt]{article}
\usepackage[breaklinks]{hyperref}
\hypersetup{
    bookmarks=true,         % show bookmarks bar?
    unicode=false,          % non-Latin characters in Acrobat’s bookmarks
    pdftoolbar=true,        % show Acrobat’s toolbar?
    pdfmenubar=true,        % show Acrobat’s menu?
    pdffitwindow=false,     % window fit to page when opened
    pdfstartview={XYZ null null 1.00},    % disable zoom
    pdftitle={Homework 3},    % title
    pdfauthor={Richard Zak},     % author
    pdfsubject={UMBC CMSC104 Problem Solving and Computer Programming},   % subject of the document
    pdfkeywords={Computer Science, Programming, Problem Solving, CSEE}, % list of keywords
    pdfnewwindow=true,      % links in new PDF window
    colorlinks=false,       % false: boxed links; true: colored links
    linkcolor=red,          % color of internal links (change box color with linkbordercolor)
    citecolor=green,        % color of links to bibliography
    filecolor=magenta,      % color of file links
    urlcolor=cyan           % color of external links
}
\usepackage{graphicx}
\usepackage{fancyhdr}
\usepackage{multicol}
\pagestyle{fancy}
\usepackage[letterpaper, margin=1in]{geometry}
\geometry{letterpaper}
\usepackage{parskip} % Disable initial indent
\usepackage{color,soul} % Highligher
\usepackage[normalem]{ulem} % Strikethrough with \sout{}

\usepackage[utf8]{inputenc}
\fancyhf{}
\renewcommand{\headrulewidth}{0pt} % Remove default underline from header package
\rhead{CMSC 104 Section 01: Homework 3}
%\rhead{}
\lhead{\begin{picture}(0,0) \put(0,-10){\includegraphics[width=1.1cm]{Images/UMBC-vertical}} \end{picture}}
\cfoot{\thepage}
\rfoot{\input{semester}}
\lfoot{CMSC 104 Section 01}
\AtEndDocument{\vfill \footnotesize{Last modified: 08 February 2023}}
\AtEndDocument{\rfoot{\input{semester}}}
\renewcommand\thesubsection{\arabic{subsection}} % Show only subsection numbers, not section.subsection

\begin{document}

\huge
\textbf{Homework 3: Starting to Code}
\normalsize
\\ ~~ \\
\textbf{Assigned: Tuesday 21 February} \\
\textbf{Due Date: Monday 27 February}

\section*{Objectives}
\paragraph{}Learn how to write, compile, and run a basic C program.

\paragraph{}NOTE: I have used my own username and home directory in the examples. While you are entering the commands, be sure the words before the \texttt{]} on your screen match what’s shown in the examples, as that’ll ensure you are in the right place to execute each command.

\paragraph{}NOTE: At the end of the lab, you will use \texttt{submit} to turn in a transcript of your Linux session. If you do not finish all the steps, just submit as much as you get done.

\section*{Assignment: Our First C Program}
\paragraph{}When programmers learn a new language, a common thing to start off with is to write what is called a ``Hello World'' program. This is the most basic program that can be written that demonstrates how to code in the language. For our first C program, we will modify the traditional ``Hello World'' program to instead print out ``I will not surf the web in class'' 20 times.

\begin{enumerate}
    \item Login to GL and change directory to \texttt{hw3} (\texttt{cd cs104/hw3}).
    \item Use \texttt{nano} to create and edit a file called \texttt{surf.c}. Note that the filenames of C programs must end with \texttt{.c}.
    \item Start with the following code:
    \begin{verbatim}
#include <stdio.h>

int main() {
    printf("Hello World!\n");
    return 0;
}
    \end{verbatim}
    \item Change the text so that instead of ``Hello World!'', it says ``I will not surf the web in class!''.
    \item Use the cut-and-paste feature of \texttt{nano} (\texttt{Ctrl-K}) to cut the line which prints out the statement mentioned previously, and paste it back (\texttt{Ctrl-U}) twenty times. \\
    \textit{NOTE: You must place your cursor at the beginning of the line you want to cut to get the whole line, then place it at the beginning of the line you want to insert it back in.}
    \item Save the file and exit \texttt{nano}.
    \item Start a script session, using the \texttt{script} command.
    \begin{verbatim}
[rzak1@linux1 hw3]$ script
Script started, file is typescript
[rzak1@linux1 hw3]$
    \end{verbatim}
    \item Check that \texttt{surf.c} is not empty.
    \begin{verbatim}
[rzak1@linux1 hw3]$ cat surf.c
NOTE: The contents of your file should display here.
[rzak1@linux1 hw3]$
    \end{verbatim}
    \item Compile your program.
    \begin{verbatim}
[rzak1@linux1 hw3]$ gcc -Wall surf.c
    \end{verbatim}
    \item Check that you have a file called \texttt{a.out}
    \begin{verbatim}
[rzak1@linux1 hw3]$ ls
a.out surf.c typescript
[rzak1@linux1 hw3]$
    \end{verbatim}
    \item Run your program
    \begin{verbatim}
[rzak1@linux1 hw3]$ ./a.out
    \end{verbatim}
    \item Quit your \texttt{script} session.
    \begin{verbatim}
[rzak1@linux1 hw3]$ exit
exit
Script done, file is typescript
[rzak1@linux1 hw3]$
    \end{verbatim}
    \item Look at your \texttt{typescript} file to make sure it is correct.
    \begin{verbatim}
[rzak1@linux1 hw3]$ cat typescript
NOTE: The contents of your file should display here.
[rzak1@linux1 hw3]$ 
    \end{verbatim}
    \item Submit your C program and the typescript file.
    \begin{verbatim}
[rzak1@linux1 hw03]$ submit cmsc104_rzak1 hw3 surf.c typescript
    \end{verbatim}
    \item Verify that you submitted two files by using the \texttt{submitls} command.
    \begin{verbatim}
[rzak1@linux1 hw03]$ submitls cmsc104_rzak1 hw3
    \end{verbatim}
\end{enumerate}

\section*{Grading Rubric}
\begin{itemize}
    \item surf.c compiles: 40 points
    \item surf.c prints correctly: 50 points
    \item typescript is complete and not garbled: 10 points
\end{itemize}

\section*{What to Submit}
\paragraph{}You should have already submitted the necessary files (surf.c and typescript) by following the instructions above.

\end{document}