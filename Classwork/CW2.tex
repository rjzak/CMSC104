\documentclass[letter,11pt]{article}
\usepackage[breaklinks]{hyperref}
\hypersetup{
    bookmarks=true,         % show bookmarks bar?
    unicode=false,          % non-Latin characters in Acrobat’s bookmarks
    pdftoolbar=true,        % show Acrobat’s toolbar?
    pdfmenubar=true,        % show Acrobat’s menu?
    pdffitwindow=false,     % window fit to page when opened
    pdfstartview={XYZ null null 1.00},    % disable zoom
    pdftitle={Classwork 1},    % title
    pdfauthor={Richard Zak},     % author
    pdfsubject={UMBC CMSC104 Problem Solving and Computer Programming},   % subject of the document
    pdfkeywords={Computer Science, Programming, Problem Solving, CSEE}, % list of keywords
    pdfnewwindow=true,      % links in new PDF window
    colorlinks=false,       % false: boxed links; true: colored links
    linkcolor=red,          % color of internal links (change box color with linkbordercolor)
    citecolor=green,        % color of links to bibliography
    filecolor=magenta,      % color of file links
    urlcolor=cyan           % color of external links
}
\usepackage{graphicx}
\usepackage{fancyhdr}
\usepackage{multicol}
\pagestyle{fancy}
\usepackage[letterpaper, margin=1in]{geometry}
\geometry{letterpaper}
\usepackage{parskip} % Disable initial indent
\usepackage{color,soul} % Highligher
\usepackage[normalem]{ulem} % Strikethrough with \sout{}

\usepackage[utf8]{inputenc}
\fancyhf{}
\renewcommand{\headrulewidth}{0pt} % Remove default underline from header package
\rhead{CMSC 104 Section 01: Classwork 2}
%\rhead{}
\lhead{\begin{picture}(0,0) \put(0,-10){\includegraphics[width=1.1cm]{Images/UMBC-vertical}} \end{picture}}
\cfoot{\thepage}
\rfoot{\input{semester}}
\lfoot{CMSC 104 Section 01}
\AtEndDocument{\vfill \footnotesize{Last modified: 01 February 2023}}
\AtEndDocument{\rfoot{\input{semester}}}
\renewcommand\thesubsection{\arabic{subsection}} % Show only subsection numbers, not section.subsection

\begin{document}

\huge
\textbf{Classwork 2: Organizing for the Semester}.
\normalsize
\\ ~~ \\
\textbf{In-class Date: Thursday 02 February} \\
\textbf{Due Date: Wednesday 08 February}

\section*{Objectives}
\paragraph{}To setup assignment directories for the semester, become more familiar with the Linux OS, understand how to use basic Unix commands, and reinforce how to submit assignments.
\paragraph{}NOTE: I have used my own username and home directory in the examples. While you are entering the commands, be sure the words before the ] on your screen match what’s shown in the examples, as that’ll ensure you are in the right place to execute each command.
\paragraph{}NOTE: Unless otherwise noted, be sure to hit Enter (PC) or Return (Mac) after every command. At the end of the lab, you will use \texttt{submit} to turn in a transcript of your Linux session. If you do not finish all the steps, just submit as much as you get done.

\section*{Assignment: Semester Prep / Exploring Unix Commands}
\paragraph{}Commands you'll use: \texttt{pwd}, \texttt{ls}, \texttt{mkdir}, \texttt{cd}, \texttt{cat}, \texttt{tree}, \texttt{mv}. Follow the steps below in order. Use the PDFs on Blackboard under Course Materials as a reference.

\begin{enumerate}
    \item Log in to your GL account.
    \item Look at the name of your home directory:
    \begin{verbatim}
        [rzak1@linux1 ~]$ pwd
        /afs/umbc.edu/users/r/z/rzak1/home
        [rzak1@linux ~]$    
    \end{verbatim}
    \item Look at the contents of your home directory. It might contain the following files and subdirectories:
    \begin{verbatim}
        [rzak1@linux1 ~]$ ls
        Mail  mybio.txt
        [rzak1@linux1 ~]$
    \end{verbatim}
    \item Look at the ``long'' contents of your home directory. You should see the same subdirectories as the previous step, but with more information about each.
    \begin{verbatim}
        [rzak1@linux1 ~]$ ls -l
        total 2
        drwx------    2 rzak1 rpc  2048 Aug 27 09:04 Mail
        -rw-r--r--    1 rzak1 rpc  1024 Aug 27 09:04 mybio.txt
        [rzak1@linux1 ~]$
    \end{verbatim}
    \item Look at all files in the directory, including files whose file names start with ., which indicates a ``hidden file.'' This listing is similar to step 4 but also includes . for the current directory, .. for the parent/previous directory, and the previously mentioned ``hidden files.''
    \begin{verbatim}
        [rzak1@linux1 ~]$ ls -a
        NOTE: There are too many files to show here, and their names when I run the
        command won’t match what you see when you run it, so just know what this
        option does.
        [rzak1@linux1 ~]$
    \end{verbatim}
    \item Create a subdirectory called ``cs104''. Verify that it has been created by again looking at the contents of your home directory.
    \begin{verbatim}
        [rzak1@linux1 ~]$ mkdir cs104
        [rzak1@linux1 ~]$ ls
        cs104  Mail mybio.txt
        [rzak1@linux1 ~]$
    \end{verbatim}
    \item Move (or, ``change directory'') to the cs104 directory. Verify that your current directory is now indeed cs104. \\ NOTE: Remember that Linux is case sensitive; cs104 is different from CS104.
    \begin{verbatim}
        [rzak1@linux1 ~]$ cd cs104
        [rzak1@linux1 ~/cs104]$ pwd
        /afs/umbc.edu/users/r/z/rzak1/home/cs104
        [rzak1@linux1 ~/cs104]$    
    \end{verbatim}
    \item Create new subdirectories for all classwork (cw1-cw10) and homework (hw1-hw10) assignments. Verify that each of the new subdirectories has been created. \\ NOTE: Will not give all commands because it’s basically copy-pasting with a different directory name each time. Remember, the up arrow brings back the last command you ran.
    \begin{verbatim}
        [rzak1@linux1 ~/cs104]$ mkdir cw1
        [rzak1@linux1 ~/cs104]$ mkdir hw1
        [rzak1@linux1 ~/cs104]$ mkdir cw2
        [rzak1@linux1 ~/cs104]$ mkdir hw2
        ....
        [rzak1@linux1 ~/cs104]$ ls
        cw1 cw3 cw5 cw7 cw9  hw1 hw3 hw5 hw7 hw9
        cw2 cw4 cw6 cw8 cw10 hw2 hw4 hw6 hw8 hw10
        [rzak1@linux1 ~/cs104]$ 
    \end{verbatim}
    \item Change directory to cw2 and make sure you are there.
    \begin{verbatim}
        [rzak1@linux1 ~/cs104]$ cd cw2
        [rzak1@linux1 cw2]$ pwd
        /afs/umbc.edu/users/r/z/rzak1/home/cs104/cw2
        [rzak1@linux1 cw2]$ 
    \end{verbatim}
    \item Use the \texttt{nano} editor to create a file called ``things2do.txt.''
    \begin{verbatim}
        [rzak1@linux1 cw2]$ nano things2do.txt
    \end{verbatim}
    \item Once you have opened the file, you should type the following:
    \begin{verbatim}
        1. Finish Classwork 1
        2. Reach the first 3 chapters from the book.
        3. Bring my pet tarantula to CMSC104 for Show'n'Tell
    \end{verbatim}
    Save the file and exit \texttt{nano} (CTRL-X, Y)
    \item Start a transcript of your Unix session. \underline{It's very important that you don't skip this step!}
    \begin{verbatim}
        [rzak1@linux1 cw2]$ script
        Script started, file is typescript
        [rzak1@linux1 cw2]$ 
    \end{verbatim}
    \item Move back to the cs104 directory and make sure you are there.
    \begin{verbatim}
        [rzak1@linux1 cw2]$ cd ..
        [rzak1@linux1 ~/cs104] pwd
        /afs/umbc.edu/users/r/z/rzak1/home/cs104
    \end{verbatim}
    \item Move the ``mybio.txt'' file in your home directory to the ``cw1'' directory. Check to see that it is no longer in your home directory. Then, check to see that it is in the cw1 directory. \\
    NOTE: The trailing / after cw1 in the second command is important because it tells the computer that cw1 is a directory and not a file, so put the first file inside it. If you forget the trailing /, the computer will think you are trying to copy mybio.txt into a new file in the cs104 directory called cw1, which is not allowed because cw1 already exists as a directory, not a file.
    \begin{verbatim}
        [rzak1@linux1 ~/cs104]$ mv ../mybio.txt cw1/
        [rzak1@linux1 ~/cs104]$ ls ..
        cs104  Mail
        [rzak1@linux1 ~/cs104]$ ls cw1
        mybio.txt
        [rzak1@linux1 ~/cs104]$
    \end{verbatim}
    \item Now show all files and directories that you have created.
    \begin{verbatim}
        [rzak1@linux1 ~/cs104]$ tree
        NOTE: You should see all your classwork and homework directories, with
        all appropriate Classwork 1 files in the cw1 directory, in a tree-like
        structure.
        [rzak1@linux1 ~/cs104]$
    \end{verbatim}
    \item Now you have to stop recording the transcript of your Unix session.
    \begin{verbatim}
        [rzak1@linux1 ~/cs104]$ exit
        exit
        Script done, file is typescript
        [rzak1@linux1 cw2]$
    \end{verbatim}
    \item Check that you have a file named ``typescript''.
    \begin{verbatim}
        [rzak1@linux1 cw2]$ ls
        things2do.txt typescript
        [rzak1@linux1 cw2]$
    \end{verbatim}
    \item Check that the file is not empty.
    \begin{verbatim}
        [rzak1@linux1 cw2]$ cat typescript
        NOTE: The contents of your file (the commands you’ve
        just executed and their results) should display here.
        [rzak1@linux1 cw2]$
    \end{verbatim}
    \item Submit your ``things2do.txt'' and ``typescript'' files.
    \begin{verbatim}
        [rzak1@linux1 cw2]$ submit cmsc104_rzak1 cw2 things2do.txt typescript
        [rzak1@linux1 cw2]$
    \end{verbatim}
    \item Double-check that the files were, in fact, submitted.
    \begin{verbatim}
        [rzak1@linux1 cw2]$ submitls cmsc104_rzak1 cw2
    \end{verbatim}
    \item Don’t forget to logout.
    \begin{verbatim}
        [rzak1@linux1 cw2]$ logout
    \end{verbatim}
\end{enumerate}

\section*{Grading Rubric}
\begin{itemize}
    \item ``things2do.txt'' is complete: 50 points
    \item typescript is complete and not garbled: 20 points
    \item \texttt{tree} in typescript shows proper directory structure: 30 points
\end{itemize}

\section*{What to Submit}
\paragraph{}You should have already submitted the two files, ``things2do.txt'' \& ``typescript'' by following the instructions above.

\end{document}