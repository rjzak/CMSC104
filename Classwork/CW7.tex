\documentclass[letter,11pt]{article}
\usepackage[breaklinks]{hyperref}
\hypersetup{
    bookmarks=true,         % show bookmarks bar?
    unicode=false,          % non-Latin characters in Acrobat’s bookmarks
    pdftoolbar=true,        % show Acrobat’s toolbar?
    pdfmenubar=true,        % show Acrobat’s menu?
    pdffitwindow=false,     % window fit to page when opened
    pdfstartview={XYZ null null 1.00},    % disable zoom
    pdftitle={Classwork 7},    % title
    pdfauthor={Richard Zak},     % author
    pdfsubject={UMBC CMSC104 Problem Solving and Computer Programming},   % subject of the document
    pdfkeywords={Computer Science, Programming, Problem Solving, CSEE}, % list of keywords
    pdfnewwindow=true,      % links in new PDF window
    colorlinks=false,       % false: boxed links; true: colored links
    linkcolor=red,          % color of internal links (change box color with linkbordercolor)
    citecolor=green,        % color of links to bibliography
    filecolor=magenta,      % color of file links
    urlcolor=cyan           % color of external links
}
\usepackage{graphicx}
\usepackage{fancyhdr}
\usepackage{multicol}
\pagestyle{fancy}
\usepackage[letterpaper, margin=1in]{geometry}
\geometry{letterpaper}
\usepackage{parskip} % Disable initial indent
\usepackage{color,soul} % Highligher
\usepackage[normalem]{ulem} % Strikethrough with \sout{}
\usepackage{attachfile} % attach files

\usepackage[utf8]{inputenc}
\fancyhf{}
\renewcommand{\headrulewidth}{0pt} % Remove default underline from header package
\rhead{CMSC 104 Section 01: Classwork 7}
%\rhead{}
\lhead{\begin{picture}(0,0) \put(0,-10){\includegraphics[width=1.1cm]{Images/UMBC-vertical}} \end{picture}}
\cfoot{\thepage}
\rfoot{\input{semester}}
\lfoot{CMSC 104 Section 01}
\AtEndDocument{\vfill \footnotesize{Last modified: 29 August 2022}}
\AtEndDocument{\rfoot{\input{semester}}}
\renewcommand\thesubsection{\arabic{subsection}} % Show only subsection numbers, not section.subsection

\begin{document}

\huge
\textbf{Classwork 7: Bar Graphs}.
\normalsize
\\ ~~ \\
\textbf{In-class Date: Tuesday 01 November} \\
\textbf{Due Date: Monday 07 November}

\section*{Objectives}
\paragraph{}To gain more practice with nested \texttt{while} loops.

\section*{Assignment}
\paragraph{}For this assignment, you will write a program which prints horizontal ``bar graphs'' using numbers entered by the user.

\subsection*{Example Compilation and Execution}
\begin{verbatim}
[rzak1@linux1 cw7]$ gcc -Wall bar.c
[rzak1@linux1 cw7]$ ./a.out
Enter a positive number (type 'quit' to end): 3
***
Enter a positive number (type 'quit' to end): 17
*****************
Enter a positive number (type 'quit' to end): 23
***********************
Enter a positive number (type 'quit' to end): 22
**********************
Enter a positive number (type 'quit' to end): 9
*********
Enter a positive number (type 'quit' to end): 4
****
Enter a positive number (type 'quit' to end): quit
[rzak1@linux1 cw7]$ ./a.out
Enter a positive number (type 'quit' to end): quit
No numbers were entered
[rzak1@linux1 cw7]$
\end{verbatim}

\paragraph{}The ``bar'' in this bar graph are the asterisks (*) printed after each user entry. The number of asterisks must be equal to the number entered by the user.

\subsection*{Starter Code}
\paragraph{}Use this code to help you get started.
\begin{verbatim}
/**************************************
** File:    bar.c
** Author:   <myName>
** Date:    <todaysDate>
** Section:  02-LEC (1068)
** E-mail:   <myEmailAddress>
**
** This file contains the main program for Classwork 7.
** TODO: <Explain what the assignment is asking you to do here.>
**************************************/

#include <stdio.h>

int main() {
   // Variable declarations
   int num;                 /* user's positive number entry */
   int count = 0;           /* how many positive numbers user has entered */
   int stop = 0;            /* don't stop */
   int matched;             /* used like flag to see if user entered 'quit' */
   int innerLoopCounter;    /* how many iterations to print * */

   // TODO: Create outer loop for user input checking

      // TODO: Prompt user for positive number

      // TODO: If user typed 'quit', set stop to 1 so we can break out of outer loop;
      // else if user entered positive number, create inner loop for printing and 
      //      updating count;
      // else notify the user they entered an invalid number


   // TODO: Let user know if no positive numbers were entered


   return 0;
}
\end{verbatim}

\subsection*{Notes}
\begin{itemize}
    \item Remember: Update the header comments with your information.
    \item You MUST use nested while loops for this assignment. The outer while loop should iterate once for each number entered by the user. The inner loop should print out the right number of asterisks.
    \item To print the asterisks, use a new variable and a while loop that counts up to the number entered by the user. (Each iteration of this inner while loop adds one to the counter.) For each iteration of the inner while loop, print out a single asterisk without a newline. After the loop ends, your program should print out a newline character. This way, the asterisks appear on one line.
    \item Use ``min.c'' on Blackboard as an optional reference which shows how the outer loop could be written.
    \item Homework 7 will be a continuation of this assignment. If you are done early, you can continue to Homework 7 (when posted), but be sure to submit the appropriate files to the correct assignment!
\end{itemize}

\section*{Grading Rubric}
\begin{itemize}
    \item Header comment: 2 points
    \item Body comments: 3 points
    \item Compiles: 40 points
    \item Displays ``No numbers entered'' appropriately: 10 points
    \item Two while loops: 25 points
    \item Correct number of asterisks printed: 20 points
\end{itemize}

\section*{What to Submit}
\paragraph{}Use the \texttt{script} command to record yourself compiling and running your program several times with different numbers. (Do not record yourself editing your program!) Exit from script, submit your program and the typescript file.
\begin{verbatim}
[rzak1@linux1 cw7]$ submit cmsc104_rzak1 cw7 bar.c typescript
\end{verbatim}

\subsection*{Verify}
\paragraph{}Make sure you submitted the assignment correctly.
\begin{verbatim}
[rzak1@linux1 cw7]$ submitls cmsc104_rzak1 cw7
\end{verbatim}

\end{document}