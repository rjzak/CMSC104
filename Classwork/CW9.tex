\documentclass[letter,11pt]{article}
\usepackage[breaklinks]{hyperref}
\hypersetup{
    bookmarks=true,         % show bookmarks bar?
    unicode=false,          % non-Latin characters in Acrobat’s bookmarks
    pdftoolbar=true,        % show Acrobat’s toolbar?
    pdfmenubar=true,        % show Acrobat’s menu?
    pdffitwindow=false,     % window fit to page when opened
    pdfstartview={XYZ null null 1.00},    % disable zoom
    pdftitle={Classwork 9},    % title
    pdfauthor={Richard Zak},     % author
    pdfsubject={UMBC CMSC104 Problem Solving and Computer Programming},   % subject of the document
    pdfkeywords={Computer Science, Programming, Problem Solving, CSEE}, % list of keywords
    pdfnewwindow=true,      % links in new PDF window
    colorlinks=false,       % false: boxed links; true: colored links
    linkcolor=red,          % color of internal links (change box color with linkbordercolor)
    citecolor=green,        % color of links to bibliography
    filecolor=magenta,      % color of file links
    urlcolor=cyan           % color of external links
}
\usepackage{graphicx}
\usepackage{fancyhdr}
\usepackage{multicol}
\pagestyle{fancy}
\usepackage[letterpaper, margin=1in]{geometry}
\geometry{letterpaper}
\usepackage{parskip} % Disable initial indent
\usepackage{color,soul} % Highligher
\usepackage[normalem]{ulem} % Strikethrough with \sout{}

\usepackage[utf8]{inputenc}
\fancyhf{}
\renewcommand{\headrulewidth}{0pt} % Remove default underline from header package
\rhead{CMSC 104 Section 01: Classwork 9}
%\rhead{}
\lhead{\begin{picture}(0,0) \put(0,-10){\includegraphics[width=1.1cm]{Images/UMBC-vertical}} \end{picture}}
\cfoot{\thepage}
\rfoot{\input{semester}}
\lfoot{CMSC 104 Section 01}
\AtEndDocument{\vfill \footnotesize{Last modified: 16 November 2021}}
\AtEndDocument{\rfoot{\input{semester}}}
\renewcommand\thesubsection{\arabic{subsection}} % Show only subsection numbers, not section.subsection

\begin{document}

\huge
\textbf{Classwork 9: Simple Functions}.
\normalsize
\\ ~~ \\
\textbf{In-class Date: Thursday 18 November}

\section*{Objectives}
\paragraph{}To practice implementing functions and passing arrays.

\section*{Part 1: Counting Change}
\paragraph{}For this part you will implement a simple function, called \texttt{countchange()}, that computes the value of a certain amount of change, given the number of coins of each kind.

\subsection*{Example Compilation and Execution}
\begin{verbatim}
[rzak1@linux1 cw9]$ gcc -Wall countchange.c
[cos1@linux1 cw9]$ ./a.out
This program will compute the worth of your change.

Enter number of quarters you have: 3
Enter number of dimes you have: 2
Enter number of nickels you have: 4
Enter number of pennies you have: 3
Your 3 quarters, 2 dimes, 4 nickels and 3 pennies are worth 118 cents.
[rzak1@linux1 cw9]$ 
\end{verbatim}

\subsection*{Starter Code}
\paragraph{}Use this code to help you get started.
\begin{verbatim}
/*****************************************
** File:  countchange.c
** Author: <myName>
** Date:  <todaysDate>
** Section: 01-LEC (10600)
** E-mail: <myEmailAddress>
**
** This file contains the main program for part 1 of Classwork 9.
** The program asks the user for a total number of 
** quarters, dimes, nickels, and pennies, then
** calculates and displays the total number of cents.
*****************************************/

#include <stdio.h>

/*
   The function prototype is written for you.
   This prototype says that countchange() has four int parameters
   and returns an int.
*/

int countchange(int , int , int , int );


/* Don't change the main() function! */
int main () {
   int cents;
   int quarters, dimes, nickels, pennies;

   printf("This program will compute the worth of your change.\n\n");
   printf("Enter number of quarters you have: ");
   scanf("%d", &quarters);
   printf("Enter number of dimes you have: ");
   scanf("%d", &dimes);
   printf("Enter number of nickels you have: ");
   scanf("%d", &nickels);
   printf("Enter number of pennies you have: ");
   scanf("%d", &pennies);

   cents = countchange(quarters, dimes, nickels, pennies);

   printf("Your %d quarters, %d dimes, %d nickels and %d pennies", 
      quarters, dimes, nickels, pennies);
   printf(" are worth %d cents.\n", cents);

   return 0;
}
/* end of main() function */


/* Function countchange
   
   Computes the worth of the given number of quarters,
   dimes, nickels and pennies.

*/

/* TODO: Implement your function here */
\end{verbatim}

\subsection*{Notes}
\paragraph{}Below the main program, write the C code to implement the \texttt{countchange()} function. Give meaningful names to your parameters. To compute the total number of cents, use 25 for quarters, 10 for dimes, 5 for nickels and 1 for pennies. Be sure to use the += operator in your function.

\section*{Part 2: Making Change}
\paragraph{}For this part you will implement a simple function, called \texttt{makechange()}, that figures out how to make change for a given number of cents.

\subsection*{Example Compilation and Execution}
\begin{verbatim}
[rzak1@linux1 cw9]$ gcc -Wall makechange.c
[rzak1@linux1 cw9]$ ./a.out
This program will figure out the change for you.

Enter number of cents: 68
Make change using 2 quarters, 1 dimes, 1 nickels and 3 pennies.
[rzak1@linux1 cw9]$
\end{verbatim}

\paragraph{}In this example, the function \texttt{makechange()} determined that 2 quarters, 1 dime, 1 nickel and 3 pennies equals 68 cents. (This is the fewest number of coins you can have to make up 68 cents.) The function ``communicates'' this result to the main program using 4 reference parameters.

\subsection*{Starter Code}
\paragraph{}Use this code to help you get started.
\begin{verbatim}
/*****************************************
** File:  makechange.c
** Author: <myName>
** Date:  <todaysDate>
** Section: 01-LEC (10600)
** E-mail: <myEmailAddress>
**
** This file contains the main program for part 2 of Classwork 9.
** The program asks the user for a total number of cents
** and makes change using quarters, dimes, nickels, and pennies.
*****************************************/

#include <stdio.h>

/*
   The function prototype is written for you.
   This prototype says that makechange() has two parameters.
   The first parameter is a normal int. The next
   parameter is an integer array of length 4.
*/

void makechange(int , int []);

/* Don't change the main() function!!! */
int main () {
   int cents;
   int change[4];

   printf("This program will figure out the change for you.\n\n");
   printf("Enter number of cents: ");
   scanf("%d", &cents);

   makechange(cents, change);

   printf("Make change using %d quarters, %d dimes, %d nickels and %d pennies.\n",
      change[0], change[1], change[2], change[3]);

   return 0;
}
/* end of main function */

/* Function makechange
   
   Stores # of quarters, dimes, nickels and pennies
   add up to n cents in the change array.
*/

/* TODO: Implement the function makechange() here: */
\end{verbatim}

\subsection*{Notes}
\paragraph{}Below the main program, write the C code to implement the \texttt{makechange()} function. Give meaningful names to your parameters. Give the reference parameters names that immediately tell you they are array pointers to \texttt{int}.

\paragraph{}To compute the number of quarters, dimes, nickels and pennies, you should use the integer division operator \texttt{/} and the modulus operator \texttt{\%}.

\section*{Grading Rubric}
\begin{itemize}
    \item countchange.c header comment: 2 points
    \item countchange.c body comments: 3 points
    \item countchange.c compiles: 20 points
    \item countchange.c counts change correctly: 25 points
    \item makechange.c header comment: 2 points
    \item makechange.c body comments: 3 points
    \item makechange.c compiles: 20 points
    \item makechange.c makes change correctly: 25 points
\end{itemize}

\section*{What to Submit}
\paragraph{}Use the \texttt{script} command to record yourself compiling and running your programs 3 times, using different numbers each time. (Do not record yourself editing your program!) Exit from \texttt{script}. Submit your programs and the typescript file.

\begin{verbatim}
[rzak1@linux1 cw9]$ submit cmsc104_10600 cw9 countchange.c makechange.c typescript
\end{verbatim}

\subsection*{Verify}
\paragraph{}Make sure you submitted the assignment correctly.
\begin{verbatim}
[rzak1@linux1 cw9]$ submitls cmsc104_10600 cw9
\end{verbatim}

\end{document}