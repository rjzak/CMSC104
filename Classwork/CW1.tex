\documentclass[letter,11pt]{article}
\usepackage[breaklinks]{hyperref}
\hypersetup{
    bookmarks=true,         % show bookmarks bar?
    unicode=false,          % non-Latin characters in Acrobat’s bookmarks
    pdftoolbar=true,        % show Acrobat’s toolbar?
    pdfmenubar=true,        % show Acrobat’s menu?
    pdffitwindow=false,     % window fit to page when opened
    pdfstartview={XYZ null null 1.00},    % disable zoom
    pdftitle={Classwork 1},    % title
    pdfauthor={Richard Zak},     % author
    pdfsubject={UMBC CMSC104 Problem Solving and Computer Programming},   % subject of the document
    pdfkeywords={Computer Science, Programming, Problem Solving, CSEE}, % list of keywords
    pdfnewwindow=true,      % links in new PDF window
    colorlinks=false,       % false: boxed links; true: colored links
    linkcolor=red,          % color of internal links (change box color with linkbordercolor)
    citecolor=green,        % color of links to bibliography
    filecolor=magenta,      % color of file links
    urlcolor=cyan           % color of external links
}
\usepackage{graphicx}
\usepackage{fancyhdr}
\usepackage{multicol}
\pagestyle{fancy}
\usepackage[letterpaper, margin=1in]{geometry}
\geometry{letterpaper}
\usepackage{parskip} % Disable initial indent
\usepackage{color,soul} % Highligher
\usepackage[normalem]{ulem} % Strikethrough with \sout{}

\usepackage[utf8]{inputenc}
\fancyhf{}
\renewcommand{\headrulewidth}{0pt} % Remove default underline from header package
\rhead{CMSC 104 Section 01: Classwork 1}
%\rhead{}
\lhead{\begin{picture}(0,0) \put(0,-10){\includegraphics[width=1.1cm]{Images/UMBC-vertical}} \end{picture}}
\cfoot{\thepage}
\rfoot{\input{semester}}
\lfoot{CMSC 104 Section 01}
\AtEndDocument{\vfill \footnotesize{Last modified: 01 February 2022}}
\AtEndDocument{\rfoot{\input{semester}}}
\renewcommand\thesubsection{\arabic{subsection}} % Show only subsection numbers, not section.subsection

\begin{document}

\huge
\textbf{Classwork 1: Introduction to \texttt{nano} and \texttt{submit}}.
\normalsize
\\ ~~ \\
\textbf{In-class Date: Tuesday 01 February}

\section*{Objectives}
\paragraph{}To become familiar with the nano text editor on the GL system and understand how to submit assignments.

\paragraph{NOTE:} I have used my own username and home directory in the examples. While you are entering the commands, be sure the words before the ] on your screen match what’s shown in the examples, as that’ll ensure you are in the right place to execute each command.

\paragraph{NOTE:} Unless otherwise noted, be sure to hit Enter(PC)or return(Mac) after every command! At the end of the lab, you will use submit to turn in a transcript of your Linux session. If you do not finish all the steps, just submit as much as you get done.

\section*{Assignment: Tell me about yourself / Submit something}
\subsection{Login to the UMBC Linux GL System}
\paragraph{}If you use a Windows machine at home, you may need to install PuTTY first, which can be download from \url{https://www.putty.org}. If you have Windows 10, you can use the Windows Subsystem for Linux to get a native SSH client on your machine. Mac users, use Terminal.app to SSH to UMBC.

\paragraph{}When connecting, make sure you're typing your user name in all lowercase.

\subsection{Use the nano Text Editor to Create an Autobiography}
\paragraph{}You will be using the \texttt{nano} text editor to create a file called ``mybio.txt``. To create your autobiography file, do the following

\begin{enumerate}
    \item Enter the nano editor by typing \texttt{nano mybio.txt}
    \item Simply type in the information specified below. Edit any mistakes using the \textit{Backspace} or \textit{Delete} key to backspace like you would in a normal word processor, like Microsoft Word. When you get to the end of a line, hit the enter(PC) or return(Mac) key at a reasonable spot instead of letting the text wrap around to the next line. Remember that \texttt{nano} is simply a text editor and does not format things nicely for us, nor does it do spell checking.
    \item \underline{Save your work} as you go by pressing \textit{CTRL-O} for ``write out''.
    \item To \underline{exit nano}, press \textit{CTRL-X}. If you have made changes since you last saved, it will ask you if you want to save the file, so press \textit{Y}for yes, then enter(PC) or return(Mac) when prompted to save the file. It is highly advisable NOT to change the filename.
    \item You will know that you have exited nano and are again back at the Linux command line if you see the prompt, which should look something like \verb|[rzak1@linux1 ~]$|
    \item You can check that the file \textit{mybio.txt} is in your directory by typing \texttt{ls}, which is the command to see the files in a directory.
\end{enumerate}

\paragraph{}The file will contain a brief autobiography about you as follows:
\begin{itemize}
    \item Begin by stating your name, major, and class standing (eg. freshman, sophomore, etc).
    \item Write a paragraph or two about yourself, answering the following:
    \begin{itemize}
        \item Where are you from, and why did you choose UMBC?
        \item Why did you pick your major?
        \item What is the best course (college, or otherwise) you've had, and what made it memorable?
        \item What is your favourite colour?
    \end{itemize}
\end{itemize}

\subsection{Submit Your File In the GL System}
\paragraph{}The command to submit the file should look like this, of course, with \underline{your} user name.

\verb|[rzak1@linux1 ~]$ submit cmsc104_1068 cw1 mybio.txt| \\
\verb|Submitting mybio.txt...OK|

\subsection{Check Your Submission}
\verb|[rzak1@linux1 ~]$ submitls cw1 cmsc104_1068| \\
\verb|total 5| \\
\verb|drwxr-xr-x  2 rzak1 rpc 2048 Sep  1 21:51 .| \\
\verb|drwxr-xr-x 23 rzak1 rpc 2048 Sep  1 21:49 ..| \\
\verb|-rw-r--r--  1 rzak1 rpc   54 Sep  1 21:51 mybio.txt|

\subsection{Logout}
\verb|[rzak1@linux1 ~]$ exit| 

or

\verb|[rzak1@linux1 ~]$ logout|

\section*{Grading Rubric}
\begin{itemize}
    \item \textit{mybio.txt} is complete: 100 points
\end{itemize}

\section*{What to Submit}
\paragraph{}You should have already submitted \textit{mybio.txt} by following the instructions above.

\end{document}