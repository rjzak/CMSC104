\documentclass[letter,11pt]{article}
\usepackage[breaklinks]{hyperref}
\hypersetup{
    bookmarks=true,         % show bookmarks bar?
    unicode=false,          % non-Latin characters in Acrobat’s bookmarks
    pdftoolbar=true,        % show Acrobat’s toolbar?
    pdfmenubar=true,        % show Acrobat’s menu?
    pdffitwindow=false,     % window fit to page when opened
    pdfstartview={XYZ null null 1.00},    % disable zoom
    pdftitle={Classwork 8},    % title
    pdfauthor={Richard Zak},     % author
    pdfsubject={UMBC CMSC104 Problem Solving and Computer Programming},   % subject of the document
    pdfkeywords={Computer Science, Programming, Problem Solving, CSEE}, % list of keywords
    pdfnewwindow=true,      % links in new PDF window
    colorlinks=false,       % false: boxed links; true: colored links
    linkcolor=red,          % color of internal links (change box color with linkbordercolor)
    citecolor=green,        % color of links to bibliography
    filecolor=magenta,      % color of file links
    urlcolor=cyan           % color of external links
}
\usepackage{graphicx}
\usepackage{fancyhdr}
\usepackage{multicol}
\pagestyle{fancy}
\usepackage[letterpaper, margin=1in]{geometry}
\geometry{letterpaper}
\usepackage{parskip} % Disable initial indent
\usepackage{color,soul} % Highligher
\usepackage[normalem]{ulem} % Strikethrough with \sout{}

\usepackage[utf8]{inputenc}
\fancyhf{}
\renewcommand{\headrulewidth}{0pt} % Remove default underline from header package
\rhead{CMSC 104 Section 01: Classwork 8}
%\rhead{}
\lhead{\begin{picture}(0,0) \put(0,-10){\includegraphics[width=1.1cm]{Images/UMBC-vertical}} \end{picture}}
\cfoot{\thepage}
\rfoot{\input{semester}}
\lfoot{CMSC 104 Section 01}
\AtEndDocument{\vfill \footnotesize{Last modified: 11 November 2021}}
\AtEndDocument{\rfoot{\input{semester}}}
\renewcommand\thesubsection{\arabic{subsection}} % Show only subsection numbers, not section.subsection

\begin{document}

\huge
\textbf{Classwork 8: Class Grades Simulator}.
\normalsize
\\ ~~ \\
\textbf{In-class Date: Thursday 11 November}

\section*{Objectives}
\paragraph{}Practice writing a program that uses a \texttt{switch} statement, the \texttt{+=} assignment operator, and a texttt{for} loop.

\section*{Assignment}
\paragraph{}Write a program to simulate class letter grades (A, B, C, D, or F) and output the count of each letter grade to the screen. To do this, you will utilize a for loop that calculates a random letter grade each time through the loop, then utilizes a switch statement to update the appropriate counter variable using the += assignment operator. Make sure the default case warns the user of invalid letter grade. Outside the loop, you will print the final results.

\subsection*{Example Compilation and Execution}
\begin{verbatim}
[rzak1@linux1 cw8]$ gcc -Wall grades.c
[rzak1@linux1 cw8]$ ./a.out
Out of 36 students, here is the class grade breakdown:
A: 10
B: 8
C: 10
D: 3
F: 5
[rzak1@linux1 cw8]$ 
\end{verbatim}

\subsection*{Notes}
\paragraph{}Use the classwork starting point provided on Blackboard and name your program \texttt{grades.c}.  
\paragraph{}Notice how the \texttt{for} loop in the program is provided for you, it terminates after the correct number of grades have been simulated, and for the basic assignment, there is no user input.

\paragraph{}Notice how the starting point file uses \texttt{\#define} to define a constant \texttt{TOTAL\_GRADES} that holds the number of grades to simulate. That way, if you want/need to change the total number, you only have to change it in one place.

\section*{Extra Credit}
\paragraph{}Try to do the following embellishments:

\paragraph{Option 1:} After the printout of results, add an if-else block that tells the user if First-Year Intervention (FYI) notifications need to be sent, and if so, how many. To determine if any need to be sent, count how many Ds and Fs there are.

\paragraph{Option 2:} Ask the user for how many grades to simulate. Warn the user if they enter something less than 0. If the user enters 0, print the following to the screen: “Sorry to see you don’t want to use my simulator” with a newline at the end for proper display on screen.

\paragraph{Option 3:} Change the provided for loop to a while loop.

\section*{Grading Rubric}
\begin{itemize}
    \item Header comment: 2 points
    \item Body comments: 3 points
    \item Compiles: 30 points
    \item Proper logic: 55 points
    \item typescript: 10 points
    \item EC1: +5 points
    \item EC2: +5 points
    \item EC3: +5 points
\end{itemize}

\section*{What to Submit}
\paragraph{}Use the \texttt{script} command to record yourself compiling and running your program 3 times. (Do not record yourself editing your program!) Exit from script. Submit your program and the typescript file.
\begin{verbatim}
[rzak1@linux1 cw8]$ submit cmsc104_10600 cw8 grades.c typescript
\end{verbatim}

\subsection*{Verify}
\paragraph{}Make sure you submitted the assignment correctly.
\begin{verbatim}
[rzak1@linux1 cw8]$ submitls cmsc104_10600 cw8
\end{verbatim}
\end{document}