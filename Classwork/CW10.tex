% Copyright 2002-2024 The University of Maryland Baltimore County (UMBC)
% 1000 Hilltop Circle, Baltimore, Maryland, 21250, USA
% https://www.csee.umbc.edu/

\documentclass[letter,11pt]{article}
\usepackage[breaklinks]{hyperref}
\hypersetup{
    bookmarks=true,         % show bookmarks bar?
    unicode=false,          % non-Latin characters in Acrobat’s bookmarks
    pdftoolbar=true,        % show Acrobat’s toolbar?
    pdfmenubar=true,        % show Acrobat’s menu?
    pdffitwindow=false,     % window fit to page when opened
    pdfstartview={XYZ null null 1.00},    % disable zoom
    pdftitle={Classwork 10},    % title
    pdfauthor={Richard Zak},     % author
    pdfsubject={UMBC CMSC104 Problem Solving and Computer Programming},   % subject of the document
    pdfkeywords={Computer Science, Programming, Problem Solving, CSEE}, % list of keywords
    pdfnewwindow=true,      % links in new PDF window
    colorlinks=false,       % false: boxed links; true: colored links
    linkcolor=red,          % color of internal links (change box color with linkbordercolor)
    citecolor=green,        % color of links to bibliography
    filecolor=magenta,      % color of file links
    urlcolor=cyan           % color of external links
}
\usepackage{graphicx}
\usepackage{fancyhdr}
\usepackage{multicol}
\pagestyle{fancy}
\usepackage[letterpaper, margin=1in]{geometry}
\geometry{letterpaper}
\usepackage{parskip} % Disable initial indent
\usepackage{color,soul} % Highligher
\usepackage[normalem]{ulem} % Strikethrough with \sout{}
\usepackage{attachfile} % attach files

\usepackage[utf8]{inputenc}
\fancyhf{}
\renewcommand{\headrulewidth}{0pt} % Remove default underline from header package
\rhead{CMSC 104 Section 01: Classwork 10}
%\rhead{}
\lhead{\begin{picture}(0,0) \put(0,-10){\includegraphics[width=1.1cm]{Images/UMBC-vertical}} \end{picture}}
\cfoot{\thepage}
\rfoot{\input{semester}}
\lfoot{CMSC 104 Section 01}
\AtEndDocument{\vfill \footnotesize{Last modified: 08 February 2023}}
\AtEndDocument{\rfoot{\input{semester}}}
\renewcommand\thesubsection{\arabic{subsection}} % Show only subsection numbers, not section.subsection

\begin{document}

\huge
\textbf{Classwork 10: Mode}
\normalsize
\\ ~~ \\
\textbf{In-class Date: Tuesday 07 May} \\
\textbf{Due Date: Monday 13 May} \textit{May be a day or two late due to the final exam.}

\section*{Objectives}
\paragraph{}To practice working with arrays.

\section*{Assignment}
\paragraph{Your assignment is to compute the mode of a sequence of scores stored in a file. To read from the file, you will use input redirection, which can be done as follows:}
\begin{verbatim}
    [rzak1@linux1 cw10]$ ./a.out < fileToReadFrom.txt
\end{verbatim}

\paragraph{}Recall that the mode of a sequence of numbers, $\{x_1, x_2, \ldots, x_n\}$, is the value that appears the most (ties are broken arbitrarily). For example, the mode of $\{1, 7, 2, 2, 4, 2, 1, 3\}$ is 2, since the value 2 appears the most, 3 times.

\paragraph{}To determine the mode, you need to create a new array called \texttt{count} so that \texttt{count[i]} keeps track of the number of times that the score $i$ has appeared. Since we will assume that the scores are between 0 and 20 (inclusive), it is good to define a constant for this value:
\begin{verbatim}
    #define MAX_SCORE 20
\end{verbatim}

\paragraph{}To complete the program, you need to do the following:
\begin{enumerate}
    \item Declare a new \texttt{count[]} array.
    \item Initialize each element of \texttt{count[]} to 0.
    \item Iterate through the elements of the \texttt{A[]} array and update \texttt{count[]} in each iteration.
    \begin{enumerate}
        \item The array \texttt{A[]} holds the scores.
        \item For example, when your program sees that \texttt{A[34] = 11}, you should add 1 to \texttt{count[11]}, indicating another observation of that score.
    \end{enumerate}
    \item Iterate through the \texttt{count[]} array and print out the number of times that each score appeared.
    \item Iterate through the elements of the \texttt{count[]} array to find the maximum count and the score with the maximum count (the mode).
    \item Print out the score that has the highest count and the number of times that score appeared.
\end{enumerate}

\subsection*{Example Compilation and Execution}
\begin{verbatim}
    [rzak1@linux1 cw10]$ gcc -Wall mode.c
    [rzak1@linux1 cw10]$ ./a.out < cw10test1.txt
    count[0] is 0.
    count[1] is 1.
    count[2] is 2.
    count[3] is 0.
    count[4] is 0.
    count[5] is 0.
    count[6] is 0.
    count[7] is 1.
    count[8] is 1.
    count[9] is 1.
    count[10] is 0.
    count[11] is 1.
    count[12] is 2.
    count[13] is 1.
    count[14] is 0.
    count[15] is 1.
    count[16] is 2.
    count[17] is 2.
    count[18] is 1.
    count[19] is 2.
    count[20] is 3.
    The mode of the scores is 20. It occurred 3 times.
    [rzak1@linux1 cw10]$
\end{verbatim}

\subsection*{Starter Code}
\paragraph{}Use this code to help you get started. \textattachfile[color=1 0 0]{cw10mode.c}{mode.c}
\begin{verbatim}
/*****************************************
** File:  mode.c
** Author: <myName>
** Date:  <todaysDate>
** Section: CMSC104 Section 01
** E-mail: <myEmailAddress>
**
** This file contains the main program for Classwork 10.
** The program asks the user for an array of numbers and
** tells them the average and mode.
*****************************************/

#include <stdio.h>

#define MAX_SCORE 20
#define MAX_SIZE 1000

int main() {
   int i = 0;       // used as index into array
   int n = 0;       // number of items in array
   int r;           // used to ensure user enters an integer
   float average;   // average of numbers entered by user
   float num;       // total number of items in array

   int A[MAX_SIZE]; // will hold the scores

   n = 0;
   i = 0;

   // read in the scores into array A[]
   while(1) {
      r = scanf("%d", &A[i] );

      // end of input?
      if ( r <= 0 ) {
          break ;
      }

      i = i + 1;

      // exceeded array size?
      if (i >= MAX_SIZE) {
         break;
      }
   }

   // Make n the number of items stored in A[]
   n = i;

   // compute the average

   int sum = 0;

   for (i = 0 ; i < n ; i++) {
      sum = sum + A[i];
   }

   num = n;  // num is float
   average = sum / num;

   printf("The average score is: %f\n", average);

   // *****
   // ***** TODO
   // *****

   // Declare a new count[] array.

   // Initialize each element of count[] to 0. (Use a for loop.)

   // Iterate through the elements of the A[] array and update count[] in
   // each iteration.

   // Iterate through the count[] array and print out the number of times
   // that each score appeared.

   // Iterate through the elements of the count[] array to find the maximum
   // count and the score with the maximum count.

   // Print out the score that has the highest count and the number of
   // times that score appeared.

   return 0;
}
\end{verbatim}

\subsection*{Notes}
\begin{itemize}
    \item Remember that you need to declare constants, such as \texttt{MAX\_SCORE}.
    \item If the largest possible score is 20, how large do you have to make the \texttt{count[]} array?
    \begin{itemize}
        \item Hint: One larger than its highest value.
    \end{itemize}
    \item Once you have the \texttt{count[]} array computed, how do you find the largest element in it? For example, it is not enough to know that the largest count so far is 5, you also need to remember which score has appeared 5 times.
    \item Two input files have been prepared for you: ``cw10test1.txt'' and ``cw10test2.txt''. The first one is a small file with only 21 numbers that is useful for testing while you developer your program. The second file has 999 numbers.
    \begin{itemize}
        \item \textattachfile[color=1 0 0]{cw10test1.txt}{cw10test1.txt}
        \item \textattachfile[color=1 0 0]{cw10test2.txt}{cw10test2.txt}
    \end{itemize}
\end{itemize}

\section*{Extra Credit}
\paragraph{}Count how many invalid scores there are in the input file. Do this by adding a check for invalid scores in the loop you wrote for Step 3. Then, after printing the mode, print how many invalid scores there were, if any.

\section*{Grading Rubric}
\begin{itemize}
    \item Header comment: 2 points
    \item Body comments: 3 points
    \item Compiles: 40 points
    \item Uses redirection to test: 10 points
    \item Displays correct Count array: 25 points
    \item Finds mode: 15 points
    \item Lists how often mode occurred: 5 points
    \item EC: +5 points
\end{itemize}

\section*{What to Submit}
\paragraph{}Use the \texttt{script} command to record yourself compiling and running your programs 3 times, using different numbers each time. (Do not record yourself editing your program!) Exit from \texttt{script}. Submit your programs and the typescript file.

\begin{verbatim}
[rzak1@linux1 cw10]$ submit cmsc104_rzak1 cw10 mode.c typescript
\end{verbatim}

\subsection*{Verify}
\paragraph{}Make sure you submitted the assignment correctly.
\begin{verbatim}
[rzak1@linux1 cw10]$ submitls cmsc104_rzak1 cw10
\end{verbatim}

\end{document}
