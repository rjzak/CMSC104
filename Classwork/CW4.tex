% Copyright 2002-2024 The University of Maryland Baltimore County (UMBC)
% 1000 Hilltop Circle, Baltimore, Maryland, 21250, USA
% https://www.csee.umbc.edu/

\documentclass[letter,11pt]{article}
\usepackage[breaklinks]{hyperref}
\hypersetup{
    bookmarks=true,         % show bookmarks bar?
    unicode=false,          % non-Latin characters in Acrobat’s bookmarks
    pdftoolbar=true,        % show Acrobat’s toolbar?
    pdfmenubar=true,        % show Acrobat’s menu?
    pdffitwindow=false,     % window fit to page when opened
    pdfstartview={XYZ null null 1.00},    % disable zoom
    pdftitle={Classwork 4},    % title
    pdfauthor={Richard Zak},     % author
    pdfsubject={UMBC CMSC104 Problem Solving and Computer Programming},   % subject of the document
    pdfkeywords={Computer Science, Programming, Problem Solving, CSEE}, % list of keywords
    pdfnewwindow=true,      % links in new PDF window
    colorlinks=false,       % false: boxed links; true: colored links
    linkcolor=red,          % color of internal links (change box color with linkbordercolor)
    citecolor=green,        % color of links to bibliography
    filecolor=magenta,      % color of file links
    urlcolor=cyan           % color of external links
}
\usepackage{graphicx}
\usepackage{fancyhdr}
\usepackage{multicol}
\pagestyle{fancy}
\usepackage[letterpaper, margin=1in]{geometry}
\geometry{letterpaper}
\usepackage{parskip} % Disable initial indent
\usepackage{color,soul} % Highligher
\usepackage[normalem]{ulem} % Strikethrough with \sout{}

\usepackage[utf8]{inputenc}
\fancyhf{}
\renewcommand{\headrulewidth}{0pt} % Remove default underline from header package
\rhead{CMSC 104 Section 01: Classwork 4}
%\rhead{}
\lhead{\begin{picture}(0,0) \put(0,-10){\includegraphics[width=1.1cm]{Images/UMBC-vertical}} \end{picture}}
\cfoot{\thepage}
\rfoot{\input{semester}}
\lfoot{CMSC 104 Section 01}
\AtEndDocument{\vfill \footnotesize{Last modified: 08 February 2023}}
\AtEndDocument{\rfoot{\input{semester}}}
\renewcommand\thesubsection{\arabic{subsection}} % Show only subsection numbers, not section.subsection

\begin{document}

\huge
\textbf{Classwork 4: Input \& Output}
\normalsize
\\ ~~ \\
\textbf{In-class Date: Tuesday 27 February} \\
\textbf{Due Date: Monday 04 March}

\section*{Objectives}
\paragraph{}Practice using \texttt{printf()}, \texttt{scanf()}, and variables.

\section*{Assignment}
\paragraph{}Write a program that asks for the user's name and height in inches. Your program then replies with the user's name and height in centimeters. A sample run of your program should look like this:
\begin{verbatim}
[rzak1@linux1 cw4]$ gcc -Wall height.c 
[rzak1@linux1 cw4]$ ./a.out
What is your name? Batman
How tall are you in inches? 78
Hello, Batman. You are 198.12 centimeters tall.
[rzak1@linux1 cw4]$ 
\end{verbatim}

\paragraph{Reminder:} Assignments are an independent effort. \underline{This is not a group effort}. Assignments are checked to ensure they aren't too similar to that of other students'.

\section*{Notes}
\begin{enumerate}
    \item Login to GL and make sure you are in your home directory (\texttt{pwd}).
    \item Change directory to ``cw4'' (\texttt{cd cs104/cw4}) so you can do this assignment in the designated workspace.
    \item Type ``nano height.c'' to start the source code file for this assignment.
    \item Start with this source code: 
    \begin{verbatim}
/**************************************
** File:        height.c
** Author:      <studentName>
** Date:        <date>
** Section:     CMSC104 Section 1
** E-mail:      <username>@umbc.edu
**
** This file contains the main program for Classwork 4.
** The program reads in a name and height in inches
** from the user, greets them, and tells them their
** height in centimeters.
**************************************/

// What is the file we have to include for printf() and scanf()?

int main() {
   char name[20];                   /* User's name (no spaces) */
   <data_type> heightInInches;      /* User's height in inches */
   <data_type> heightInCentimeters; /* User's height in centimeters */

   // Prompt for user's name

   // Prompt for user's height in inches

   // Calculate how many centimeters from number of inches

   // Greet the user and tell them how many centimeters tall they are

   return 0;
}
    \end{verbatim}
    \item Fill in the necessary items:
    \begin{enumerate}
        \item Prior to ``int main()'', what is the line of code we need? Hint: begins with ``\#include''.
        \item For the two height variables, fill in the ``$<$data\_type$>$'' items with the appropriate data type. Is it an integer or float?
        \item Fill in the code block where you ask the user for their name, store the result.
        \item Fill in the code block where you ask the user for their height, store the result.
        \item Multiply the height in inches by 2.54, store as the height in centimeters variable.
        \item Fill in the code block where you present the user with the last line of text, showing their name and height in centimeters.
    \end{enumerate}
\end{enumerate}

\section*{Extra Credit}
\paragraph{}Use the appropriate code in \texttt{printf()} to only show two values after the decimal when printing the user's height in centimeters.

\section*{Grading Rubric}
\begin{itemize}
    \item Compiles: 50 points
    \item Accurate calculation: 40 points
    \item Typescript: 10 points
    \item Extra Credit: +5 points
\end{itemize}

\section*{What to Submit}
\paragraph{}Use the \texttt{script} command to record yourself compiling and running your program three times with different names and numbers. Do not record yourself editing the code!! Submit your code and ``typescript'' file.
\begin{verbatim}
    [rzak1@linux1 cw4]$ submit cmsc104_rzak1 cw4 height.c typescript
\end{verbatim}

\subsection*{Verify Submission}
\begin{verbatim}
    [rzak1@linux1 cw4]$ submitls cmsc104_rzak1 cw4
\end{verbatim}

\end{document}