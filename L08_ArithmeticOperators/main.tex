\documentclass[graphics]{beamer}
\usepackage{graphicx}
\usepackage{listings} % Syntax highlighing
\usepackage{fancyvrb} % Inline verbatim
\usepackage{hyperref} % Hyperlinks
\hypersetup{pdfpagemode=FullScreen}

\usepackage[normalem]{ulem}               % to striketrhourhg text
\newcommand\redout{\bgroup\markoverwith
{\textcolor{red}{\rule[0.5ex]{2pt}{0.8pt}}}\ULon}

% header in tables
\newcommand*{\thead}[1]{\multicolumn{1}{c}{\bfseries #1}}

% used for arrows from one point in the slide to another
\usepackage{tikz}
\usetikzlibrary{arrows,shapes,tikzmark}

\usetheme{Boadilla}
\title{Lecture 8: Arithmetic Operators}
\author{UMBC CMSC 104}
\date{Th \sout{23} 30 Sept 2021}

\begin{document}

\begin{frame}{}
\centering
    Arithmetic Operators
\end{frame}

\begin{frame}{Arithmetic Operators}
    Topics:
    \begin{itemize}
        \item \hyperlink{sec:athricop}{Arithmetic Operators}
        \item \hyperlink{sec:assignop}{Assignment Operators}
        \item \hyperlink{sec:evalarithexp}{Evaluating Arithmetic Expressions}
        \item \hyperlink{sec:opprec}{Operator Precedence}
        \item \hyperlink{sec:incprog}{Incremental Programming}
    \end{itemize}
\end{frame}

\section*{Arithmetic Operators}\label{sec:athricop}
\begin{frame}{Arithmetic Operators in C}
    \begin{itemize}
        \item Binary Operators:
        \begin{itemize}
            \item \texttt{new\_value = height + margin;}
            \item \texttt{area = length * width;}
        \end{itemize}
        \item Unary Operators:
        \begin{itemize}
            \item \texttt{new\_value = -old\_value;}
            \item \texttt{negation = !true\_value;}
        \end{itemize}
    \end{itemize}
\end{frame}

\begin{frame}{Arithmetic Operators in C}
    \begin{tabular}{l l l}
        Name & Operator & Example  \\ \hline
        Addition & + & num1 + num2 \\
        Subtraction & - & initial - spent \\
        Multiplication & * & fathoms * 6 \\
        Division & / & sum / count \\
        Modulus & \% & m\%n
    \end{tabular}
\end{frame}

\section*{Assignment Operators}\label{sec:assignop}
\begin{frame}{Types \& Promotion}
    \begin{itemize}
        \item Can mix types in numerical expressions
        \item Hierarchy of types:
        \begin{itemize}
            \item By precision: int $\rightarrow$ float
            \item By size: short $\rightarrow$ long
        \end{itemize}
        \item Lower size or precision is \textit{promoted} to greater size or precision before operation is applied.
        \item Result is also of the promoted type.
    \end{itemize}
\end{frame}

\begin{frame}[fragile]{Types \& Promotion}
\begin{verbatim}
int num_sticks = 5;
double avg_stick_length = 4.5;
double total_length;

total_length = num_sticks * avg_stick_length;
\end{verbatim} ~~ \\
\texttt{num\_sticks} would be converted to double precision, then multiplied by \texttt{avg\_stick\_length}.
\end{frame}

\section*{Evaluating Arithmetic Expressions}\label{sec:evalarithexp}
\begin{frame}{Division}
    \only<1> {
        \begin{itemize}
            \item If both operands of a division expression are integers, you will get an integer answer. Fractional portions are discarded.
            \item Examples:
            \begin{itemize}
                \item $17/5 = 3$
                \item $4/3 = 1$
                \item $35/9 = 3$
            \end{itemize}
        \end{itemize}
    }
    \only<2> {
        \begin{itemize}
            \item Division where at least one operand is a floating point number will produce a floating point answer.
            \item Examples:
            \begin{itemize}
                \item $17.0 / 5 = 3.4$
                \item $4/3.2 = 1.25$
                \item $35.2/9.1 = 3.86813$
            \end{itemize}
            \item What happens? The integer operand is temporarily converted to a floating point, then the division is performed.
        \end{itemize}
    }
    \only<3> {
        Given: \texttt{int my\_integer = 5, my\_product;}, what is \texttt{my\_product}>
        \begin{itemize}
            \item \texttt{my\_product = (my\_integer / 2) * 2.0;}
            \item \texttt{my\_product = (my\_integer / 2.0) * 2.0;}
        \end{itemize}
    }
\end{frame}


\begin{frame}{Division by Zero}
    \begin{itemize}
        \item Division by zero is mathematically undefined.
        \item If you have division by zero in a program, it will cause a \textbf{fatal error}. Your program will terminate execution and give an error message.
        \item \textbf{Non-fatal} errors do not cause program termination, just incorrect results.
    \end{itemize}
    
    \vfill
    \footnotesize{Technically, it's the operating system which terminates the program and shows the error message.}
\end{frame}

\begin{frame}{Modulus}
    \begin{itemize}
        \item The expression \texttt{m\%n} yields the integer remainer after \texttt{m} is divided by \texttt{n}.
        \item Modulus is an integer operation -- both operands must be integers.
        \item Examples:
        \begin{itemize}
            \item 17 \% 5 = 2
            \item 6 \% 3 = 0
            \item 9 \% 2 = 1
            \item 5 \% 8 = 5
        \end{itemize}
    \end{itemize}
\end{frame}

\begin{frame}{Uses for Modulus}
    \begin{itemize}
        \item Used to determine if an integer is even or odd:
        \begin{itemize}
            \item \texttt{5 \% 2 == 1 /* odd */}
            \item \texttt{4 \% 2 == 0 /* even */}
        \end{itemize}
        \item Modulus by 2 of an integer results in a 1 if odd, or zero if even.
        \item It's used to see if some number is divisible by another.
        \begin{itemize}
            \item \texttt{25 \% 5 == 0 /* divisible by 5 */}
            \item \texttt{49 \% 7 == 0 /* divisible by 7 */}
        \end{itemize}
    \end{itemize}
\end{frame}

\section*{Operator Precedence}\label{sec:opprec}
\begin{frame}{Rules of Operator Precedence}
    \begin{tabular}{l l}
        Operator(s) & Precedence \& Associativity \\ \hline
        ( ) & Evaluated first. If \textbf{nested (embedded)}, innermost first. \\
        & If on the same level, left to right. \\
        * / \% & Evaluated second. If there are several, evaluated left to right. \\
        + - & Evaluated third. If there are several, evaluated left to right. \\
        = & Evaluated latest, right to left.
    \end{tabular}
\end{frame}

\begin{frame}{Using Parentheses}
    \begin{itemize}
        \item Use parentheses to change the order in which an expression is evaluated. \\
        ~~ ~~ ~~ ~~ ~~ $a + b * c$: \\ 
        $b*c$ first, then $a$ is added
        \item If you wanted $a + b$ to be done first: \\
        ~~ ~~ ~~ ~~ ~~ $(a + b) * c$
        \item Also use parentheses to clarify a complex expression, which improves readability.
    \end{itemize}
\end{frame}

\begin{frame}{Practice with Evaluating Expressions}
    Given: \texttt{int a = 1, b = 2, c = 3, d = 4;}, evaluate:
    \begin{itemize}
        \item $a + b - d + d$
        \item $a * b / c$
        \item $1 + a * b \% c$
        \item $e = b = d + c / b - a$
    \end{itemize}
\end{frame}

\section*{Incremental Programming}\label{sec:incprog}
\begin{frame}{Good Programming Practice}
    \begin{itemize}
        \item It's best not to take the \textbf{``big bang'' approach} to coding.
        \item Use an \textbf{incremental approach} by writing your code in incomplete, yet working, pieces.
        \item For example, for your projects:
        \begin{itemize}
            \item Don't write the whole program at once.
            \item Just write enough to display the user prompt on the screen.
            \item Get that part working first (compile and run).
            \item Next, write the part that gets the value from the user, and just print it out.
            \item Get that part working first (compile and run).
            \item Next, change the code so you use the value in a calculation. Print answer.
            \item Get that part working first (compile and run).
            \item Repeat until the requirements are complete.
        \end{itemize}
        \item This also helps you find any semantic errors sooner.
        \item Always have a working version of your program!
    \end{itemize}
\end{frame}

\end{document}